\chapter{Codes for FD from simulated data}

\lstset{frame=tb,
  language=Matlab,
  aboveskip=3mm,
  belowskip=3mm,
  showstringspaces=false,
  columns=flexible,
  basicstyle={\small\ttfamily},
  numbers=none,
  numberstyle=\tiny\color{gray},
  keywordstyle=\color{blue},
  commentstyle=\color{dkgreen},
  stringstyle=\color{mauve},
  breaklines=true,
  breakatwhitespace=true,
  tabsize=3
}

\section{Read Me file for the aircraft simulation codes in Matlab}

HOW TO: 
\begin{enumerate}
  \item run simDrone.m 
  \item If you want to change aircraft parameters, change them from configDrone.m.
  \item If you want to change simulation time, change $sim\_duration\_min$ in simDrone.m
\end{enumerate}

ASSUMPTIONS: 

\begin{enumerate}
  \item NED (North East Down) navigation frame is inertial where Newton's law apply.
  \item Attitude sequence considered is Yaw-Pitch-Roll. And yes sequence matters!
\end{enumerate}


INFO FOR BEGINNERS: 
Sequential Euler angle rotations relate the orientation of the aircrafts body-fixed frame to the navigation frame. For simulations quaternion preferred due to singularity issues of Euler angles. It is proven that attitude representations will either have a redundant component (4 components) as in quaternions or singularity (Euler angles have 3 components resulting in singularity for \\
pitch (theta) = 90 degrees - devision by 0). Quaternion representation used here has the scalar component as the first component: q = q0 + q1 * i + q2 * j + q3 k

w $\triangleq$ angular velocity vector with components p, q, r w = [p q r]' in detail, w describes the angular motion of the body frame b with respect to navigation frame n (NED), expressed in body frame.

Attitude transformation matrix (Direction Cosine Matrix) is used to change the frame of interest that the vector or points expressed in. Lets say that we have a vector A fixed in inertial frame. Its representation in two frames will differ even it is the same vector. The frame to express it could be changed utilizing a direction cosine matrix. In this work, to express vectors in different frames is necessary which makes the use of DCM essential. $C_n^b$ transforms the vector A expressed in the navigation frame $A^n$ into $A^b$, a vector expressed in the drone body-fixed frame. $A^b = C_n^b * A^n$ -----> in the code : $c\_n\_to\_b$ Likewise a direction cosine matrix $C_b^n$, changes the representation of vector A expressed in drone body-fixed frame $A^b$, to a representation of the same vector $A$ in navigation frame(NED) $A^n$. $A^n = C_b^n * A^b$ and beware the relationship between these to transformation: $C_b^n = inverse(C_n^b) = transpose(C_n^b)$

TIPS AND TRICKS: Forces and Moments in the dynamical equations of motion for the attitude and translational motion, are calculated in modelDrone. During the numerical integration of dynamic and kinematic equations (more specifically during the function evaluations for calculating k1 k2 k3 k4), modelDrone is evaluated with different states (due to the nature of RK4). Since forces and moments are calculated in this model, beware that they are evaluated with different state values, since they are modeled as a function of states, ending up the change of forces and moments during one time step of numerical integration.

\clearpage
\newpage

\section{configDrone.m}
\begin{lstlisting}
% Copyright 2016 Elgiz Baskaya

% This file is part of curedRone.

% curedRone is free software: you can redistribute it and/or modify
% it under the terms of the GNU General Public License as published by
% the Free Software Foundation, either version 3 of the License, or
% (at your option) any later version.
% 
% curedRone is distributed in the hope that it will be useful,
% but WITHOUT ANY WARRANTY; without even the implied warranty of
% MERCHANTABILITY or FITNESS FOR A PARTICULAR PURPOSE.  See the
% GNU General Public License for more details.
% 
% You should have received a copy of the GNU General Public License
% along with curedRone.  If not, see <http://www.gnu.org/licenses/>.
clear all;
clc;

global g_e mass inert wing_tot_surf wing_span m_wing_chord prop_dia nc tho_n cl_alpha1 cl_ele1 cl_p 
global cl_r cl_beta cm_1 cm_alpha1 cm_ele1 cm_q cm_alpha cn_rud_contr cn_r_tilda cn_beta cx1 
global cx_alpha cx_alpha2 cx_beta2 cz_alpha cz1 cy1 cft1 cft2 cft3

% Earth gravitational constant
g_e = 9.81;

% UAV mass [kg]
mass = 28;

% UAV inertia matrix [kg*m^2]
inert = [2.56 0 0.5; 0 10.9 0; 0.5 0 11.3];

% wing total surface S [m^2]
wing_tot_surf = 1.8;

% wing span b [m]
wing_span = 3.1;

% mean aerodynamic wing chord c [m]
m_wing_chord = 0.58;

% diameter of the propeller prop_dia [m]
prop_dia = 0.79;

% engine speed reference signal nc 
nc = 80;

% time constant of the engine tho_n [s]
tho_n = 0.4;

% roll derivatives 
cl_alpha1 = - 3.395e-2;% cl_alpha2 = - clalpha1
cl_ele1 = - 0.485e-2;% cl_ele2   = - clele1
cl_p = - 1.92e-1;
cl_r = 3.61e-2;
cl_beta = - 1.3e-2;

% pitch derivatives
cm_1 = 2.08e-2;
cm_alpha1 = 0.389e-1;% cmalpha2 = cmalpha1
cm_ele1 = 2.725e-1;% cmele2 = cmele1
cm_q = -9.83;
cm_alpha = -9.03e-2;

% yaw derivatives
cn_rud_contr = 5.34e-2;
cn_r_tilda = -2.14e-1;
cn_beta = 8.67e-2;

% lift, drag, side force derivatives
cx1 = -2.12e-2;
cx_alpha = -2.66e-2;
cx_alpha2 = -1.55;
cx_beta2 = -4.01e-1;
cz_alpha = -3.25;
cz1 = 1.29e-2;
cy1 = -3.79e-1;

% thrust derivatives 
cft1 = 8.42e-2;
cft2 = -1.36e-1;
cft3 = -9.28e-1;

\end{lstlisting}

\clearpage
\newpage

\section{modelDrone.m}
\begin{lstlisting}
% Copyright 2016 Elgiz Baskaya

% This file is part of curedRone.

% curedRone is free software: you can redistribute it and/or modify
% it under the terms of the GNU General Public License as published by
% the Free Software Foundation, either version 3 of the License, or
% (at your option) any later version.
% 
% curedRone is distributed in the hope that it will be useful,
% but WITHOUT ANY WARRANTY; without even the implied warranty of
% MERCHANTABILITY or FITNESS FOR A PARTICULAR PURPOSE.  See the
% GNU General Public License for more details.
% 
% You should have received a copy of the GNU General Public License
% along with curedRone.  If not, see <http://www.gnu.org/licenses/>.

% inputs : stateInitial .:. States from the previous time t - 1. 
%          VT .:. total airspeed of the aircraft
%          conAil1, conAil2, conEle1,conEle2, conRud .:. controlTorques 

% Attitude kinematic and dynamic equations of motion
% Translational Motion 

function state_dot = modelDrone(state_prev, contr_deflect, wind_ned)

global g_e inert mass wing_tot_surf wing_span m_wing_chord prop_dia cl_alpha1 cl_ele1 cl_p 
global cl_r cl_beta cm_1 cm_alpha1 cm_ele1 cm_q cm_alpha cn_rud_contr cn_r_tilda cn_beta cx1 
global cx_alpha cx_alpha2 cx_beta2 cz_alpha cz1 cy1 cft1 cft2 cft3 tho_n nc

quat_normalize_gain = 1;

% q .:. quaternion
% q = q0 + q1 * i + q2 * j + q3 * k;  
q0 = state_prev(1);
q1 = state_prev(2);
q2 = state_prev(3);
q3 = state_prev(4);

% w .:. angular velocity vector with components p, q, r
% w = [p q r]' 
% w describes the angular motion of the body frame b with respect to
% navigation frame ned, expressed in body frame.
p = state_prev(5);
q = state_prev(6);
r = state_prev(7);

% x .:. position of the drone in North East Down reference frame
% x = [x_n y_e z_d]';
% x_n = state_prev(8);
% y_e = state_prev(9);
% z_d = state_prev(10);

% v .:. translational velocity of the drone
% v = [u_b v_b w_b]
u_b = state_prev(11);
v_b = state_prev(12);
w_b = state_prev(13);

% Engine speed
eng_speed = state_prev(14);

% Flight altitude
altitude = state_prev(10);

% Control surface deflections
con_ail1 = contr_deflect(1);
con_ail2 = contr_deflect(2);
con_ele1 = contr_deflect(3);
con_ele2 = contr_deflect(4);
con_rud  = contr_deflect(5);

% If A is any vector
% A^n = C^n_b * A^b = c_b_to_n * A^b = inv(c_n_to_b) * A^b = c_n_to_b' * A^b
% Direction cosine matrix C^b_n representing the transformation from
% the navigation frame to the body frame
c_n_to_b = [1 - 2 * (q2^2 + q3^2) 2 * (q1 * q2 + q0 * q3) 2 * (q1 * q3 - q0 * q2); ...
2 * (q1 * q2 - q0 * q3) 1 - 2 * (q1^2 + q3^2) 2 * (q2 * q3 + q0 * q1); ...
2 * (q1 * q3 + q0 * q2) 2 * (q2 * q3 - q0 * q1) 1 - 2 * (q1^2 + q2^2)];

vel_t = [u_b; v_b; w_b] - c_n_to_b * wind_ned;

% Total airspeed of drone
vt = sqrt(vel_t(1)^2 + vel_t(2)^2 + vel_t(3)^2);

% alph .:. angle of attack
alph = atan2(vel_t(3),vel_t(1));

% bet .:. side slip angle
bet  = asin(vel_t(2)/vt);

% Low altitude atmosphere model (valid up to 11 km)
% t0 = 288.15; % Temperature [K]
% a_ = - 6.5e-3; % [K/m]
% r_ = 287.3; % [m^2/K/s^2] 
% p0 = 1013e2; %[N/m^2]
% t_= t0 * (1 + a_ * altitude / t0);
% ro = p0 * (1 + a_ * altitute /t0)^5.2561 / r_ / t_;
t_ = 288.15 * (1 - 6.5e-3 * altitude / 288.15);
ro = 1013e2 * (1 - 6.5e-3 * altitude / 288.15)^5.2561 / 287.3 / t_;
dyn_pressure = ro * vt^2 / 2;

p_tilda = wing_span * p / 2 / vt;
r_tilda = wing_span * r / 2 / vt;
q_tilda = m_wing_chord * q / 2 / vt;

% calculation of aerodynamic derivatives
% (In the equations % CLalpha2 = - CLalpha1 and so on used not to inject new names to namespace)
cl = cl_alpha1 * con_ail1 - cl_alpha1 * con_ail2 + cl_ele1 * con_ele1 - cl_ele1 * con_ele2 ...
+ cl_p * p_tilda + cl_r * r_tilda + cl_beta * bet;
cm = cm_1 + cm_alpha1 * con_ail1 + cm_alpha1 * con_ail2 + cm_ele1 * con_ele1 + cm_ele1 * con_ele2 ...
+ cm_q * q_tilda + cm_alpha * alph;
cn = cn_rud_contr * con_rud + cn_r_tilda * r_tilda + cn_beta * bet;

l = dyn_pressure * wing_tot_surf * wing_span * cl;
m = dyn_pressure * wing_tot_surf * m_wing_chord * cm;
n = dyn_pressure * wing_tot_surf * wing_span * cn;

moment = [l m n]';

% tilda is to ignore output of the quat2angle function, since it is not
% used, a warning appears otherwise
[~, teta fi] = quat2angle([q0 q1 q2 q3]);

% ft .:. thrust force 
ft = ro * eng_speed^2 * prop_dia^4 * (cft1 + cft2 * vt / prop_dia / pi / eng_speed + ...
									  cft3 * vt^2 / prop_dia^2 / pi^2 / eng_speed^2);
% Model of the aerodynamic forces in wind frame
% xf_w .:. drag force in wind frame
xf_w = dyn_pressure * wing_tot_surf * (cx1 + cx_alpha * alph + cx_alpha2 * alph^2 + ...
									   cx_beta2 * bet^2);
% yf_w .:. lateral force in wind frame
yf_w = dyn_pressure * wing_tot_surf * (cy1 * bet);
% zf_w .:. lift force in wind frame
zf_w = dyn_pressure * wing_tot_surf * (cz1 + cz_alpha * alph);

% describe forces in body frame utilizing rotation matrix c^w_b
% A^w = C^w_b * A^b     OR     A^b = C^b_w * A^w = (C^w_b)' * A^w  here
% c_b_to_w = C^w_b
c_b_to_w = [cos(alph)* cos(bet) sin(bet) sin(alph) * cos(bet);...
-sin(bet) * cos(alph) cos(bet) -sin(alph) * sin(bet); -sin(alph) 0 cos(alph)];

force = mass * [- g_e * sin(teta); g_e * sin(fi) * cos(teta); g_e * cos(fi) * cos(teta)] +...
    ([ft; 0; 0] + c_b_to_w' * [xf_w; yf_w; zf_w]);

% Kinematic and dynamic equations of motion of the drone

% Attitude dynamics of drone
% Skew symmetric matrix is used for cross product
pqr_dot = inert \ (moment - [ 0 -r q; r 0 -p; -q p 0] * (inert * [p q r]'));

% Attitude kinematics of drone
q_dot = 1 / 2 * [-q1 -q2 -q3; q0 -q3 q2; q3 q0 -q1; -q2 q1 q0] * [p q r]' ...
    + quat_normalize_gain * (1 - (q0^2 + q1^2 + q2^2 + q3^2)) * [q0 q1 q2 q3]';

% x_dot is in NED frame. So a change in expression of v is needed.
x_dot = c_n_to_b' * [u_b v_b w_b]';

% dynamics for translational motion of the center of mass of the drone
v_dot = force / mass - [(q * w_b - r * v_b); (r * u_b - p * w_b); (p * v_b - q * u_b)];

% dynamics for engine speed
eng_speed_dot = - 1 / tho_n * eng_speed + 1 / tho_n * nc;

state_dot = [q_dot; pqr_dot; x_dot; v_dot; eng_speed_dot];
\end{lstlisting}

\clearpage
\newpage

\section{quat\_to\_euler.m}
\begin{lstlisting}
% Copyright 2016 Elgiz Baskaya

% This file is part of curedRone.

% curedRone is free software: you can redistribute it and/or modify
% it under the terms of the GNU General Public License as published by
% the Free Software Foundation, either version 3 of the License, or
% (at your option) any later version.
% 
% curedRone is distributed in the hope that it will be useful,
% but WITHOUT ANY WARRANTY; without even the implied warranty of
% MERCHANTABILITY or FITNESS FOR A PARTICULAR PURPOSE.  See the
% GNU General Public License for more details.
% 
% You should have received a copy of the GNU General Public License
% along with curedRone.  If not, see <http://www.gnu.org/licenses/>.

function euler_ang = quat_to_euler(quater)

euler_ang = [atan2(2 .* (quater(3,:) .* quater(4,:) - quater(1,:) .* quater(2,:)), ...
    2 * ((quater(1,:)).^2  - 1 + (quater(4,:)).^2));...
    - atan((2 * (quater(2,:) .* quater(4,:) + quater(1,:) .* quater(3,:))) ./ sqrt(1 - (2 * (quater(2,:) .* quater(4,:) + quater(1,:) .* quater(3,:))).^2)); ...
    atan2(2 * (quater(2,:) .* quater(3,:) - quater(1,:) .* quater(4,:)), ...
    2 * ((quater(1,:)).^2 - 1 + 2 * (quater(2,:)).^2))];
    
 \end{lstlisting}
 
\clearpage
\newpage

\section{rungeKutta4.m}
\begin{lstlisting}
% Copyright 2016 Elgiz Baskaya

% This file is part of curedRone.

% curedRone is free software: you can redistribute it and/or modify
% it under the terms of the GNU General Public License as published by
% the Free Software Foundation, either version 3 of the License, or
% (at your option) any later version.
% 
% curedRone is distributed in the hope that it will be useful,
% but WITHOUT ANY WARRANTY; without even the implied warranty of
% MERCHANTABILITY or FITNESS FOR A PARTICULAR PURPOSE.  See the
% GNU General Public License for more details.
% 
% You should have received a copy of the GNU General Public License
% along with curedRone.  If not, see <http://www.gnu.org/licenses/>.

function xn = rungeKutta4(func, xo, cntrl, wind_ned, h)

k1 = feval(func, xo, cntrl, wind_ned);
k2 = feval(func, xo + 1/2 * h * k1, cntrl, wind_ned);
k3 = feval(func, xo + 1/2 * h * k2, cntrl, wind_ned);
k4 = feval(func, xo + h * k3, cntrl, wind_ned);

xn = xo + 1/6 * h * (k1 + 2 * k2 + 2 * k3 + k4);
end
 \end{lstlisting}
 
\clearpage
\newpage

\section{simDrone.m}
\begin{lstlisting}

% Copyright 2016 Elgiz Baskaya

% This file is part of curedRone.

% curedRone is free software: you can redistribute it and/or modify
% it under the terms of the GNU General Public License as published by
% the Free Software Foundation, either version 3 of the License, or
% (at your option) any later version.
% 
% curedRone is distributed in the hope that it will be useful,
% but WITHOUT ANY WARRANTY; without even the implied warranty of
% MERCHANTABILITY or FITNESS FOR A PARTICULAR PURPOSE.  See the
% GNU General Public License for more details.
% 
% You should have received a copy of the GNU General Public License
% along with curedRone.  If not, see <http://www.gnu.org/licenses/>.

% DRONE DYNAMICS SIMULATION

configDrone;

ti = 0.1;
sim_duration_min = 10;
tf = 60 * sim_duration_min;
t_s = 0 : ti : tf;

x_real = zeros(14,length(t_s));

% initial condition for the states
% xReal = [q0 q1 q2 q3 p q r x_n y_e z_d u_b v_b w_b eng_speed]';
x_real(:,1) = [1 0 0 0 0 0 0 0 0 0 1e-5 1e-5 1e-5 1e-2]';

control_deflections = [0 0 0 0 0]';
% controlTorque = [contAileron1 contAileron2 contElevator1 contElevator2
% contRudder]'

wind_ned = [0 0 0]';
% wind_ned .:. [wind_n wind_e wind_d]'

for i=1:length(t_s)-1
  % Nonlinear attitude propagation
  % Integration via Runge - Kutta integration Algorithm
  x_real(:,i+1) = rungeKutta4('modelDrone', x_real(:,i), control_deflections, wind_ned, ti); 
end

 \end{lstlisting}