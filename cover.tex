% -*-latex-*-
% 
% For questions, comments, concerns or complaints:
% thesis@mit.edu
% 
%
% $Log: cover.tex,v $
% Revision 1.8  2008/05/13 15:02:15  jdreed
% Degree month is June, not May.  Added note about prevdegrees.
% Arthur Smith's title updated
%
% Revision 1.7  2001/02/08 18:53:16  boojum
% changed some \newpages to \cleardoublepages
%
% Revision 1.6  1999/10/21 14:49:31  boojum
% changed comment referring to documentstyle
%
% Revision 1.5  1999/10/21 14:39:04  boojum
% *** empty log message ***
%
% Revision 1.4  1997/04/18  17:54:10  othomas
% added page numbers on abstract and cover, and made 1 abstract
% page the default rather than 2.  (anne hunter tells me this
% is the new institute standard.)
%
% Revision 1.4  1997/04/18  17:54:10  othomas
% added page numbers on abstract and cover, and made 1 abstract
% page the default rather than 2.  (anne hunter tells me this
% is the new institute standard.)
%
% Revision 1.3  93/05/17  17:06:29  starflt
% Added acknowledgements section (suggested by tompalka)
% 
% Revision 1.2  92/04/22  13:13:13  epeisach
% Fixes for 1991 course 6 requirements
% Phrase "and to grant others the right to do so" has been added to 
% permission clause
% Second copy of abstract is not counted as separate pages so numbering works
% out
% 
% Revision 1.1  92/04/22  13:08:20  epeisach

% NOTE:
% These templates make an effort to conform to the MIT Thesis specifications,
% however the specifications can change.  We recommend that you verify the
% layout of your title page with your thesis advisor and/or the MIT 
% Libraries before printing your final copy.
\title{Fault Detection\&Diagnosis for Drones using Machine Learning}

\author{Elgiz Baskaya}
% If you wish to list your previous degrees on the cover page, use the 
% previous degrees command:
%       \prevdegrees{A.A., Harvard University (1985)}
% You can use the \\ command to list multiple previous degrees
%       \prevdegrees{B.S., University of California (1978) \\
%                    S.M., Massachusetts Institute of Technology (1981)}
\department{Applied Mathematics}

% If the thesis is for two degrees simultaneously, list them both
% separated by \and like this:
% \degree{Doctor of Philosophy \and Master of Science}
\degree{Doctor of Philisophy in Aeronautics and Astronautics}

% As of the 2007-08 academic year, valid degree months are September, 
% February, or June.  The default is June.
\degreemonth{February}
\degreeyear{2018}
\thesisdate{February 22, 2018}

%% By default, the thesis will be copyrighted to MIT.  If you need to copyright
%% the thesis to yourself, just specify the `vi' documentclass option.  If for
%% some reason you want to exactly specify the copyright notice text, you can
%% use the \copyrightnoticetext command.  
%\copyrightnoticetext{\copyright IBM, 1990.  Do not open till Xmas.}

% If there is more than one supervisor, use the \supervisor command
% once for each.
\supervisor{Daniel Delahaye}{Professor}
\supervisor{Murat Bronz}{Assistant Professor}

% This is the department committee chairman, not the thesis committee
% chairman.  You should replace this with your Department's Committee
% Chairman.
\chairman{Arthur C. Smith}{Chairman, Department Committee on Graduate Theses}

% Make the titlepage based on the above information.  If you need
% something special and can't use the standard form, you can specify
% the exact text of the titlepage yourself.  Put it in a titlepage
% environment and leave blank lines where you want vertical space.
% The spaces will be adjusted to fill the entire page.  The dotted
% lines for the signatures are made with the \signature command.
\maketitle

% The abstractpage environment sets up everything on the page except
% the text itself.  The title and other header material are put at the
% top of the page, and the supervisors are listed at the bottom.  A
% new page is begun both before and after.  Of course, an abstract may
% be more than one page itself.  If you need more control over the
% format of the page, you can use the abstract environment, which puts
% the word "Abstract" at the beginning and single spaces its text.

%% You can either \input (*not* \include) your abstract file, or you can put
%% the text of the abstract directly between the \begin{abstractpage} and
%% \end{abstractpage} commands.

% First copy: start a new page, and save the page number.
\cleardoublepage
% Uncomment the next line if you do NOT want a page number on your
% abstract and acknowledgments pages.
% \pagestyle{empty}
\setcounter{savepage}{\thepage}
\begin{abstractpage}
% $Log: abstract.tex,v $
% Revision 1.1  93/05/14  14:56:25  starflt
% Initial revision
% 
% Revision 1.1  90/05/04  10:41:01  lwvanels
% Initial revision
% 
%
%% The text of your abstract and nothing else (other than comments) goes here.
%% It will be single-spaced and the rest of the text that is supposed to go on
%% the abstract page will be generated by the abstractpage environment.  This
%% file should be \input (not \include 'd) from cover.tex.

This new era of small UAVs currently populating the airspace introduces many safety concerns, due to the absence of a pilot onboard and the less accurate nature of the sensors. 
This necessitates intelligent approaches to address the emergency situations that will inevitably arise for all classes of UAV operations as defined by EASA (European Aviation Safety Agency). 
Hardware limitations for these small vehicles point to the utilization of \emph{analytical redundancy}, rather than to the usual practice of hardware redundancy in manned aviation. 
In the course of this study, machine learning practices are implemented in order to diagnose faults on a small fixed-wing UAV to avoid the burden of accurate modeling needed in model-based fault diagnosis. 
A supervised classification method, SVM (Support Vector Machines) is used to classify the faults. 
The data used to diagnose the faults are gyro and accelerometer measurements. 
The idea to restrict the data set to accelerometer and gyro measurements is to check the method's classification ability, with a small and inexpensive chip set and without the need to access the data from the autopilot, such as the control input information. 
% To access information from the autopilot used was quite easy since it is open-source, but still a solution for a wider community is pursued involving the systems equipped with closed solutions. 
% Although serving for flexibility, lacking control input information from the autopilot introduces challenges mainly due to lack of knowledge about the controller's compensation for the fault. 
This work addresses the faults in the control surfaces of a UAV. 
More specifically, the faults considered are the control surface stuck at an angle and the loss of effectiveness. 
% Since SVM is a supervised machine learning algorithm, labeled data is needed to accomplish the training of the classifier. 
First, a model of an aircraft is simulated. This model is not used for the design of Fault Detection and Diagnosis (FDD) algorithms, but is instead utilized to generate data. 
Simulated data are used instead of flight data in order to isolate the probable effects of the controller on the diagnosis, which may complicate a preliminary study on FDD for drones.
The results show that for simulated measurements, SVM gives very accurate results on the classification of the loss of effectiveness faults on the control surfaces. 
These promising results call for further investigation so as to assess SVM performance on fault classification with flight data.
Real flights were arranged to generate faulty flight data by manipulating the open source autopilot, \emph{Paparazzi}. 
All data and the code are available in the code sharing and versioning system, \emph{Github}. 
Training is held offline due to the need for labeled data and the computational burden of the tuning phase of the classifiers. 
Results show that from the flight data, SVM yields an F1 score of 0.98 for the classification of control surface stuck faults.
For the loss of efficiency faults, some feature engineering, involving the addition of past measurements is needed in order to attain the same classification performance. 
A promising result is discovered when \emph{spinors} are used as features instead of angular velocities. 
Results show that by using spinors for classification, there is a vast improvement in classification accuracy, especially when the classifiers are untuned. Using spinors and a Gaussian Kernel, an untuned classifier gives an F1 score of 0.9555, which was 0.2712 when gyro measurements were used as features.
In summary, this work shows that SVM gives a satisfactory performance for the classification of faults on the control surfaces of a drone using flight data.

\end{abstractpage}

% Additional copy: start a new page, and reset the page number.  This way,
% the second copy of the abstract is not counted as separate pages.
% Uncomment the next 6 lines if you need two copies of the abstract
% page.
% \setcounter{page}{\thesavepage}
% \begin{abstractpage}
% % $Log: abstract.tex,v $
% Revision 1.1  93/05/14  14:56:25  starflt
% Initial revision
% 
% Revision 1.1  90/05/04  10:41:01  lwvanels
% Initial revision
% 
%
%% The text of your abstract and nothing else (other than comments) goes here.
%% It will be single-spaced and the rest of the text that is supposed to go on
%% the abstract page will be generated by the abstractpage environment.  This
%% file should be \input (not \include 'd) from cover.tex.

This new era of small UAVs currently populating the airspace introduces many safety concerns, due to the absence of a pilot onboard and the less accurate nature of the sensors. 
This necessitates intelligent approaches to address the emergency situations that will inevitably arise for all classes of UAV operations as defined by EASA (European Aviation Safety Agency). 
Hardware limitations for these small vehicles point to the utilization of \emph{analytical redundancy}, rather than to the usual practice of hardware redundancy in manned aviation. 
In the course of this study, machine learning practices are implemented in order to diagnose faults on a small fixed-wing UAV to avoid the burden of accurate modeling needed in model-based fault diagnosis. 
A supervised classification method, SVM (Support Vector Machines) is used to classify the faults. 
The data used to diagnose the faults are gyro and accelerometer measurements. 
The idea to restrict the data set to accelerometer and gyro measurements is to check the method's classification ability, with a small and inexpensive chip set and without the need to access the data from the autopilot, such as the control input information. 
% To access information from the autopilot used was quite easy since it is open-source, but still a solution for a wider community is pursued involving the systems equipped with closed solutions. 
% Although serving for flexibility, lacking control input information from the autopilot introduces challenges mainly due to lack of knowledge about the controller's compensation for the fault. 
This work addresses the faults in the control surfaces of a UAV. 
More specifically, the faults considered are the control surface stuck at an angle and the loss of effectiveness. 
% Since SVM is a supervised machine learning algorithm, labeled data is needed to accomplish the training of the classifier. 
First, a model of an aircraft is simulated. This model is not used for the design of Fault Detection and Diagnosis (FDD) algorithms, but is instead utilized to generate data. 
Simulated data are used instead of flight data in order to isolate the probable effects of the controller on the diagnosis, which may complicate a preliminary study on FDD for drones.
The results show that for simulated measurements, SVM gives very accurate results on the classification of the loss of effectiveness faults on the control surfaces. 
These promising results call for further investigation so as to assess SVM performance on fault classification with flight data.
Real flights were arranged to generate faulty flight data by manipulating the open source autopilot, \emph{Paparazzi}. 
All data and the code are available in the code sharing and versioning system, \emph{Github}. 
Training is held offline due to the need for labeled data and the computational burden of the tuning phase of the classifiers. 
Results show that from the flight data, SVM yields an F1 score of 0.98 for the classification of control surface stuck faults.
For the loss of efficiency faults, some feature engineering, involving the addition of past measurements is needed in order to attain the same classification performance. 
A promising result is discovered when \emph{spinors} are used as features instead of angular velocities. 
Results show that by using spinors for classification, there is a vast improvement in classification accuracy, especially when the classifiers are untuned. Using spinors and a Gaussian Kernel, an untuned classifier gives an F1 score of 0.9555, which was 0.2712 when gyro measurements were used as features.
In summary, this work shows that SVM gives a satisfactory performance for the classification of faults on the control surfaces of a drone using flight data.

% \end{abstractpage}

\cleardoublepage

\section*{Acknowledgments}

This work is supported by ENGIE Ineo - Groupe ADP - Safran RPAS Chair.

%%%%%%%%%%%%%%%%%%%%%%%%%%%%%%%%%%%%%%%%%%%%%%%%%%%%%%%%%%%%%%%%%%%%%%
% -*-latex-*-
