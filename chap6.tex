\chapter{Conclusion}

Integration of drones into airspace needs the introduction of indigenous designs that will serve safe solutions for drones. One of the aspects of the problem is to assure a safe flight by designing fault detection and diagnosis with cheaper avionics common in a vast number of drones.
This work aims to design a classifier via SVM to solve FDD for drones with actuator faults.
For that purpose, we introduce an end-to-end design to achieve data-driven fault diagnosis for control surface faults on drones. All data and the code available in code sharing and versioning system \emph{Github}. 

In this thesis, fault classification simulations are investigated under two main sections: classification of faults based on simulated flight measurements and classification of faults based on real flight data. 

For classification on data generated from simulations, model of a MAKO UAV is simulated.
Sensor measurements (accelerometer and gyro data) have been calculated using information of drone's motion and the specifications of the sensors. 
Generated data is usually more structured compared to the real flight data. 
In this preliminary application of SVM to fault diagnosis, we aimed to start with an easier problem, and used data generated from models.
There is no controller involved in the model in this preliminary application of SVM to detection to discard the controller's effect on the diagnosis. 
The results show that SVM classifier was very accurate and fast in diagnosing the fault on the control surfaces with a classification accuracy of $10^{-5}$.

Then fault detection with real flight data is investigated. 
Since SVM is a supervised classification method, labeled data is necessary to train the algorithm. For that reason, real flights have been arranged to generate faulty flight data by manipulating the open source autopilot \emph{Paparazzi}.  
The training is held offline due to the need for labeled data and computational burden of the tuning phase of the classifiers. 
Two types of faults have been mainly investigated, the control surface stuck and loss of effectiveness of the elevon. Results indicate that the control surface stuck can be detected relatively easily with 3 gyros and 3 accelerometers data compared to loss of effectiveness (efficiency). 
The results show that over the flight data, SVM yields an F1 score of 0.98 for classification of control surface stuck fault. 
Addition of features to accommodate previous measurements improve classification performance for tuned classifiers while the untuned classification performance deteriorates. 
Classification performs poorly for loss of efficiency faults especially for smaller ineffectiveness values. 
For the loss of efficiency fault, some feature engineering, involving the addition of past measurements is needed to attain the same classification performance.

A very promising result is discovered when \emph{spinors} are used as features instead of angular velocities. 
For that purpose, kinematics equations have been solved to calculate the \emph{quaternions} using angular velocity measurements. Then \emph{spinors} are calculated as a function of \emph{quaternions}. Result show that by using spinors for classification, there is a vast improvement in the classification accuracy especially when the classifiers are untuned. Using spinors and a Gaussian Kernel, untuned classifier gives f1Score of 0.9555 which was 0.2712 when gyro measurements are used as features.

In general, this work shows that SVM gives satisfactory performance for classification of faults on control surfaces of a drone using flight data.

\section{Future Work}

This thesis focuses on only two class classification problem. 
The data used to train the classifier is comprised of a selected faulty and a selected nominal phase of the flight. 
Future work should consider multi-class classification since it will lead to a more realistic application on fault diagnosis for drones. 
If all possible faults are to be issued, the problem could be very complicated due to the vast number of possible faults. 
For such a problem, methods such as deep learning could be selected instead of SVM, since deep learning offers appealing performance on classification problems with a large number classes. 
Another possible workaround could be to label the faults as severe, moderate and mild. 
Then, rather than determining the exact fault, a situation awareness about the severity of the fault could be gained. 
This knowledge could be used to plan mitigation measures such as recovery or ditching to reduce the harm on the ground.

Fault detection\&diagnosis should also be complemented by a recovery. The abilities of a drone after a fault should be assesses and a control action should be taken to mitigate the faulty situation. If recovery is not a viable option, a ditching maneuver could be proceed to reduce the harm.
