\chapter{Conclusion}

To integrate drones into the airspace it is necessary to introduce innovative designs that will enable safe solutions for unmanned flights. One of the aspects of the problem is to design Fault Detection and Diagnosis (FDD) compatible with the cheaper avionics found in a vast number of drones.
In this work a data-driven FDD for drones with faulty control surfaces is established via Support Vector Machine (SVM). All the data and the software code are available in the code sharing and versioning system \emph{Github}. 

In this thesis, fault classification simulations are investigated under two main sections: classification of faults based on simulated flight measurements and classification of faults based on real flight data. We started with the easier problem: classification of faults based on simulated flight measurements.

For classification on data generated from simulations, a model of a MAKO Unmanned Aerial Vehicle (UAV) is simulated.
Sensor measurements (accelerometer and gyrometer data) were simulated using the information of the drone's motion and the specifications of the real sensors. 
The data generated thus is usually more structured compared with the real flight data.
There are no flight control loops involved in the model: discarding the controller's effect eases the diagnosis. 
The results show that the SVM classifier was very accurate and fast in diagnosing the fault on the control surfaces with a classification accuracy of $10^{-5}$.

Then fault detection with real flight data is investigated. 
Since SVM is a supervised classification method, labeled data is necessary to train the algorithm. For that reason, real flights have been arranged to generate faulty flight data by manipulating the open source autopilot \emph{Paparazzi}.  
The training is held offline due to the need for labeled data and computational burden of the tuning phase of the classifiers. 
Two types of faults have been mainly investigated: a stuck elevon and the loss of effectiveness of the elevon. Results indicate that with 3 gyros and 3 accelerometers it is easier to detect the stuck control surface than it is to detect the loss of effectiveness. 

The results show that over the flight data, tuned SVM yields an F1 score of 0.98 for the classification of the stuck control surface. 
The addition of features to accommodate the previous measurements improves the classification performance for tuned classifiers but deteriorates the untuned classification performance. 

Classification performs poorly for loss of efficiency faults especially for small losses of effectiveness. 
In order to improve classification performance for the loss of efficiency faults, some feature engineering is required: for example the addition of past measurements. A very promising result is discovered when \emph{spinors} are used as features instead of angular velocities. For that purpose, the kinematic equations have been solved to calculate the \emph{quaternions} using angular velocity measurements. Then \emph{spinors} are calculated as a function of \emph{quaternions}. Results show that by using spinors for classification, there is a vast improvement in the classification accuracy especially when the classifiers are untuned. Using spinors and a Gaussian Kernel, the untuned classifier gives an F1 score of 0.9555 which was 0.2712 when the gyrometer measurements were used as features. 

This work thus shows that SVM yields satisfactory performance levels for the classification of faults on control surfaces of a drone using real flight data. The FDD algorithm designed here is a first step towards safer drone flights. The next section introduces possible future steps. 

\section{Future Work}

For each failure type - jammed elevon or ineffective elevon - this thesis applies a two class classification problem: faulty data and nominal data. 
Future work should consider multi-class classification since it will lead to a more realistic application of fault diagnosis for drones; but the problem could be very complicated due to the vast number of possible faults. 
For such a problem, methods such as deep learning could be selected instead of SVM, since deep learning offers appealing performance on classification problems with a large number of classes. 
Another possible workaround could be to label the faults as severe, moderate and mild. Then, rather than determining the exact  \emph{nature} of the fault, at least an awareness of the  \emph{severity} of the fault could be gained.
Fault Detection and Diagnosis should also be complemented by recovery. The abilities of a drone after a fault should be assessed and a control action should be taken to mitigate the faulty situation. If recovery is not a viable option, a ditching maneuver could be undertaken to reduce the harm on the ground. Knowledge of the fault's severity could be used to choose an adequate mitigation measure.
