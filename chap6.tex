\chapter{Conclusion}

The integration of drones into the airspace demands the introduction of innovative designs to provide safe solutions for drones. One aspect of this issue is to ensure safe flight by designing fault detection and diagnosis systems with less expensive avionics, common in a vast number of drones.
This work aims to design a classifier via SVM to solve FDD for drones with actuator faults.
For that purpose, we introduce an end-to-end design to achieve data-driven fault diagnosis for the control surface faults in drones.
All the data and the software code are available in the code sharing and versioning system \emph{Github}. 

In this thesis, fault classification simulations are investigated under two main sections: first, the classification of faults based on simulated flight measurements; and second, the classification of faults based on real flight data. We started with the easier problem: classification of faults based on simulated flight measurements.

For the classification problem using data generated by simulations, a model of a MAKO Unmanned Aerial Vehicle (UAV) is simulated.
Sensor measurements (accelerometer and gyro data) are simulated using the information on the drone's motion and the specifications of the real sensors. 
Generated data is usually more structured compared to real flight data.
There are no flight control loops involved in the model: discarding the controller's effect eases the diagnosis. 
The results show that the SVM classifier is accurate and fast in diagnosing the fault on the control surfaces, with a classification accuracy of $10^{-5}$.

Next, fault detection with real flight data is investigated. 
Since SVM is a supervised classification method, labeled data is necessary in order to train the algorithm. For this reason, real flights are arranged to generate faulty flight data by manipulating the open source autopilot, \emph{Paparazzi}.  
Training is held offline due to the need for labeled data and the computational burden of the tuning phase of the classifiers. 
Two types of faults are the focus of the investigation: a stuck elevon fault and the loss of effectiveness of the elevon. 
Results indicate that the control surface stuck fault can be detected relatively easily with three gyros and three accelerometer measurements, compared to the loss of effectiveness fault. 
The results show that over the flight data, tuned SVM yields an F1 score of 0.98 for the classification of control surface faults. 
The addition of features to accommodate the previous measurements improves the classification performance for tuned classifiers, while the untuned classification performance deteriorates. 
Classification performs poorly for the loss of efficiency faults, especially for small losses of effectiveness. 
For the loss of efficiency fault, some feature engineering --- involving the addition of past measurements --- is needed in order to attain similar classification performance. 

A very promising result is discovered when \emph{spinors} are used as features instead of angular velocities. Thus, the kinematic equations have been solved so as to calculate the \emph{quaternions} using angular velocity measurements. Then \emph{spinors} are calculated as a function of \emph{quaternions}. Results show that by using spinors for classification, there is a vast improvement in classification accuracy, especially when the classifiers are untuned. Using spinors and a Gaussian Kernel, the untuned classifiers give an F1 score of 0.9555 which was 0.2712 when the gyro measurements were used as features. 

This work thus shows that SVM yields a satisfactory performance levels for the classification of faults on the control surfaces of a drone using real flight data. The FDD algorithm designed here is a first step towards safer drone flights. The next section introduces possible future steps. 

\section{Future Work}

For each failure type - jammed elevon or ineffective elevon - this thesis applies a two class classification problem: faulty data and nominal data. 
Future work should consider multi-class classification since it will lead to a more realistic application of fault diagnosis for drones; but the problem could be very complicated due to the vast number of possible faults. 
For such a problem, methods such as deep learning could be selected instead of SVM, since deep learning offers appealing performance on classification problems with a large number of classes. 
Another possible workaround could be to label the faults as severe, moderate and mild. Then, rather than determining the exact  \emph{nature} of the fault, at least an awareness of the  \emph{severity} of the fault could be gained.
Fault Detection and Diagnosis should also be complemented by recovery. The abilities of a drone after a fault should be assessed and a control action should be taken to mitigate the faulty situation. If recovery is not a viable option, a ditching maneuver could be undertaken to reduce the harm on the ground. Knowledge of the fault's severity could be used to choose an adequate mitigation measure.
