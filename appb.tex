\chapter{Codes for FD from flight data}

\lstset{frame=tb,
  language=Matlab,
  aboveskip=3mm,
  belowskip=3mm,
  showstringspaces=false,
  columns=flexible,
  basicstyle={\small\ttfamily},
  numbers=none,
  numberstyle=\tiny\color{gray},
  keywordstyle=\color{blue},
  commentstyle=\color{dkgreen},
  stringstyle=\color{mauve},
  breaklines=true,
  breakatwhitespace=true,
  tabsize=3
}

\section{dataRead.m}
\begin{lstlisting}
% Written by Ewoud Smeur 
% Modified by Elgiz Baskaya

%%%%%%% This part from Ewoud %%%%%%%%%
% filename = '17_04_20__14_22_51_SD.data'; % Mulitplicative fault only
filename = '17_07_06__10_21_07_SD.data'; % Muret with Michel
% filename = '17_09_07__10_07_55_SD.data'; 

formatSpec = '%f%f%s%f%s%f%f%f%f%f%f%f%f%f%f%f%f%f%f%f%f%f';

formatSpecHeader = '%s%s%s%s%s%s%s%s%s%s%s%s%s%s%s%s%s%s%s%[^\n\r]';
delimiter = ' ,';
startRow = 1;
fileID = fopen(filename,'r');
header = textscan(fileID, formatSpecHeader,1, 'Delimiter', delimiter, 'EmptyValue' ,NaN);
dataArray = textscan(fileID, formatSpec, 'Delimiter', delimiter, 'EmptyValue' ,NaN,'HeaderLines' ,startRow, 'ReturnOnError', false);
fclose(fileID);
 
N = length(dataArray{1, 1})-1;

%%%%%%% This part from Elgiz %%%%%%%%%

% Selecting the drone with whose data you want to work with
index_drone_select = find(dataArray{1,2}==52);
drone_select_id = zeros(length(dataArray{1,1}),1);
% Set indexes to 1s if it is the drone of interest
drone_select_id(index_drone_select) = 1;

% Finding flight interval
index_altitude = find(dataArray{1,8}>190000);
altitude_limit_id = zeros(length(dataArray{1,8}),1);
% Sets indexes to 1s if the altitude is greater than the given limit
altitude_limit_id(index_altitude) = 1;

% dataArray{1,8} can be something else then GPS data so select the GPS 
% indexed as well
gps_id = strcmp(dataArray{1,3},'GPS');

% find out the first time of pass of the altitude limit of drone of interest
% and last time of pass of the altitude limit of the drone of interest
% And indexes i. if it is GPS data
%            ii. if it is greater than the altitude of interest
%           iii. if it is drone of interest

index_drone_gps_alt = altitude_limit_id & gps_id & drone_select_id;
first_altPass = find(index_drone_gps_alt, 1, 'first');
last_altPass = find(index_drone_gps_alt, 1, 'last');

% All the times in between the first passing of altitude limit and last
% passing of the altitude limit are assumed to be the flight duration
flight_duration_id = zeros(length(dataArray{1,1}),1);
flight_duration_id(first_altPass:last_altPass) = 1;

%%%%%%%%% Ewoud again ! %%%%%%%%%%%%%
array_col_5 = zeros(length(dataArray{1, 5}),1);
for i = 1:length(dataArray{1, 5})
    try
        array_col_5(i) = str2num(dataArray{1,5}{i});
    end
end

%%%%%%%% Here we welcome Elgiz %%%%%%%
% The idea is to AND all the required indexes

gyro_id_only = strcmp(dataArray{1,3},'IMU_GYRO');
gyro_id = gyro_id_only & drone_select_id & flight_duration_id;
gyro(:,1) = dataArray{1, 4}(gyro_id);
gyro(:,2) = array_col_5(gyro_id);
gyro(:,3) = dataArray{1, 6}(gyro_id);
t_gyro = dataArray{1, 1}(gyro_id);

accel_id_only = strcmp(dataArray{1,3},'IMU_ACCEL');
accel_id = accel_id_only & drone_select_id & flight_duration_id;
accel(:,1) = dataArray{1, 4}(accel_id);
accel(:,2) = array_col_5(accel_id);
accel(:,3) = dataArray{1, 6}(accel_id);
t_accel = dataArray{1, 1}(accel_id);

commands_id_only = strcmp(dataArray{1,3},'COMMANDS');
commands_id = drone_select_id & commands_id_only & flight_duration_id;
commands_index = find(commands_id == 1);
commands(:,1) = dataArray{1, 7}(commands_id);
commands(:,2) = dataArray{1, 8}(commands_id);
t_commands = dataArray{1, 1}(commands_id);

gps_id = drone_select_id & gps_id & flight_duration_id;
altitude(:,1) = dataArray{1, 8}(gps_id)/1000;
t_altitude = dataArray{1, 1}(gps_id);

% Labeling outputs (Fault, Normal)

% FAULT (Here different faults are all labeled under one category which is fault) 
% Finding the faulty command indexes
% SETTINGS give the multiplication factor that is used to inject the fault
% to control surfaces. 

% Each time a fault is injected from the ground station, there
% appears a SETTINGS message, with the information on the fault signal.
% An example : 535.8420 18 SETTINGS 1.000000 0.500000
%              [time  droneNum typeMass leftContSurfaceEfficiency
%              rightContSurfaceEfficiency]
% The values 1.000000 and 0.500000 are manual inputs from GCS by operator.

settings_id = strcmp(dataArray{1,3},'SETTINGS');
settings_index = find(settings_id == 1);

% Indexes of setting command where drone set to nominal control surface
% condition (1.00 1.00 for multiplicative fault)
set_nominal = settings_index((dataArray{1,4}(settings_index)==1)&...
(array_col_5(settings_index)==1)&(dataArray{1,6}(settings_index)==0)&...
(dataArray{1,7}(settings_index)==0));

% number of fault sets 
num_fault_set = length(settings_index) - length(set_nominal);

% Initialization fault_start_stop and nominal_start_stop vectors
fault_start_stop = zeros(2, num_fault_set);
% adding the intervals before any fault injected (after an certain altitude
% to first fault anf after the last fault finishes to a certain altitude)
nominal_start_stop = zeros(2, length(set_nominal) + 1);
j = 1;
k = 2;
for i = 1 : (length(settings_index) - 1)
    if ~any(set_nominal==settings_index(i))
        % fault_start_stop  rows : starting_index end_index
        %                   coloum : each coloumn is for a different fault
        %                   injected
        fault_start_stop(1:2,j) = [settings_index(i) (settings_index(i + 1) - 1)]';
        j = j + 1;
        
    else
        nominal_start_stop(1:2,k) = [settings_index(i) (settings_index(i + 1) - 1)]';
        k = k + 1;
    end
end
% Adding the nominal intervals starting from passing an altitude to the
% first falut injected (at start of the flight) and starting after the last
% fault finishes until it descents until the same altitude limit (at the 
% end of the flight) 
nominal_start_stop(1:2,1) = [first_altPass (fault_start_stop(1,1) - 1)]';
nominal_start_stop(1:2,length(set_nominal) + 1) = [fault_start_stop(2,num_fault_set)+1 last_altPass]';

%%%%%%%%%  Hello Ewoud %%%%%%%%%%
% act_id = strcmp(dataArray{1,3},'ROTORCRAFT_CMD');
% u_in(:,1) = dataArray{1, 4}(act_id);
% u_in(:,2) = array_col_5(act_id);
% u_in(:,3) = dataArray{1, 6}(act_id);
% u_in(:,4) = dataArray{1,  7}(act_id);
% t_act = dataArray{1, 1}(act_id);
% 
% gps_id = strcmp(dataArray{1,3},'GPS_INT');
% ecefv(:,1) = dataArray{1, 11}(gps_id)/100;
% ecefv(:,2) = dataArray{1, 12}(gps_id)/100;
% ecefv(:,3) = dataArray{1, 13}(gps_id)/100;
% t_gps = dataArray{1, 1}(gps_id);
 \end{lstlisting}
 
\section{selectDataToInvest.m}
\begin{lstlisting}
% Copyright 2017 Elgiz Baskaya

% This file is part of cureDDrone.

% cureDDrone is free software: you can redistribute it and/or modify
% it under the terms of the GNU General Public License as published by
% the Free Software Foundation, either version 3 of the License, or
% (at your option) any later version.
% 
% curedRone is distributed in the hope that it will be useful,
% but WITHOUT ANY WARRANTY; without even the implied warranty of
% MERCHANTABILITY or FITNESS FOR A PARTICULAR PURPOSE.  See the
% GNU General Public License for more details.
% 
% You should have received a copy of the GNU General Public License
% along with curedRone.  If not, see <http://www.gnu.org/licenses/>.

% FAULT DETECTION VIA SVM
% This code assumes that you already have a data set of normal and faulty 
% situation sensor outputs.

%% FAULT SELECTION
% to check available fault indexes (the index when they are set by
% operator) setdiff(settings_index,set_nominal)
% and their corresponding start_index end_index, check fault_start_stop

fault_id = zeros(length(dataArray{1,1}),1);
% Select which fault interval you would like to investigate
% fault_id(fault_start_stop(1,FAULT_NUM_YOU_WANTTO_SIMULATE):...
% fault_start_stop(2,FAULT_NUM_YOU_WANTTO_SIMULATE)) = 1;

% % One surface stuck at zero fault
% fault_id(fault_start_stop(1,23):fault_start_stop(2,23)) = 1;

% One surface loss if efficiency fault
fault_id(fault_start_stop(1,3):fault_start_stop(2,3)) = 1;

% All faulty phase indexes
% for i = 1 : length(fault_start_stop)
%     fault_id(fault_start_stop(1,i):fault_start_stop(2,i)) = 1;
% end

gyro_fault_cond_id = fault_id & gyro_id_only;

gyro_fault_cond(:,1) = dataArray{1, 4}(gyro_fault_cond_id);
gyro_fault_cond(:,2) = array_col_5(gyro_fault_cond_id);
gyro_fault_cond(:,3) = dataArray{1, 6}(gyro_fault_cond_id);
t_gyro_fault_cond = dataArray{1, 1}(gyro_fault_cond_id);

accel_fault_cond_id = fault_id & accel_id_only;

accel_fault_cond(:,1) = dataArray{1, 4}(accel_fault_cond_id);
accel_fault_cond(:,2) = array_col_5(accel_fault_cond_id);
accel_fault_cond(:,3) = dataArray{1, 6}(accel_fault_cond_id);
t_accel_fault_cond = dataArray{1, 1}(accel_fault_cond_id);

% Selection of nominal condition
% to check available fault indexes (the index when they are set by
% operator) : see variable set_nominal
% and their corresponding start_index end_index, check nominal_start_stop

nominal_id = zeros(length(dataArray{1,1}),1);
% Select which nominal phase interval you would like to investigate
% nominal_id(nominal_start_stop(1,NOMINAL_COND_NUM_YOU_WANTTO_SIMULATE):...
% nominal_start_stop(2,NOMINAL_COND_NUM_YOU_WANTTO_SIMULATE)) = 1;

% % One surface stuck at zero fault
% nominal_id(nominal_start_stop(1,5):nominal_start_stop(2,5)) = 1;

% One surface loss if efficiency fault
nominal_id(nominal_start_stop(1,1):nominal_start_stop(2,1)) = 1;

% All nominal phase indexes
% for i = 1 : length(nominal_start_stop)
%     nominal_id(nominal_start_stop(1,i):nominal_start_stop(2,i)) = 1;
% end

gyro_nominal_cond_id = nominal_id & gyro_id_only;

gyro_nominal_cond(:,1) = dataArray{1, 4}(gyro_nominal_cond_id);
gyro_nominal_cond(:,2) = array_col_5(gyro_nominal_cond_id);
gyro_nominal_cond(:,3) = dataArray{1, 6}(gyro_nominal_cond_id);
t_gyro_nominal_cond = dataArray{1, 1}(gyro_nominal_cond_id);

accel_nominal_cond_id = nominal_id & accel_id_only;

accel_nominal_cond(:,1) = dataArray{1, 4}(accel_nominal_cond_id);
accel_nominal_cond(:,2) = array_col_5(accel_nominal_cond_id);
accel_nominal_cond(:,3) = dataArray{1, 6}(accel_nominal_cond_id);
t_accel_nominal_cond = dataArray{1, 1}(accel_nominal_cond_id);

% Forming the feature and output vectors to apply classification.
% Here we form the matrix as 
% feature_vector = [acc_x_nominal acc_y_nom   acc_z_nom   gyro_x_nom  gyro_y_nom   gyro_z_nom
%                   acc_x_fault   acc_y_fault acc_z_fault gyro_x_faul gyro_y_fault gyro_z_fault]
feature_vector = [accel_nominal_cond gyro_nominal_cond; accel_fault_cond gyro_fault_cond];
% Assuming the time steps for the gyro and the accelerometers are sync.
t_features = [t_accel_nominal_cond; t_accel_fault_cond];
% Assuming same number of gyro and accelerometer data
% Labelling data

% nominal_label = cell(length(gyro_nominal_cond),1);
% nominal_label(:) = {'nominal'};
% fault_label = cell(length(gyro_fault_cond),1);
% fault_label(:) = {'fault'};
% label = [nominal_label; fault_label];
% output_vector = label;

nominal_label = zeros(length(gyro_nominal_cond),1);
fault_label = ones(length(gyro_fault_cond),1);
output_vector = [nominal_label; fault_label];

%% ADD FEATURES OF CONSEQUENT MEASUREMENTS

% % Number of next (and previous) measurements to add to the feature vector : N
% feature_vector_original = feature_vector;
% clear feature_vector;
% N = 3;
% [row,col] = size(feature_vector_original);
% 
% addedFeat = zeros(row, N + 1);

% for i = 1 : col
%     % If features added before the current time measurement
%     addedFeat = addFeaturesBefore(feature_vector_original(:,i),N);
%     feature_vector(:,((i-1)*(N+1)+1):((i-1)*(N+1)+1+N)) = addedFeat;
%     
%     % If features added both before and after the current time measurement
%     addedFeat = addFeaturesBeforeAfter(feature_vector_original(:,i),N);
%     feature_vector(:,((i-1)*(2*N+1)+1):((i-1)*(2*N+1)+1+2*N)) = addedFeat;
% end

% Figures to visualize data
% feature = [accel_nominal_cond;accel_fault_cond];
gscatter(feature_vector(:,1),feature_vector(:,3),output_vector,'gr')
legend('normal','fault')
set(legend,'FontSize',11);
xlabel({'$a_x$'},...
'FontUnits','points',...
'interpreter','latex',...
'FontSize',15,...
'FontName','Times')
ylabel({'$a_y$'},...
'FontUnits','points',...
'interpreter','latex',...
'FontSize',15,...
'FontName','Times')
print -depsc2 feat1vsfeat3.eps
 \end{lstlisting}
 
\section{arrangeDataSet.m}
\begin{lstlisting}
% Copyright 2017 Elgiz Baskaya

% This file is part of cureDDrone.

% cureDDrone is free software: you can redistribute it and/or modify
% it under the terms of the GNU General Public License as published by
% the Free Software Foundation, either version 3 of the License, or
% (at your option) any later version.
% 
% curedRone is distributed in the hope that it will be useful,
% but WITHOUT ANY WARRANTY; without even the implied warranty of
% MERCHANTABILITY or FITNESS FOR A PARTICULAR PURPOSE.  See the
% GNU General Public License for more details.
% 
% You should have received a copy of the GNU General Public License
% along with curedRone.  If not, see <http://www.gnu.org/licenses/>.

% FAULT DETECTION VIA SVM
% This code assumes that you already have a data set of normal and faulty 
% situation sensor outputs.

%% Arrange training/test sets

feature_vec = feature_vector;
output_vec = output_vector;

% training set (around %80 percent of whole data set)
trainingDataExNum = ceil(80 / 100 * (length(feature_vec)));

% Select %80 of data for training and leave the rest for testing
randomSelectionColoumnNum = randperm(length(feature_vec),trainingDataExNum);

% Training set for feature and output
% feature_vec_training .:. feature matrix for training
% output_vec_training .:. output vector for training
feature_vec_training = feature_vec(randomSelectionColoumnNum, :);
output_vec_training = output_vec(randomSelectionColoumnNum, :);

% Test set for feature and output
feature_vec_test = feature_vec;
feature_vec_test(randomSelectionColoumnNum, :) = [];

output_vec_test = output_vec;
output_vec_test(randomSelectionColoumnNum, :) = [];

test_set_time = t_features;
test_set_time(randomSelectionColoumnNum) = [];

% To have same partions for cross-validations 
rng(1);
cFold = cvpartition(length(feature_vec_training),'KFold',10);
 \end{lstlisting}

 
 \section{svmFD.m}
\begin{lstlisting}
% Copyright 2017 Elgiz Baskaya

% This file is part of cureDDrone.

% cureDDrone is free software: you can redistribute it and/or modify
% it under the terms of the GNU General Public License as published by
% the Free Software Foundation, either version 3 of the License, or
% (at your option) any later version.
% 
% curedRone is distributed in the hope that it will be useful,
% but WITHOUT ANY WARRANTY; without even the implied warranty of
% MERCHANTABILITY or FITNESS FOR A PARTICULAR PURPOSE.  See the
% GNU General Public License for more details.
% 
% You should have received a copy of the GNU General Public License
% along with curedRone.  If not, see <http://www.gnu.org/licenses/>.

% FAULT DETECTION VIA SVM
% This code assumes that you already have a data set of normal and faulty 
% situation sensor outputs.

%% TRAINING PHASE
tic
% SVMModel is a trained ClassificationSVM classifier.
SVMModel = fitcsvm(feature_vec_training,output_vec_training, 'KernelFunction','rbf','Standardize',true,'ClassNames',{'0','1'});
toc

% Support vectors
sv = SVMModel.SupportVectors;

%% CROSS VALIDATION
% 10-fold cross validation on the training data
% inputs : trained SVM classifier (which also stores the training data)
% outputs : cross-validated (partitioned) SVM classifier from a trained SVM
% classifier

% CVSVMModel is a ClassificationPartitionedModel cross-validated classifier.
% ClassificationPartitionedModel is a set of classification models trained 
% on cross-validated folds.

CVSVMModel = crossval(SVMModel,'CVPartition',cFold);

% To assess predictive performance of SVMModel on cross-validated data 
% "kfold" methods and properties of CVSVMModel, such as kfoldLoss is used

% Evaluate 10-fold cross-validation error.
% (Estimate the out-of-sample misclassification rate.)
crossValClassificErr = kfoldLoss(CVSVMModel);

% Predict response for observations not used for training
% Estimate cross-validation predicted labels and scores.
[elabelUntuned,escoreUntuned] = kfoldPredict(CVSVMModel);

max(escoreUntuned)
min(escoreUntuned)

% FIT POSTERIOR PROBABILITES fitPosterior(SVMModel) / fitSVMPosterior(CVSVMModel)
% [ScoreCVSVMModel,ScoreParameters] = fitSVMPosterior(CVSVMModel);
% Predict does not work here?


%% PREDICTION PHASE

[labelUntuned,scoreUntuned] = predict(SVMModel,feature_vec_test);

% %% FIT POSTERIOR PROBABILITES fitPosterior(SVMModel) / fitSVMPosterior(CVSVMModel)
% % "The transformation function computes the posterior probability 
% % that an observation is classified into the positive class (SVMModel.Classnames(2)).
% % The software fits the appropriate score-to-posterior-probability 
% % transformation function using the SVM classifier SVMModel, and 
% % by conducting 10-fold cross validation using the stored predictor data (SVMModel.X) 
% % and the class labels (SVMModel.Y) as outlined in REF : Platt, J. 
% % "Probabilistic outputs for support vector machines and comparisons 
% % to regularized likelihood methods". In: Advances in Large Margin Classifiers. 
% % Cambridge, MA: The MIT Press, 2000, pp. 61-74"
% ScoreSVMModel = fitPosterior(SVMModel);
% [~,postProbability] = predict(ScoreSVMModel,feature_vec_test);

%% EVALUATING THE PERFORMANCE OF CLASSIFICATION WITH NEW DATA

% Evaluating the prediction performance of classification via
% CompactClassificationSVM class methods (e.g compareHoldout, edge, loss, margin, 
% predict)

eUntuned = edge(SVMModel, feature_vec_test, output_vec_test);
mUntuned = margin(SVMModel, feature_vec_test, output_vec_test);

% Evaluating the prediction performance of classification via confusion matrix

[f1scoreUntuned, precisionUntuned, recallUntuned] = calcF1score(output_vec_test, str2double(labelUntuned));
%% Plot results
figure
gscatter(feature_vec_training(:,1),feature_vec_training(:,2),output_vec_training)
hold on
plot(sv(:,1),sv(:,2),'ko','MarkerSize',10)
legend('normal','fault','Support Vector')
legend('normal','fault')
hold off
set(legend,'FontSize',11);
xlabel({'$a_x$'},...
'FontUnits','points',...
'interpreter','latex',...
'FontSize',15,...
'FontName','Times')
ylabel({'$a_y$'},...
'FontUnits','points',...
'interpreter','latex',...
'FontSize',15,...
'FontName','Times')
print -depsc2 feat1vsfeat2.eps

 \end{lstlisting}
 
 \section{svmFDtuningViaHeuristic.m}
\begin{lstlisting}
% Copyright 2017 Elgiz Baskaya

% This file is part of cureDDrone.

% cureDDrone is free software: you can redistribute it and/or modify
% it under the terms of the GNU General Public License as published by
% the Free Software Foundation, either version 3 of the License, or
% (at your option) any later version.
% 
% curedRone is distributed in the hope that it will be useful,
% but WITHOUT ANY WARRANTY; without even the implied warranty of
% MERCHANTABILITY or FITNESS FOR A PARTICULAR PURPOSE.  See the
% GNU General Public License for more details.
% 
% You should have received a copy of the GNU General Public License
% along with curedRone.  If not, see <http://www.gnu.org/licenses/>.

% FAULT DETECTION VIA SVM
% This code assumes that you already have a data set of normal and faulty 
% situation sensor outputs.

%% TUNING THE SVM CLASSIFIER using heuristic approach to select kernel scale

tic
% SVMModelTune is a trained ClassificationSVM classifier 
% By passing 'KernelScale','auto' the software utilizes a heuristic
% approach to select kernel scale
SVMModelTune1 = fitcsvm(feature_vec_training,output_vec_training, 'KernelFunction','rbf', 'KernelScale','auto','Standardize',true,'ClassNames',{'0','1'});

%% CROSS VALIDATION
% 10-fold cross validation on the training data
% inputs : trained SVM classifier (which also stores the training data)
% outputs : cross-validated (partitioned) SVM classifier from a trained SVM
% classifier

% CVSVMModelTune is a ClassificationPartitionedModel cross-validated classifier.
% ClassificationPartitionedModel is a set of classification models trained 
% on cross-validated folds.
CVSVMModelTune1 = crossval(SVMModelTune1,'CVPartition',cFold);

% To assess predictive performance of SVMModelTune on cross-validated data 
% "kfold" methods and properties of CVSVMModelTune, such as kfoldLoss is used

% Evaluate 10-fold cross-validation error.
% (Estimate the out-of-sample misclassification rate.)
crossValClassificErrTuning1 = kfoldLoss(CVSVMModelTune1);

% Predict response for observations not used for training
% Estimate cross-validation predicted labels and scores.
[elabelTune1,escoreTune1] = kfoldPredict(CVSVMModelTune1);

max(escoreTune1)
min(escoreTune1)

%% RETRAIN SVM CLASSIFIER
% Retrain for different values of BoxConstraint and KernelScale
% This KernelScale is the KernelScale found by the heuristic approach
sampleSpace = 11;
kernelScaleFactor = zeros(1,sampleSpace + 1);
boxConstraint = zeros(1,sampleSpace + 1);
crossValClassificErrTuning2 = zeros(sampleSpace,sampleSpace);

ks = SVMModelTune1.KernelParameters.Scale;
boxConstraint(1) = 1e-5;
kernelScaleFactor(1) = 1e-5;
minCrossValClassificError = 100;

for i = 1 : sampleSpace
    for j = 1 : sampleSpace
        
        SVMModelTune2 = fitcsvm(feature_vec_training,output_vec_training, 'KernelFunction','rbf', 'KernelScale',ks * kernelScaleFactor(j),'BoxConstraint',boxConstraint(i),...
        'Standardize',true,'ClassNames',{'0','1'});
        
        % CrossValidate
        CVSVMModelTune2 = crossval(SVMModelTune2,'CVPartition',cFold);
        crossValClassificErrTuning2(i,j) = kfoldLoss(CVSVMModelTune2);
        if crossValClassificErrTuning2(i,j) < minCrossValClassificError
            minCrossValClassificError = crossValClassificErrTuning2(i,j);
            kernelScaleOptim = SVMModelTune2.KernelParameters.Scale;
            boxConstraintOptim = SVMModelTune2.ModelParameters.BoxConstraint;
        end
        kernelScaleFactor(j + 1) = kernelScaleFactor(j) * 10;
    end
    boxConstraint(i + 1) = boxConstraint(i) * 10;
end

toc

%% TRAIN AGAIN WITH THE TUNED KernelScale and BoxConstraint
SVMModelTune = fitcsvm(feature_vec_training,output_vec_training, 'KernelFunction','rbf', 'KernelScale',kernelScaleOptim,'BoxConstraint',boxConstraintOptim,...
'Standardize',true,'ClassNames',{'0','1'});

%% PREDICTION PHASE

[labelTune,scoreTune] = predict(SVMModelTune,feature_vec_test);

%% EVALUATING THE PERFORMANCE OF CLASSIFICATION WITH NEW DATA

% Evaluating the prediction performance of classification via
% CompactClassificationSVM class methods (e.g compareHoldout, edge, loss, margin, 
% predict)

eTune = edge(SVMModelTune, feature_vec_test, output_vec_test);
mTune = margin(SVMModelTune, feature_vec_test, output_vec_test);

% Evaluating the prediction performance of classification via confusion matrix

[f1scoreTune, precisionTune, recallTune] = calcF1score(output_vec_test, str2double(labelTune));
 \end{lstlisting}
 
 
  \section{svmFDtuningViaOptim.m}
\begin{lstlisting}
% Copyright 2017 Elgiz Baskaya

% This file is part of cureDDrone.

% cureDDrone is free software: you can redistribute it and/or modify
% it under the terms of the GNU General Public License as published by
% the Free Software Foundation, either version 3 of the License, or
% (at your option) any later version.
% 
% curedRone is distributed in the hope that it will be useful,
% but WITHOUT ANY WARRANTY; without even the implied warranty of
% MERCHANTABILITY or FITNESS FOR A PARTICULAR PURPOSE.  See the
% GNU General Public License for more details.
% 
% You should have received a copy of the GNU General Public License
% along with curedRone.  If not, see <http://www.gnu.org/licenses/>.

% FAULT DETECTION VIA SVM
% This code assumes that you already have a data set of normal and faulty 
% situation sensor outputs.

%% TUNING THE SVM CLASSIFIER using Bayesian Optimization
sigma = optimizableVariable('sigma',[1e-5,1e5],'Transform','log');
box = optimizableVariable('box',[1e-5,1e5],'Transform','log');

minfn = @(z)kfoldLoss(fitcsvm(feature_vec_training,output_vec_training,...
    'CVPartition',cFold,'KernelFunction','rbf','BoxConstraint',z.box,...
    'KernelScale',z.sigma));

results = bayesopt(minfn,[sigma,box],'IsObjectiveDeterministic',true,...
    'AcquisitionFunctionName','expected-improvement-plus')

z(1) = results.XAtMinObjective.sigma;
z(2) = results.XAtMinObjective.box;
SVMModelTuned = fitcsvm(feature_vec_training,output_vec_training,...
'KernelFunction','rbf','KernelScale',z(1),'BoxConstraint',z(2));

%% CROSS VALIDATION
% 10-fold cross validation on the training data
% inputs : trained SVM classifier (which also stores the training data)
% outputs : cross-validated (partitioned) SVM classifier from a trained SVM
% classifier

% CVSVMModel is a ClassificationPartitionedModel cross-validated classifier.
% ClassificationPartitionedModel is a set of classification models trained 
% on cross-validated folds.

CVSVMModelTuned = crossval(SVMModelTuned,'CVPartition',cFold);

% To assess predictive performance of SVMModel on cross-validated data 
% "kfold" methods and properties of CVSVMModel, such as kfoldLoss is used

% Evaluate 10-fold cross-validation error.
% (Estimate the out-of-sample misclassification rate.)
crossValClassificErrTuned = kfoldLoss(CVSVMModelTuned);

%% PREDICTION PHASE
[labelTuned,scoreTuned] = predict(SVMModelTuned,feature_vec_test);

%% EVALUATING THE PERFORMANCE OF CLASSIFICATION WITH NEW DATA

% Evaluating the prediction performance of classification via
% CompactClassificationSVM class methods (e.g compareHoldout, edge, loss, margin, 
% predict)

eTuned = edge(SVMModelTuned, feature_vec_test, output_vec_test);
mTuned = margin(SVMModelTuned, feature_vec_test, output_vec_test);

% Evaluating the prediction performance of classification via confusion matrix

[f1scoreTuned, precisionTuned, recallTuned] = calcF1score(output_vec_test, labelTuned);
 \end{lstlisting}
 
 
 
  \section{calcF1score.m}
\begin{lstlisting}
function [f1Score,precision,recall] = calcF1score(labelActual, labelPredicted)

truePositive = sum(labelPredicted & labelActual);
falsePositive = sum(~((~labelPredicted)|labelActual));
falseNegative = sum((~labelPredicted) & labelActual);
% trueNegative = sum(~(labelPredicted|labelActual));

precision = truePositive / (truePositive + falsePositive);
recall = truePositive / (truePositive + falseNegative);
f1Score = 2 * precision * recall / (precision + recall);
end
 \end{lstlisting}
 
 
   \section{addFeaturesBefore.m}
\begin{lstlisting}
% Copyright 2017 Elgiz Baskaya

% This file is part of cureDDrone.

% cureDDrone is free software: you can redistribute it and/or modify
% it under the terms of the GNU General Public License as published by
% the Free Software Foundation, either version 3 of the License, or
% (at your option) any later version.
% 
% curedRone is distributed in the hope that it will be useful,
% but WITHOUT ANY WARRANTY; without even the implied warranty of
% MERCHANTABILITY or FITNESS FOR A PARTICULAR PURPOSE.  See the
% GNU General Public License for more details.
% 
% You should have received a copy of the GNU General Public License
% along with curedRone.  If not, see <http://www.gnu.org/licenses/>.

% FAULT DETECTION VIA SVM
% This code assumes that you already have a data set of normal and faulty 
% situation sensor outputs.

%% FEATURE ADDITION
% Add features (measurements) of up to N - 1 measurements before
% An example : Lets say N = 3
% Before adding the features the feature vector 
% v = [v1 
%      v2
%      v3 
%      v4
%      v5
%      v6
%       .
%       .
%      v_m]     where m is the number of measurements

% For a given feature matrix above this file outputs 
% REMINDER the number of measurements before and after you want to add to
% the feature vector is N - 1 both sides of the original feature vector.
% So this file will add N - 1 coloumns before and N - 1 coloumns after 
% the original feature vector. And the indexes for the resulting matrix will be: 
% m is the number of measurements and also the number of rows in the
% feature vector
% Coloumn marked with * is the original feature vector
%                                           *    
% v_(2-N)     ...      v_(-1)     v_0      v_1  
% v_(2-N+1)   ...        v_0      v_1      v_2 
% v_(2-N+2)   ...        v_1      v_2      v_3
% v_(2-N+3)   ...        v_2      v_3      v_4
%     .       ...         .        .        . 
%     .       ...         .        .        .
% v_(m-N)     ...     v_(m-2)   v_(m-1)    v_m

function [vNew] = addFeaturesBefore(v,N) 

N = N + 1;
v_b = v;
vNew(:,N) = v;

for i = 1 : N - 1
    v_b(2:end,:) = v_b(1:end-1,:);
    vNew(:,N - i) = v_b;
end
 \end{lstlisting}

