%% This is an example first chapter.  You should put chapter/appendix that you
%% write into a separate file, and add a line \include{yourfilename} to
%% main.tex, where `yourfilename.tex' is the name of the chapter/appendix file.
%% You can process specific files by typing their names in at the 
%% \files=
%% prompt when you run the file main.tex through LaTeX.
\chapter{Nonlinear Aircraft Model}

The movement of any object can be represented by changes in its location (translation) or changes in its attitude (rotation) or a combination of both. 
The motion of an aircraft usually involves both translation and rotation. 
Studying aircraft motion is complicated as these two motions are coupled; for example a rotation might cause a change in aerodynamic forces which then affects the translation. 
Thus, to ease the process, the motion is broken into easier problems utilizing some assumptions. 
Such an assumption for the aircraft motion is to assume that the aircraft is a point mass; all of its mass is collected at its center of gravity, so it translates from one point to the other. 
Then, the aircraft's rotation is investigated by no longer assuming it as a point mass but a rigid body in space. 

In this chapter, modeling both of these motions (translation and rotation motion) is presented in detail.
Equations of rotation and translation motion are driven for generic aircraft. Then, calculation of forces and moments, which are required to solve those equations, are given for two types of drones (the aerodynamic force derivatives and stability derivatives are specific for each drone). 
These two examples of drones are by ETH Zurich and MAKO (used in ENAC UAV LAB).
 
In this thesis, drone motion is simulated to generate data.
The accelerometer and gyro data are generated for MAKO model, considering the specifications of IMU \emph{InvenSense MPU-9250 Nine-axis} used in \emph{Paparazzi Autopilot Apogee} onboard. 
The models derived here are not used in detection and diagnosis algorithms. 
The detection and diagnosis algorithms implemented in this thesis only use data. 

If the reader is not interested in detection and diagnosis via simulated data but interested in detection and diagnosis via real data, it is not essential to read this chapter as it explains the models that are used to simulate measurements.
Nonetheless, having background information on the physics of the system may help to understand the features (translational acceleration and angular velocities) used in both model-based and data-driven fault diagnosis.

%To clarify, machine learning methods also train a model. 
%But in that case, the model is not the equations of motion of an aircraft but rather a simpler model and may not explicitly describe the physical . 
%The only exception to not using physical models in this work for diagnosis, is the part that the diagnosis is realized using spinors as features. To calculate the spinors, kinematic equations are numerically. But still, our idea to generalize by having no models holds since only kinematic equations are used. 

% I AM HERE!
  
\section{Attitude motion modeling}

Rotation is a change in an object's attitude. 
A change in attitude is modeled using rotations about the center of gravity.  
This section derives the equations for rotation (attitude motion) in detail. 
Note that the equations for attitude kinematics and attitude dynamics can be found in Equations~\ref{eqn:compactKinematics} and \ref{eqn:attitudeDynamics}, respectively.

Rotations are directly affected by external torques and moments while translations are directly affected by external forces. 
The attitude of an aircraft during translation also affects the aerodynamic forces causing changes in translation.

A force applied at a distance from the center of mass causes a rotation.
A very common approach in aircraft control is to balance those rotations, by trimming the aircraft, such that the aircraft will not rotate.

In this section, attitude motion is represented using kinematic and dynamics equations. 
First different parametrization of attitude, such as Euler angles and quaternions, are discussed. 
Then, kinematic and dynamic equations of attitude motion are derived for a general rigid body. 
These equations are later specified for the aircraft as the rigid body of interest. 

\subsection{Attitude representations}
Let $\bm{b_1}, \bm{b_2}, \bm{b_3}$ be a triplet of unit vectors, representing an orthogonal coordinate system attached to a rigid body such that:

\begin{equation}
\label{eqn:unitVectors}
\bm{b_1}\times \bm{b_2}= \bm{b_3}
\end{equation}

The rigid body is composed of points, which do not experience any distance change between themselves during the motion of the body. 
The problem in representing attitude can simply be thought of specifying the orientation of this triplet with respect to some reference frame A, such as in Fig.~\ref{fig:theTwoFrames}.

\begin{figure}
\begin{center}
%\includegraphics[width=8.3cm]{figures/theTwoFrames}    % The printed column width is 8.4 cm.
\includegraphics[width=13cm]{figures/DarkoAxesElgiz}    % The printed column width is 8.4 cm.
\caption{Attitude representation is simply specifying the orientation of 
aircraft body axes $\bm{b_1}, \bm{b_2}, \bm{b_3}$ in the reference frame A} 
\label{fig:theTwoFrames}
\end{center}
\end{figure}

Expressing the basis vectors $b_1, b_2, b_3$ of B in terms of basis vectors $a_1, a_2, a_3$ of A is given by :

\begin{align}
\label{eqn:C_B_A}
\begin{split}
\bm{b_1} = C_{11}\bm{a_1} + C_{12}\bm{a_2} + C_{13}\bm{a_3}  ,
\\
\bm{b_2} = C_{21}\bm{a_1} + C_{22}\bm{a_2} + C_{23}\bm{a_3}  ,
\\
\bm{b_3} = C_{33}\bm{a_1} + C_{32}\bm{a_2} + C_{33}\bm{a_3}  .
\end{split}
\end{align}

where $C_{ij} \equiv {\bm{b_i} \cdot \bm{a_j}} $  is called \emph{direction cosine} as it corresponds to the cosine of the angle between  $\bm{b_i}$ and $\bm{a_j}$. 
When the previous equation set is written in matrix form, we have:

\begin{equation}{\label{eqn:C_B_Amatrix}}
\begin{bmatrix}
\bm{b_1}\\[0.3em]
\bm{b_2}\\[0.3em]
\bm{b_3}\\[0.3em]
\end{bmatrix}
=\,
\begin{bmatrix}
C_{11} & C_{12} & C_{13}\\[0.3em]
C_{11} & C_{12} & C_{13}\\[0.3em]
C_{11} & C_{12} & C_{13}\\[0.3em]
\end{bmatrix}
\,
\begin{bmatrix}
\bm{a_1}\\[0.3em]
\bm{a_2}\\[0.3em]
\bm{a_3}\\[0.3em]
\end{bmatrix}
=\,
\bm{C}^{B/A}
\,
\begin{bmatrix}
\bm{a_1}\\[0.3em]
\bm{a_2}\\[0.3em]
\bm{a_3}\\[0.3em]
\end{bmatrix}
\end{equation} 

Here $\bm{C}^{B/A}$ is called the \emph{direction cosine matrix}, also known as the \emph{rotation matrix} or \emph{coordinate transformation matrix} from A to B\cite{wie2008space}.  
The direction cosines, the elements of the direction cosine matrix, are not all independent  \cite{wertz1978spacecraftAttitude}.

The direction cosine matrix is an orthonormal matrix as the basis vectors of each reference frames are orthogonal, so:

\begin{equation}
\label{eqn:orthonormality}
\bm{C}^{-1} = \bm{C}^{\rm T}
\end{equation}

and:

\begin{equation}
\label{eqn:orthonormality2}
\bm{C}\bm{C}^{\rm T} = \bm{C}^{\rm T}\bm{C} = \mathds{1}
\end{equation}

where $\mathds{1}$ is the identity matrix.
When the orientation is preserved, an additional condition occurs:

\begin{equation}
\label{eqn:noRotation}
\begin{vmatrix}
C\\[0.01em]
\end{vmatrix}
=\,
\mathds{1}
\end{equation}

Matrices satisfying the last two properties belong to the special orthogonal group SO(3). 
The following relations are valid between $\bm{C}^{B/A}$ --- the direction cosine matrix of B relative to A or the direction cosine matrix from A to B --- and $\bm{C}^{A/B}$ --- the direction cosine matrix of A relative to B, or the direction cosine matrix from B to A --- :

\begin{align}
\label{eqn:C_B_A_vs_C_A_B}
\begin{split}
[\bm{C}^{A/B}]^{-1} = [\bm{C}^{A/B}]^{\rm T} = \bm{C}^{B/A} 
\\
[\bm{C}^{B/A}]^{-1} = [\bm{C}^{B/A}]^{\rm T} = \bm{C}^{A/B}
\end{split}
\end{align}

The direction cosine maps the vectors from reference frame to body frame.  
Let us write an arbitrary vector $\bm{H}$ in the reference frame A and in the body frame B:

\begin{align}
\label{eqn:vectorInRefFrame}
\begin{split}
\bm{H} & = H_1 \bm{a_1} + H_2 \bm{a_2} + H_3 \bm{a_3}
\\
& = H_1^{'} \bm{b_1} + H_2^{'} \bm{b_2} + H_3^{'} \bm{b_3}
\end{split}
\end{align}

Using the direction cosine matrix in the following equation, components of the arbitrary vector $\bm{H}$ are transformed from A to B:

\begin{equation}{\label{transformVectorH}}
\begin{bmatrix}
H_1^{'}\\[0.3em]
H_2^{'}\\[0.3em]
H_3^{'}\\[0.3em]
\end{bmatrix}
=\,
\begin{bmatrix}
 \bm{b_1} \cdot \bm{a_1}  &  \bm{b_1} \cdot \bm{a_2}  &  \bm{b_1} \cdot \bm{a_3} \\[0.3em]
 \bm{b_2} \cdot \bm{a_1}  & \bm{b_2} \cdot \bm{a_2}  & \bm{b_2} \cdot \bm{a_3} \\[0.3em]
 \bm{b_3} \cdot \bm{a_1}  & \bm{b_3} \cdot \bm{a_2}  &  \bm{b_3} \cdot \bm{a_3} \\[0.3em]
\end{bmatrix}
\,
\begin{bmatrix}
 H_1\\[0.3em]
 H_2\\[0.3em]
 H_3\\[0.3em]
\end{bmatrix}
=\,
\bm{C}^{B/A}
\,
\begin{bmatrix}
H_1\\[0.3em]
H_2\\[0.3em]
H_3\\[0.3em]
\end{bmatrix}
\end{equation} 

\subsubsection{Euler Angles}

One of the approaches used to represent the attitude is the use of Euler angles. 
It is a procedure to rotate three times in succession about one axis of the rotated body fixed reference frame. 
The first rotation is about any of the fixed body axes. 
The second one is about any of the other two axes which have not been used in the first rotation. 
The third one is about one of the axes which has not been used in the second rotation. 
The result is a combination of 12 sets of rotation types. 
A sequence of rotations about three different axes of reference frame A, describing the orientation of the body frame B with respect to reference frame A, can be represented as:

\begin{align}
\label{eqn:sequence}
\begin{split}
{\bm{C}}_3(\theta_{3}) & :      A^{'} \leftarrow A   ,
\\
{\bm{C}}_2(\theta_{2}) & :      A^{''} \leftarrow A^{'}   ,
\\
{\bm{C}}_1(\theta_{1}) & :      B \leftarrow A^{''}  .
\end{split}
\end{align}

The direction cosine matrix B relative to A can be given as:

\begin{equation}
\label{eqn:sequentialOrientation}
\bm{C}^{B/A}= \bm{C}_{1}(\theta_{1}) \bm{C}_{2}(\theta_{2}) \bm{C}_{3}(\theta_{3})
\end{equation}

where $\theta_{1}$, $\theta_{2}$, $\theta_{3}$ are the Euler angles. $\bm{C}_{i}(\theta_{i})$  
denotes a rotation of angle $\theta_{i}$, about the $i^{th}$ axis of the body frame. The orientation of B with respect to A is given as:

\begin{align}{\label{eqn:C_B_AmatrixEuler}}
\begin{split}
\bm{C}^{B/A}
& =\,
\begin{bmatrix}
1 & 0 & 0\\[0.3em]
0 & \cos\theta_1 & \sin\theta_1\\[0.3em]
0 & -\sin\theta_1 & \cos\theta_1\\[0.3em]
\end{bmatrix}
\,
\begin{bmatrix}
\cos\theta_2 & 0 & -\sin\theta_2\\[0.3em]
0 & 1 &0\\[0.3em]
\sin\theta_2 & 0 & \cos\theta_2\\[0.3em]
\end{bmatrix}
\,
\begin{bmatrix}
\cos\theta_3 & \sin\theta_3 & 0\\[0.3em]
-\sin\theta_3 & \cos\theta_3 & 0\\[0.3em]
0 & 0 & 1\\[0.3em]
\end{bmatrix}
\\
& =\,
\begin{bmatrix}
\cos\theta_2 \cos\theta_3 & \cos\theta_2 \sin\theta_3 & -\sin\theta_2\\[0.3em]
\sin\theta_1 \sin\theta_2 \cos\theta_3 - \cos\theta_1 \sin\theta_3 & \sin\theta_1 \sin\theta_2 \cos\theta_3 + \cos\theta_1 \cos\theta_1 \cos\theta_3 & \sin\theta_1 \cos\theta_2\\[0.3em]
\cos\theta_1 \sin\theta_2 \cos\theta_3 + \sin\theta_1 \sin\theta_3 & \cos\theta_1 \sin\theta_2 \sin\theta_3 - \sin\theta_1 \cos\theta_3 & \cos\theta_1 \cos\theta_2\\[0.3em]
\end{bmatrix}
\end{split}
\end{align}

For aircrafts, the Euler angles ($\theta_1,\theta_2,\theta_3$) are called roll, pitch and yaw, respectively, and can be seen in Fig.~\ref{fig:eulerAngSequence}.

The rotation sequence can be selected in different ways depending on the needs of the problem. 
An example would be selecting a rotation sequence for a transitioning vehicle. The assumption that the pitch angle is constrained to $0<\theta<90^\circ$ for a conventional drone to avoid singularity, is not really feasible for a transitioning vehicle which encounters $-90<\theta<90$ pitch during the whole flight envelope. For such a problem, selecting the sequence as yaw-roll-pitch, rather than the conventional yaw-pitch-roll sequence, is useful. 

Thus, if the set of rotations are selected in a different way, such as:

\begin{align}
\label{eqn:sequence2}
\begin{split}
{\bm{C}}_2(\theta_{2}) & :      A^{'} \leftarrow A   ,
\\
{\bm{C}}_3(\theta_{3}) & :      A^{''} \leftarrow A^{'}   ,
\\
{\bm{C}}_1(\theta_{1}) & :      B \leftarrow A^{''}  .
\end{split}
\end{align}

then the direction cosine matrix would differ from Eq.~\ref{eqn:C_B_AmatrixEuler}, and be given as:

\begin{align}{\label{eqn:C_B_AmatrixEuler_132}}
\begin{split}
\bm{C}^{B/A}
& =\,
\begin{bmatrix}
\cos\theta_2 \cos\theta_3 & \sin\theta_2 & -\cos\theta_2 \sin\theta_3 \\[0.3em]
-\cos\theta_1 \sin\theta_2 \cos\theta_3 + \sin\theta_1 \sin\theta_3 & \cos\theta_1 \cos\theta_2 & \cos\theta_1 \sin\theta_2 \sin\theta_3 + \sin\theta_2 \cos\theta_3\\[0.3em]
\sin\theta_1 \sin\theta_2 \cos\theta_3 + \cos\theta_1 \sin\theta_3 & -\sin\theta_1 \cos\theta_2 & -\sin\theta_1 \sin\theta_2 \sin\theta_3 + \cos\theta_1 \cos\theta_3\\[0.3em]
\end{bmatrix}
\end{split}
\end{align}



%\begin{figure}
%\begin{center}
%\includegraphics[width=11cm]{figures/eulerAnglesAircraft}    % The printed column width is 8.4 cm.
%\caption{Euler angle sequence \cite{ducard2009fault}} 
%\label{fig:eulerAnglesAircraft}
%\end{center}
%\end{figure}

\subsubsection{Quaternions}

The idea behind this representation is the Euler's \emph{eigen-axis rotation} theorem. 
According to Euler, \say{the most general displacement of a rigid body with one point fixed is a rotation about some axis.} In other words, it states that there exists a unit vector $\bm{e}$, with the property:

\begin{equation}
\label{eqn:quat1}
\bm{C}\bm{e}= \bm{e}
\end{equation}

The $\bm{e}$ vector has the same components in body and reference frames :

\begin{align}
\label{eqn:quat2}
\begin{split}
\bm{e} & = e_1 \bm{a_1} + e_2 \bm{a_2} + e_3 \bm{a_3}
\\
& = e_1 \bm{b_1} + e_2 \bm{b_2} + e_3 \bm{b_3}
\end{split}
\end{align}
 
By rotating a rigid body about this axis $\bm{e}$, a rotation from any given orientation to any other orientation can be achieved. 
Such an axis is called an \emph{Euler axis}, after the Swiss mathematician and theoretical physicist, Leonard Euler (1707-1783).

\begin{landscape}
\begin{figure}
\begin{center}
%\includegraphics[width=11cm]{figures/eulerAngSequence}    % The printed column width is 8.4 cm.
\includegraphics[width=23cm]{figures/ZagiEulerAngleSequence}
\caption{Euler angle sequence \cite{ducard2009fault}} 
\label{fig:eulerAngSequence}
\end{center}
\end{figure}
\end{landscape}

\emph{Euler symmetric parameters}, also known as \emph{quaternions}, are defined as:
 
 \begin{equation}
 \label{eqn:quat3}
\bm{q}
=\,
\begin{bmatrix}
q_0\\[0.3em]
q_1\\[0.3em]
q_2\\[0.3em]
q_3\\[0.3em]
\end{bmatrix}
=\,
\begin{bmatrix}
\cos(\phi/2)\\[0.3em]
e_1 \sin(\phi/2)\\[0.3em]
e_2 \sin(\phi/2)\\[0.3em]
e_3 \sin(\phi/2)\\[0.3em]
\end{bmatrix}
\end{equation}
 
where $\phi$ is the \emph{Euler rotation angle}. 
Hamilton (1805-1865) is considered to be the first to mention quaternions. 
For ease of the mathematical representations, we define a vector as :

\begin{equation}
\label{eqn:quat3}
\bm{q} = \bm{e}\sin{(\phi/2)}
\end{equation}

Euler symmetric parameters are not independent, and satisfy the constraint:

\begin{equation}
\label{eqn:quatNormConstraint}
\bm{q}^{\rm T} \bm{q} + q_0^2 = q_0^2 + q_1^2 +q_1^2 +q_3^2 = 1
\end{equation}

The direction cosine matrix can also be written in terms of quaternions as below

\begin{align}\label{eqn:Cquat}
\begin{split}
C(\bm{q})
 & =\,
\begin{bmatrix}
q_0^2 + q_1^2 - q_2^2 - q_3^2 & 2(q_1 q_2 + q_3 q_0) & 2(q_1 q_3 - q_2 q_0)\\[0.3em]
2(q_1 q_2 - q_3 q_0) & q_0^2 - q_1^2 + q_2^2 - q_3^2 & 2(q_2 q_3 + q_1 q_0)\\[0.3em]
2(q_1 q_3 + q_2 q_0) & 2(q_2 q_3 - q_1 q_0) & -q_1^2 - q_2^2 + q_3^2 + q_0^2\\[0.3em]
\end{bmatrix}
\\
& = (q_0^2 - \bm{q}^{\rm T}\bm{q} )\mathds{1} + 2\bm{q}\bm{q}^{\rm T} - 2q_0\bm{Q}
\end{split}
\end{align}
 
where:

\begin{equation}
\label{eqn:Qmatrix}
\bm{Q}
=\,
\begin{bmatrix}
0 & - q_3 & q_2 \\[0.3em]
q_3 & 0 & - q_1 \\[0.3em]
- q_2 & q_1 & 0\\[0.3em]
\end{bmatrix}
\end{equation}

Given the direction cosine matrix, quaternions can be calculated as:

\begin{align}\label{eqn:quat4}
\begin{split}
\bm{q}
& =\,
\begin{bmatrix}
q_1\\[0.3em]
q_2\\[0.3em]
q_3\\[0.3em]
\end{bmatrix}
 =\,
\frac{1}{4q_0}
\begin{bmatrix}
C_{23} - C_{32}\\[0.3em]
C_{31} - C_{13}\\[0.3em]
C_{12} - C_{21}\\[0.3em]
\end{bmatrix}
\\
q_0
& =\,
\pm{\frac{1}{2}}{(1 + C_{11} + C_{22} + C_{33})}^{\frac{1}{2}}
\end{split}
\end{align}

\subsubsection{Attitude parametrization selection}

In 1776, Euler showed that SO(3) has three dimensions \cite{stuelpnagel1964parametrization}. 
Representations with more than three parameters are subject to constraints. 
Also, \cite{stuelpnagel1964parametrization} states that no parameter set 
can be both global and nonsingular. 
So, we are faced with choosing the representation parameters as either singular or redundant. 
Euler angle representation has its advantages, such as having a clear physical interpretation and minimum parameter set achieving no redundancy. 
However, due to their important disadvantage, that is, the possibility of having singularities when describing motion, Euler angles are not selected to represent attitude in this study.
Another attitude parametrization, the Gibbs vector, is not often used, and can be thought of as an interval step on the way to quaternion parametrization. 
There are other representation types, which are not often preferred, such as the axis-azimuth representation, which will not be discussed in this thesis \cite{bak1999spacecraft}. 
Quaternion (Euler symmetric parameters) representation will be used in this study due to its advantageous nature in simulations, and having no singularities.
Another advantage of quaternions is that the kinematic equations are linear in terms of quaternions. 
Also, quaternion multiplication offers a useful way to express composite rotations. 
With all of these preferable properties, quaternions are the choice for attitude representations for many attitude control missions. Table~\ref{arm:attRepSelection} shows a brief comparison of attitude representations \cite{bak1999spacecraft}.\\
In the previous sections, a variety of parameters to represent rotations are identified and the inherent properties of each technique have been examined. 
Now we examine how to represent attitude depending on time; namely the equations of motion. 
Equations of motion are generally presented in two sections: the kinematic equations and dynamic equations. 
The kinematic equations provide the relationships between the time derivative of the attitude representation and the angular velocity, while the dynamic (or kinetic) equations describe the development of angular velocities under the influence of external moments.

\begin{table}
\caption{Attitude parametrizations of the rotation group SO(3) \cite{bak1999spacecraft}}
\label{arm:attRepSelection}
\begin{center}
 \begin{tabular}{||c || c | c ||}
 \hline
 &Number of  &  \\ [0.5ex] 
Representation &parameter set & Properties \\ [0.5ex] 
 \hline\hline
& & Minimal set\\ 
& & Clear physical interpretation\\ 
& & Often computed directly\\ 
 Euler angles   & 3 & Trigonometric functions in rotation matrix \\ 
& & No simple composition rule \\ 
& & Singular for certain rotations \\ 
& & Trigonometric functions in kinematic relation \\ 
 \hline
& & Easy orthogonality of rotation matrix\\ 
& & Bilinear composition rule\\ 
& & Not singular at any rotation\\ 
Quaternions   & 4 & Linear kinematic equations \\ 
& & No clear physical interpretation \\ 
& &One redundant parameter \\ 
& & Simple kinematic relation \\ 
 \hline
& & Minimal set\\ 
Gibbs vector   & 4 & Clear composition rule\\ 
& & Singular for certain rotations \\ 
& & Simple kinematic relation \\[1ex] 
 \hline
\end{tabular}
\end{center}
\end{table}

%\begin{table}
%\caption{Attitude representations comparison}
%\label{tab:attRepSelection}
%\begin{center}
%\begin{tabular}{||l|l||}\hline
%Representation & Number of parameter set & Properties \\\hline
%our	   & friends \\\hline
%\end{tabular}
%\end{center}
%\end{table}


%\begin{table}
%\caption{Armadillos}
%\label{arm:table}
%\begin{center}
%\begin{tabular}{||l|l||}\hline
%Armadillos & are \\\hline
%our	   & friends \\\hline
%\end{tabular}
%\end{center}
%\end{table}


\subsection{Attitude kinematics}

The time evolution of attitude is identified by a set of first order differential equations called the kinematic equations. 
The angular velocity of a reference frame B with respect to a reference frame A given in the B frame, can be written as follows:

\begin{equation}\label{eqn:quaternion1}
\bm{\omega}_B^{B/A}
 =\,
\omega_1 \bm{b_1} + \omega_2 \bm{b_2} + \omega_3 \bm{b_3}
\end{equation}

Recalling Eq. ~\ref{eqn:C_B_Amatrix} and orthonomality property in Eq. ~\ref{eqn:orthonormality}:

\begin{equation}{\label{eqn:kinematicsDerivation1}}
\begin{bmatrix}
\bm{a_1}\\[0.2em]
\bm{a_2}\\[0.2em]
\bm{a_3}\\[0.2em]
\end{bmatrix}
=\,
\Big[{\bm{C}_B^{B/A}}\Big]^{-1}
\,
\begin{bmatrix}
\bm{b_1}\\[0.2em]
\bm{b_2}\\[0.2em]
\bm{b_3}\\[0.2em]
\end{bmatrix}
=\,
\Big[{\bm{C}_B^{B/A}}\Big]^{\rm T}
\,
\begin{bmatrix}
\bm{b_1}\\[0.2em]
\bm{b_2}\\[0.2em]
\bm{b_3}\\[0.2em]
\end{bmatrix}
\end{equation} 

Taking the time derivative with respect to frame A:

\begin{align}{\label{eqn:C_B_Amatrix_timeDerivative}}
\begin{split}
\frac{d}{dt}
\left.
\begin{bmatrix}
\bm{a_1}\\[0.2em]
\bm{a_2}\\[0.2em]
\bm{a_3}\\[0.2em]
\end{bmatrix}
\right|_A &=\,
\Big[{\dot{\bm{C}}_B^{B/A}}\Big]^{\rm T}
\,
\begin{bmatrix}
\bm{b_1}\\[0.2em]
\bm{b_2}\\[0.2em]
\bm{b_3}\\[0.2em]
\end{bmatrix}
+\,
\Big[{\bm{C}_B^{B/A}}\Big]^{\rm T}
\,
\begin{bmatrix}
\dot{\bm{b_1}}\\[0.2em]
\dot{\bm{b_2}}\\[0.2em]
\dot{\bm{b_3}}\\[0.2em]
\end{bmatrix}
=\,
\Big[{\dot{\bm{C}}_B^{B/A}}\Big]^{\rm T}
\,
\begin{bmatrix}
\bm{b_1}\\[0.2em]
\bm{b_2}\\[0.2em]
\bm{b_3}\\[0.2em]
\end{bmatrix}
+\,
\Big[{\bm{C}_B^{B/A}}\Big]^{\rm T}
\,
\begin{bmatrix}
\bm{\omega} \times \bm{b_1}\\[0.2em]
\bm{\omega} \times \bm{b_2}\\[0.2em]
\bm{\omega} \times \bm{b_3}\\[0.2em]
\end{bmatrix}
\\
& =\,
\Big[{\dot{\bm{C}}_B^{B/A}}\Big]^{\rm T}
\,
\begin{bmatrix}
\bm{b_1}\\[0.2em]
\bm{b_2}\\[0.2em]
\bm{b_3}\\[0.2em]
\end{bmatrix}
+\,
\Big[{\bm{C}_B^{B/A}}\Big]^{\rm T}
\,
\begin{bmatrix}
0 & - \omega_3 & \omega_2 \\[0.3em]
\omega_3 & 0 & - \omega_1 \\[0.3em]
- \omega_2 & \omega_1 & 0\\[0.3em]
\end{bmatrix}
\,
\begin{bmatrix}
\bm{b_1}\\[0.2em]
\bm{b_2}\\[0.2em]
\bm{b_3}\\[0.2em]
\end{bmatrix}
\end{split}
\end{align}

If the skew-symmetric matrix can be represented as $\bm{\Omega}$:

\begin{equation}\label{eqn:Qmatrix}
\bm{\Omega}
=\,
\begin{bmatrix}
0 & - \omega_3 & \omega_2 \\[0.3em]
\omega_3 & 0 & - \omega_1 \\[0.3em]
- \omega_2 & \omega_1 & 0\\[0.3em]
\end{bmatrix}
\end{equation}

and if we rearrange the terms as such:

\begin{equation}{\label{eqn:C_dot_T_C_T_Omega}}
\bigg[\Big[{\dot{\bm{C}}_B^{B/A}}\Big]^{\rm T} - \Big[{\bm{C}_B^{B/A}}\Big]^{\rm T} \bm{\Omega} \bigg]
\,
\begin{bmatrix}
\bm{b_1}\\[0.2em]
\bm{b_2}\\[0.2em]
\bm{b_3}\\[0.2em]
\end{bmatrix}
 =\,
\begin{bmatrix}
0\\[0.2em]
0\\[0.2em]
0\\[0.2em]
\end{bmatrix}
\end{equation}

implies:

\begin{equation}{\label{eqn:C_dot_T_C_T_Omega2}}
\bigg[\Big[{\dot{\bm{C}}_B^{B/A}}\Big]^{\rm T} - \Big[{\bm{C}_B^{B/A}}\Big]^{\rm T} \bm{\Omega} \bigg]
 =\,
\bm{0}
\end{equation}

Taking the transpose and using the relationship $\Omega^{\rm T} = - \Omega$, the kinematic differential equation for the direction cosine matrix can be written as:

\begin{equation}{\label{eqn:kinematicEquDQM}}
{\dot{\bm{C}}_B^{B/A}} + \bm{\Omega} {\bm{C}_B^{B/A}}
 =\,
\bm{0}
\end{equation}

Angular velocity components, given in the B reference frame, can be written as:

\begin{align}{\label{eqn:angVelocityKinematics}}
\begin{split}
\omega_1 & = \dot{C}_{21} C_{31} + \dot{C}_{22} C_{32} + \dot{C}_{23} C_{33} \\
\omega_2 & = \dot{C}_{31} C_{11} + \dot{C}_{32} C_{12} + \dot{C}_{33} C_{13} \\
\omega_3 & = \dot{C}_{11} C_{21} + \dot{C}_{12} C_{22} + \dot{C}_{13} C_{23} \\
\end{split}
\end{align}

From that point, derivation depends on the attitude parameters used. 
When the direction cosines and their derivatives are substituted with their equivalents in terms of Euler angles, the result will give the time dependence of Euler angles. 
In this study however, due to reasons already discussed, quaternions are selected to represent the attitude, so the direction cosines will be written in terms of quaternions:

\begin{align}{\label{eqn:angVelocityQuaternion}}
\begin{split}
\omega_1 & = 2 (\dot{q}_1 q_0 + \dot{q}_2 q_3 - \dot{q}_3 q_2 - \dot{q}_0 q_1) \\
\omega_2 & = 2 (\dot{q}_2 q_0 + \dot{q}_3 q_1 - \dot{q}_1 q_3 - \dot{q}_0 q_2) \\
\omega_3 & = 2 (\dot{q}_3 q_0 + \dot{q}_1 q_2 - \dot{q}_2 q_1 - \dot{q}_2 q_3)\\
\end{split}
\end{align}

The fourth equation comes from the differentiation of the quaternion norm constraint:

\begin{equation}{\label{eqn:kinematicEquDQM}}
0 = 2 ( \dot{q}_0 q_0 + \dot{q}_1 q_1 + \dot{q}_2 q_2 + \dot{q}_3 q_3)
\end{equation}

Writing in matrix form:

\begin{equation}{\label{eqn:kinematicEquAngVel}}
\begin{bmatrix}
0\\[0.2em]
\omega_1\\[0.2em]
\omega_2\\[0.2em]
\omega_3\\[0.2em]
\end{bmatrix}
 =\,
 2\,
\begin{bmatrix}
q_0 & q_1 & q_2 & q_3 \\[0.2em]
-q_1 & q_0 & q_3 & -q_2 \\[0.2em]
-q_2 & -q_3 & q_0 & q_1 \\[0.2em]
-q_3 & q_2 & -q_1 & q_0 \\[0.2em]
\end{bmatrix}
\,
\begin{bmatrix}
\dot{q}_0\\[0.2em]
\dot{q}_1\\[0.2em]
\dot{q}_2\\[0.2em]
\dot{q}_3\\[0.2em]
\end{bmatrix}
\end{equation}
 
As the matrix, which is composed of quaternions, is an orthonormal matrix, the kinematic equations of motion in terms of quaternions can be given as \cite{wie2008space}:

\begin{align} \label{eqn:kinematicArrange}
\begin{split}
\begin{bmatrix}
\dot{q}_0\\[0.2em]
\dot{q}_1\\[0.2em]
\dot{q}_2\\[0.2em]
\dot{q}_3\\[0.2em]
\end{bmatrix}
& =\,
\frac{1}{2}
\,
\begin{bmatrix}
q_0 & -q_1 & -q_2 & -q_3 \\[0.2em]
q_1 & q_0 & -q_3 & q_2 \\[0.2em]
q_2 & q_3 & q_0 & -q_1 \\[0.2em]
q_3 & -q_2 & q_1 & q_0 \\[0.2em]
\end{bmatrix}
\,
\begin{bmatrix}
0\\[0.2em]
\omega_1\\[0.2em]
\omega_2\\[0.2em]
\omega_3\\[0.2em]
\end{bmatrix} \\
& =\,
\frac{1}{2}
\,
\begin{bmatrix}
0 & -\omega_1 & -\omega_2 & -\omega_3 \\[0.2em]
\omega_1 & 0 & \omega_3 & -\omega_2 \\[0.2em]
\omega_2 & -\omega_3 & 0 & \omega_1 \\[0.2em]
\omega_3 & \omega_2 & -\omega_1 & 0 \\[0.2em]
\end{bmatrix}
\begin{bmatrix}
q_0\\[0.2em]
q_1\\[0.2em]
q_2\\[0.2em]
q_3\\[0.2em]
\end{bmatrix}
\end{split}
\end{align}

In a more compact form:

\begin{empheq}[box=\fbox]{align}{\label{eqn:compactKinematics}}
\begin{split}
\dot{q}_0 &= -\frac{1}{2} \bm{q}_\nu^T \bm{\omega}_B^{B/A}\\
\dot{\bm{q}}_\nu &= \frac{1}{2}\Big(\bm{q}_\nu^\times + q_0 \bm{I}_3 \Big) \bm{\omega}_B^{B/A} \\
\end{split}
\end{empheq}

where:

\begin{equation}\label{skew_symmetric}
\bm{x} ^ \times= \begin{bmatrix} 
0 & -x_3 & x_2 \\
x_3 & 0 & -x_1 \\
-x_2 & x_1 & 0 \\
 \end{bmatrix}
\end{equation}

Finally the kinematic equations of motion for the aircraft in Eq.~\ref{eqn:kinematicArrange} (first equation row) can be rewritten as in Eq. \ref{eqn:kinematicArrangeYPR}, since the first column of the matrix on the right side of the equation (first equation row in Eq.~\ref{eqn:kinematicArrange}) is multiplied by zero: 

\begin{equation} \label{eqn:kinematicArrangeYPR}
\begin{bmatrix}
\dot{q}_0\\[0.2em]
\dot{q}_1\\[0.2em]
\dot{q}_2\\[0.2em]
\dot{q}_3\\[0.2em]
\end{bmatrix}
 =\,
\frac{1}{2}
\,
\begin{bmatrix}
-q_1 & -q_2 & -q_3 \\
q_0 & -q_3 & q_2 \\
q_3 & q_0 & -q_1 \\
-q_2 & q_1 & q_0\\
\end{bmatrix}
\,
\begin{bmatrix}
p\\[0.2em]
q\\[0.2em]
r\\[0.2em]
\end{bmatrix} 
\end{equation}

\subsection{Attitude dynamics}

The equation describing the rotational motion of a rigid body moving relative to an inertial frame can written as \cite{wie2008space}

\begin{equation}{\label{eqn:kinematicEquDQM}}
\int \bm{r} \times \ddot{\bm{R}} dm = \bm{M_0}
\end{equation}

Here, the rotation of the rigid body takes place about an arbitrary point O. 
Let us take an infinitesimal mass element $dm$. 
In Fig.\ref{fig:attDyn1}, $\bm{r}$ is the position vector of $dm$ relative to O, $\bm{R}$ is the position vector of $dm$ relative to the origin of the inertial frame, $\ddot{\bm{R}}$ is the acceleration of $dm$, $\bm{M_0}$ is the total external torque about point O.

\begin{figure}
\begin{center}
\includegraphics[width=9cm]{figures/attDyn1}    % The printed column width is 8.4 cm.
\caption{Rigid body rotating about an arbitrary point O, given in the N frame} 
\label{fig:attDyn1}
\end{center}
\end{figure}

\begin{figure}
\begin{center}
\includegraphics[width=10cm]{figures/attDyn2}    % The printed column width is 8.4 cm.
\caption{Body frame B attached to the rigid body, given in the inertial frame N} 
\label{fig:attDyn2}
\end{center}
\end{figure}

Now, let us take a body fixed frame B with origin at the center of mass of the rigid body as shown in Fig. \ref{fig:attDyn2}. 
In Fig. \ref{fig:attDyn2}, $\bm{\rho}$  is the position vector of $dm$ mass with respect to center of mass of the rigid body, $\bm{R}_c$ is the position vector of the center of mass of the rigid body with respect to the origin of the inertial frame N, and $\bm{R}$ is the position vector of $dm$ with respect to the origin of the inertial frame N. 

The angular velocity of the rigid body in the inertial frame N is denoted as $\bm{\omega} \equiv \bm{\omega}_B^{B/N}$. 
It represents the angular velocity of the body frame B with respect to inertial frame N given in the body frame B. 
Angular momentum vector $\bm{H}$ of a rigid body about its center of mass is given by :

\begin{equation}{\label{eqn:kinematicEquDQM}}
\bm{H} = \int \bm{\rho} \times \dot{\bm{R}} dm
\end{equation}

Since $\bm{R}_C$ is constant

\begin{equation}{\label{eqn:kinematicEquDQM}}
\dot{\bm{R}} = \dot{\bm{R}}_C + \dot{\bm{\rho}} = \dot{\bm{\rho}}
\end{equation}

And from rigidity of the body,

\begin{equation}{\label{eqn:kinematicEquDQM}}
\dot{\bm{\rho}} \equiv \bigg\{ \frac{d\bm{\rho}}{dt} \bigg\}_N = {\cancel{\bigg\{ \frac{d\bm{\rho}}{dt} \bigg\}}}_B + \bm{\omega} \times \bm{\rho} = \bm{\omega} \times \bm{\rho}
\end{equation}

Then, the angular momentum is given as :

\begin{equation}{\label{eqn:kinematicEquDQM}}
\bm{H} = \int \bm{\rho} \times \dot{\bm{R}} dm = \int \bm{\rho} \times \dot{\bm{\rho}} dm = \int \bm{\rho} \times  (\bm{\omega} \times \bm{\rho} )dm
\end{equation}

The components of $\bm{\rho}$ and $\bm{\omega}$ in the body frame B are written as;

\begin{align}{\label{eqn:omega_and_rho}}
\begin{split}
\bm{\rho} & = \rho_1 \bm{b}_1 + \rho_2 \bm{b}_2 + \rho_3 \bm{b}_3 \\
\bm{\omega} & = \omega_1 \bm{b}_1 + \omega_2 \bm{b}_2 + \omega_3 \bm{b}_3 \\
\end{split}
\end{align}

Then, the angular momentum can be written as;

\begin{equation}{\label{eqn:angMomentumComp}}
\bm{H} = H_1 \bm{b}_1 + H_2 \bm{b}_2 + H_3 \bm{b}_3 
\end{equation}

where

\begin{align}{\label{eqn:angularMomentum1}}
\begin{split}
H_1 & = I_{11} \omega_1 + I_{12} \omega_2 + I_{13} \omega_3 \\
H_2 & = I_{21} \omega_1 + I_{22} \omega_2 + I_{23} \omega_3 \\
H_3 & = I_{31} \omega_1 + I_{32} \omega_2 + I_{33} \omega_3 \\
\end{split}
\end{align}

Writing in matrix form gives,

\begin{equation}{\label{eqn:angularMomentumInMatrixForm}}
\begin{bmatrix}
H_1\\[0.2em]
H_2\\[0.2em]
H_3\\[0.2em]
\end{bmatrix}
 =\,
\begin{bmatrix}
I_{11} & I_{12} & I_{13} \\[0.2em]
I_{21} & I_{22} & I_{23} \\[0.2em]
I_{31} & I_{32} &I_{33} \\[0.2em]
\end{bmatrix}
\,
\begin{bmatrix}
\omega_1\\[0.2em]
\omega_2\\[0.2em]
\omega_3\\[0.2em]
\end{bmatrix}
\end{equation}

and in a compact form

\begin{equation}{\label{eqn:angMomentumComp2}}
\bm{H} = \bm{I} \bm{\omega}
\end{equation}

where $\bm{I}$  is called the inertia matrix of the rigid body about a body fixed reference frame B with origin at the center of mass of the rigid body.

Let us write a set of axes to achieve all the products of inertia, or all the elements of the inertia matrix except diagonal elements, are zero. 
This set is called the principal axes and the moments of inertia are called the principal moments of inertia. 
Assuming the axes of the body reference frame B are the principal axes, then the equation for the angular momentum becomes;

\begin{equation}{\label{eqn:angularMomentumInPrincipleAxes}}
\begin{bmatrix}
H_1\\[0.2em]
H_2\\[0.2em]
H_3\\[0.2em]
\end{bmatrix}
 =\,
\begin{bmatrix}
I_{11} & 0 & 0 \\[0.2em]
0 & I_{22} & 0 \\[0.2em]
0 & 0 &I_{33} \\[0.2em]
\end{bmatrix}
\,
\begin{bmatrix}
\omega_1\\[0.2em]
\omega_2\\[0.2em]
\omega_3\\[0.2em]
\end{bmatrix}
\end{equation}

Now, it is time to express the rotational equations of motion - also  known as Euler's equations of motion - for a rigid body.  Remember the angular momentum equation

\begin{equation}{\label{eqn:angMomentumComp}}
\bm{M} = \dot{\bm{H}} 
\end{equation}

where $\bm{H}$ is the angular momentum vector of the rigid body about its center of mass and  $\bm{M}$ is the external moment acting on the rigid body about its center of mass.  We can also write it as;

\begin{equation}{\label{eqn:attDynDerivation1}}
\dot{\bm{H}} = \bigg\{ \frac{d\bm{H}}{dt} \bigg\}_N 
=\,
\bigg\{ \frac{d\bm{H}}{dt} \bigg\}_B  + \bm{\omega}_B^{B/N} \times \bm{H}
 =\,
  \bm{M}
\end{equation}

By taking the time derivative of Eq. \ref{eqn:angMomentumComp2}, assuming the inertia is not dependent on time, and evaluating in  Eq. \ref{eqn:attDynDerivation1}, Euler's rotational equation of motion can be written as 

\begin{empheq}[box=\fbox]{equation}{\label{eqn:attitudeDynamics}}
\dot{\bm{\omega}}_B^{B/N}  = \bm{I}_B^{-1} (\bm{M}_B - \bm{\omega}_B^{B/N} \times \bm{I}_B \bm{\omega}_B^{B/N}) 
\end{empheq}

\section{Translation modeling}

The movement of any object can be represented by changes in its location (translation) or changes in its attitude (rotation) or a combination of both. 
Studying aircraft motion is complicated since those two motions are coupled, e.g a rotation might cause a change in aerodynamic forces which affects the translation. 
Thus, to ease the modeling process, the aircraft is assumed to be a point mass, all its mass collected at its center of gravity, while it translates from one point to the other.
Then, the motion of the that point, at its center of gravity, is described by Newton's laws of motion.
The translations are in direct response to external forces, namely the lift, drag, thrust and weight.
Unfortunately some of those forces depend on the attitude of the aircraft.

\subsection{Translational kinematics}

Time change of the position of the aircraft $\bm{x}_N$ expressed in the navigation frame can be written in terms of translational velocity $\bm{v}_B$ expressed in the body frame as

\begin{align}{\label{eqn:translationalKinematics1}}
\begin{split}
\dot{\bm{x}}_N & = \frac{d}{dt} \big( \bm{x}_N \big) \\
                        & = \frac{d}{dt} \big(\bm{C}_B^N \bm{x}_B \big) \\
                        & = \dot{\bm{C}}_B^N \bm{x}_B + \bm{C}_B^N  \dot{\bm{x}}_B \\
                        & = \bm{C}_B^N \bm{v}_B
\end{split}
\end{align}

using $\bm{x}_B=0$. Eq. \ref{eqn:translationalKinematics1} can also be written as : 

\begin{empheq}[box=\fbox]{equation}{\label{eqn:angularMomentumInPrincipleAxes}}
\begin{bmatrix}
\dot{x}_n\\[0.2em]
\dot{x}_e\\[0.2em]
\dot{x}_d\\[0.2em]
\end{bmatrix}
 =\,
\bm{C}_B^N
\,
\begin{bmatrix}
u\\[0.2em]
v\\[0.2em]
w\\[0.2em]
\end{bmatrix}
\end{empheq}

where, $\bm{v}_B = [u \quad v \quad w]^T$ is the inertial velocity of the center of mass of the body expressed in the body frame $B$ and $\dot{\bm{x}}_N = [\dot{x}_n \quad  \dot{x}_e \quad \dot{x}_d]^T$ is the ground speed vector expressed in the navigation frame $N$. $N$ is assumed to be a local inertial frame.
$\bm{C}_B^N$ is the direction cosine matrix which transforms a vector expressed in the body frame to its equivalent expressed in the navigation frame $N$.


\subsection{Translational dynamics}

The aircraft is assumed to be a point mass, all its mass collected at its center of gravity, while it translates from one point to an other.
Then, the motion of that point is described by Newton's laws of motion.
The translations are in direct response to external forces, namely the lift, drag, thrust and weight.
From now on, the aircraft will be assumed to be flying over a small region compared to size of the Earth so that the Earth is locally flat to neglect centripetal acceleration (due to Earth's curvature). 
Another assumption is that the frame attached to Earth being an inertial frame by ignoring Coriolis acceleration so that Newton's laws apply. 

\begin{equation}{\label{eqn:newtonsSecondLaw}}
\sum_j \bm{F}_j = \left. \frac{d}{dt} \big( m \bm{v} \big) \right|_I
\end{equation}

Unfortunately, those forces that directly effect the translation depend on the attitude of the aircraft, complicating the dynamics.

Here, the subscript $I$ represents the frame in which the time derivation occurs, which is an inertial frame in this case. 
To represent time derivation in the inertial frame in terms of time derivation in body frame, relative rotation of the body frame with respect to the inertial frame should be included such that


\begin{equation}{\label{eqn:newtonsDerivationFrameDependency}}
\left. \frac{d}{dt} \big( m \bm{v} \big) \right|_I = \left. \frac{d}{dt} \big( m \bm{v} \big) \right|_B + \bm{\omega}^{B/I} \times \big( m\bm{v} \big)
\end{equation}

Substituting Eq. ~\ref{eqn:newtonsDerivationFrameDependency} in Eq. ~\ref{eqn:newtonsSecondLaw} and then projecting the vector variables in the body frame as well as assuming mass is constant gives


 \begin{equation}{\label{eqn:newtonsSecondLaw2}}
\frac{1}{m} \bigg( \sum_j \bm{F}_{B_j} \bigg)= \left. \frac{d\big(\bm{v}_B \big)}{dt}  \right|_B + \bm{\omega}_B^{B/I} \times \bm{v}_B
\end{equation}

Writing the summation of forces in terms of the forces acting on the aircraft :

\begin{equation}{\label{eqn:dynamicsEquation1}}
\frac{1}{m} \Big( m \bm{g}_B + \bm{F}_{{thrust}_B} + \bm{F}_{{aero}_B} \Big)  =\,
\begin{bmatrix}
\dot{u}\\[0.2em]
\dot{v}\\[0.2em]
\dot{w}\\[0.2em]
\end{bmatrix}
+\,
\begin{bmatrix}
p\\[0.2em]
q\\[0.2em]
r\\[0.2em]
\end{bmatrix}
\times \,
\begin{bmatrix}
u\\[0.2em]
v\\[0.2em]
w\\[0.2em]
\end{bmatrix}
\end{equation}

Rearranging terms and writing the forces in more detail gives :

\begin{empheq}[box=\fbox]{equation}{\label{eqn:dynamicsEquation2}}
\begin{bmatrix}
\dot{u}\\[0.2em]
\dot{v}\\[0.2em]
\dot{w}\\[0.2em]
\end{bmatrix}
=\,
\begin{bmatrix}
{-g \sin{\theta}}\\[0.2em]
{g \sin{\phi} \cos{\theta} }\\[0.2em]
{g \cos{\phi} \cos{\theta}}\\[0.2em]
\end{bmatrix}
+\,
\frac{1}{m}
\,
\begin{bmatrix}
F_{thrust}\\[0.2em]
0\\[0.2em]
0\\[0.2em]
\end{bmatrix}
+\,
\frac{1}{m}
\,
\begin{bmatrix}
X^b\\[0.2em]
Y^b\\[0.2em]
Z^b\\[0.2em]
\end{bmatrix}
-\,
\begin{bmatrix}
qw-rv\\[0.2em]
ru-pw\\[0.2em]
pv-qu\\[0.2em]
\end{bmatrix}
\end{empheq}

which are the equations of motion of the aircraft. 

\section{Drone model}

The calculation of aerodynamic forces and moments that are necessary to solve the equations of motion of a drone will be given in this section. 
During the course of this PhD, two kind of drones have been simulated : a drone from ETH university \cite{ducard2009fault} and MAKO (used in drone lab) given in Fig.~\ref{figure:mako}. 
ETH drone has 2 ailerons, 2 elevators and a rudder as its control surfaces, while MAKO has only 2 control surfaces: elevons. 
An example of an elevon can be seen in the schematic of Zagi given in Fig.~\ref{fig:bodyNEDframes}. 
Changed in the same direction, elevons are used as an elevators (changes the pitch) while changed in reverse direction, they act as ailerons (changes the roll). 

\begin{figure}
\begin{center}
%\includegraphics[width=11cm]{figures/bodyNEDframes}    % The printed column width is 8.4 cm.
\includegraphics[width=13cm]{figures/ZagiElevon}    % The printed column width is 8.4 cm.
\caption{Body fixed frame and North East Down (NED) frame representations} 
\label{fig:bodyNEDframes}
\end{center}
\end{figure}

First, to calculate the moments of inertia of MAKO, it is hanged by two strings, at different orientations, as shown in Fig.~\ref{fig:inertia}, and measurements performed by timing the oscillation period for each axis. 
The resultant moment of inertias are given Table~\ref{arm:MAKOspecs}. 

\begin{figure}
\centering
\includegraphics[width=0.7\columnwidth]{figures/makoEmptyBack}
\caption{MAKO}
\label{figure:mako}
\end{figure}

 \begin{figure*}
      \centering
      \includegraphics[width=0.9\textwidth]{figures/Mako_Inertia_combined_small.png}
      \caption{Moments of inertia measurements for each axis, $I_{xx} , I_{yy} , I_{zz} $.}
      \label{fig:inertia}
 \end{figure*}
 
 \begin{table}
\caption{General specifications of MAKO \cite{bronz2016aerodynamic}}
\label{arm:MAKOspecs}
\begin{center}
\begin{tabular}{ ||p{4cm}|p{3cm}|p{2cm}||}\hline
\textbf{Parameter} & \textbf{Value} & \textbf{Definition} \\\hline
Wing span (b)                  & $\ \ \, 1.288 $	   & $[m]$ \\\hline
Wing surface area (S)      & $ \ \ \, 0.27 $           &  $[m^2]$ \\\hline
Mean aero chord ($\bar{c}$)         & $\ \ \, 0.21$           & $[m]$ \\\hline
Take-off mass (m)             & $\ \ \, 0.7 - 2.0$       & $[kg]$ \\\hline
Flight velocity           & $\ \ \, 10 - 25$       & $[m/s]$ \\\hline
$I_{xx}$                         & $\ \ \, 0.02471284$   & $[kg \cdot m^2]$ \\\hline
$I_{yy}$                         & $\ \ \, 0.015835159$   & $[kg \cdot m^2]$ \\\hline
$I_{zz}$                         & $\ \ \, 0.037424499$   & $[kg \cdot m^2]$ \\\hline
\end{tabular}
\end{center}
\end{table}

 \begin{table}
\caption{Parameters of ETH drone \cite{bronz2016aerodynamic}}
\label{arm:ethDrone}
\begin{center}
\begin{tabular}{ ||p{6cm}|p{3cm}|p{2cm}||}\hline
\textbf{Parameter} & \textbf{Value} & \textbf{Definition} \\\hline
Wing span (b)                & $\ \ \, 3.1 $	   & $[m]$ \\\hline
Wing surface area (S)      & $ \ \ \, 1.80$           &  $[m^2]$ \\\hline
Mean aero chord ($\bar{c}$)          & $\ \ \, 0.58$           & $[m]$ \\\hline
Take-off mass (m)             & $\ \ \, 28$       & $[kg]$ \\\hline
Propeller diameter (D)           & $\ \ \, 28$       & $[m]$ \\\hline
time constant of the engine ($\tau_n$)           & $\ \ \, 28$       & $[m]$ \\\hline
$I_{xx}$                         & $\ \ \, 2.56$   & $[kg \cdot m^2]$ \\\hline
$I_{yy}$                         & $\ \ \, 10.9$   & $[kg \cdot m^2]$ \\\hline
$I_{zz}$                         & $\ \ \, 11.3$   & $[kg \cdot m^2]$ \\\hline
$I_{zx}$                         & $\ \ \, 0.5$   & $[kg \cdot m^2]$ \\\hline
$I_{xz}$                         & $\ \ \, 0.5$   & $[kg \cdot m^2]$ \\\hline
\end{tabular}
\end{center}
\end{table}

\subsection{Modeling of aerodynamic moments}

The attitude of the aircraft changes with the torques applied to the airframe. 

\begin{equation}{\label{eqn:torqueTotal}}
\bm{M}_B
= \,
\begin{bmatrix}
L_{b} \\
M_{b}\\
N_{b}\\
\end{bmatrix}
\end{equation}

Here, $L_b$ is the roll torque, $M_b$ is the pitch torque, and $N_b$ is the yaw torque given in body frame shown in Fig.~\ref{fig:bodyNEDframes}.

The stability derivatives required to calculate the moments are given under Table~\ref{arm:ethcraftStabilityDeriv} for ETH drone and under Table~\ref{arm:MAKOstabilityDeriv} for MAKO. 
Those values for ETH drone are taken from \cite{ducard2009fault} while for MAKO they are calculated via AVL. 
AVL is an open source program developed at MIT and uses vortex-lattice method for the aerodynamic and stability calculations.
The output of the program is linearized at a selected trim condition, therefore all the coefficients are calculated around the equilibrium point at $14m/s$ cruise flight condition.
The center of gravity is located at $X_{CG}= 0.295\,m$, which corresponds to a $8\,\%$ of positive static margin that has been flight tested. %FIXME we have to check if we used modified AVL with viscous add-on or not...

\subsubsection{Roll torque}

Roll torque is given as a multiplication of dynamic pressure $\bar{q}$, wing surface area $S$, wingspan $b$,  and dimensionless roll torque $C_L$ as :

\begin{equation}{\label{eqn:rollTorque}}
L_B = \bar{q} \, S \, b \, C_L
\end{equation}

Here, the dynamic pressure is calculated as :

\begin{equation}{\label{eqn:dynamicPressure}}
\bar{q}=\frac{\rho V_T^2}{2} 
\end{equation}

while the air density $\rho$ is given by the international standard atmosphere model for low altitude (<11000m) as :

\begin{equation}{\label{eqn:airDensityStandard}}
\rho = \frac{p_0 \Big[ 1+\frac{ah}{T_0} \Big]^{5.2561}}{R\,T}
\end{equation}

where the temperature $T$ is given as

\begin{equation}{\label{eqn:temperature}}
T=T_0\Big[ 1 + \frac{ah}{T_0} \Big]
\end{equation}

with $T_0=288.15K$, $a = -6.5 \times 10^{-3} \, K/m$, $R=287.3\;m^2K^{-1}s^{-2}$ and $p_0=1013 \times 10^2 \;Nm^{-2}$.

Next, is to calculate the dimensionless roll torque in Equ.~\ref{eqn:rollTorque}.
For ETH drone, it is given by \cite{stevens2015aircraft,ducard2009fault,mockli2006guidance} :

\begin{table}
\label{arm:momentsETHcraft}
\caption{Stability derivatives for ETH UAV \cite{ducard2009fault}}
\label{arm:ethcraftStabilityDeriv}
\begin{center}
\begin{tabular}{ ||p{3cm}|p{3cm}|p{3cm}||}\hline
\textbf{Parameter} & \textbf{Value} & \textbf{Definition} \\\hline
$C_{L_{a1}} = - C_{L_{a2}}$ & $-3.395 \times 10^{-2}$	   & roll derivative \\\hline
$C_{L_{e1}} = - C_{L_{e2}}$ & $-0.485 \times 10^{-2}$         & roll derivative \\\hline
$C_{L_{\tilde{p}}}$                 & $-1.92 \times 10^{-1}$	   & roll derivative \\\hline
$C_{L_{\tilde{r}}} $                 & $\ \ \, 3.61 \times 10^{-2}$     & roll derivative \\\hline
$C_{L_\beta}$                        & $-1.30 \times 10^{-2}$	   & roll derivative \\\hline
$ C_{M_{1}}$                          & $\ \ \, 2.08 \times 10^{-2}$	   & pitch derivative \\\hline
$C_{M_{a1}} = C_{M_{a2}} $ & $\ \ \, 0.389 \times 10^{-1}$  & pitch derivative \\\hline
$C_{M_{e1}} = C_{M_{e2}} $ & $\ \ \, 2.725 \times 10^{-1}$  &  pitch derivative \\\hline
$C_{M_{\tilde{q}}} $               & $-9.83$	                            & pitch derivative \\\hline
$C_{M_\alpha} $                    & $-9.03 \times 10^{-2}$ 	   & pitch derivative \\\hline
$C_{N_{\delta r}}$                  & $\ \ \, 5.34 \times 10^{-2}$ 	   & yaw derivative \\\hline
$ C_{N_{\tilde{r}}}$                 & $-2.14 \times 10^{-1}$	   & yaw derivative \\\hline
$C_{N_\beta} $                       & $\ \ \, 8.67 \times 10^{-2}$	     & yaw derivative \\\hline
\end{tabular}
\end{center}
\end{table}

\begin{table}
\label{arm:momentsMAKO}
\caption{Stability derivatives for MAKO extracted from AVL program at $14 m/s$ equilibrium cruise speed \cite{bronz2016aerodynamic}}
\label{arm:MAKOstabilityDeriv}
\begin{center}
\begin{tabular}{ ||p{3cm}|p{3cm}|p{3cm}||}\hline
\textbf{Parameter} & \textbf{Value} & \textbf{Definition} \\\hline
$C_{L_a}$                             & $-0.1956 \times 10^{-2}$	   & roll derivative \\\hline
$C_{L_{\tilde{p}}}$                 & $-4.095 \times 10^{-1}$	   & roll derivative \\\hline
$C_{L_{\tilde{r}}} $                 & $\ \ \, 6.203 \times 10^{-2}$     & roll derivative \\\hline
$C_{L_\beta}$                        & $\ \ \, 3.319 \times 10^{-2}$	   & roll derivative \\\hline
$C_{M_0}$ 			     & $\ \ \, 0$  &  pitch derivative \\\hline
$C_{M_e}$ 			     & $-0.076 \times 10^{-1}$  &  pitch derivative \\\hline
$C_{M_{\tilde{q}}} $               & $-1.6834$	                            & pitch derivative \\\hline
$C_{M_\alpha} $                    & $-32.34 \times 10^{-2}$ 	   & pitch derivative \\\hline
$C_{N_0}$                             & $\ \ \, 0$              	   & yaw derivative \\\hline
$C_{N_a}$                             & $-0.0126 \times 10^{-2}$	   & yaw derivative \\\hline
$C_{N_{\tilde{p}}}$                 & $-4.139 \times 10^{-2}$ 	   & yaw derivative \\\hline
$C_{N_{\tilde{r}}}$                 & $-0.1002 \times 10^{-1}$	   & yaw derivative \\\hline
$C_{N_\beta} $                      & $\ \ \, 2.28 \times 10^{-2}$	   & yaw derivative \\\hline
\end{tabular}
\end{center}
\end{table}

\begin{equation}{\label{eqn:bigCraftC_L}}
C_L = C_{L_{a1}} \, \delta_{a1} + C_{L_{a2}} \, \delta_{a2} + C_{L_{e1}} \, \delta_{e1} + C_{L_{e2}} \, \delta_{e2} + C_{L_{\tilde{p}}} \, \tilde{p} + C_{L_{\tilde{r}}} \, \tilde{r} +  C_{L_\beta} \, \beta 
\end{equation}

For MAKO, the dimensionless roll torque is given by :

\begin{equation}{\label{eqn:makoC_L}}
C_L = C_{L_{a}} \, \delta_{a} + C_{L_{\tilde{p}}} \, \tilde{p} + C_{L_{\tilde{r}}} \, \tilde{r} +  C_{L_\beta} \, \beta 
\end{equation}

where $\delta_{a1},\,\delta_{a2}$ are the aileron deflections, $\delta_{e1},\, \delta_{e2}$ are the elevator deflections, $\delta_a$ is the aileron deflection for MAKO, $\beta$ is the sideslip angle representing the angle between airspeed vector $\bm{V_T}$ and the projection of $\bm{V_T}$ onto $x_b-z_b$ plane as shown in Fig.~\ref{fig:windFrame} and given as :

\begin{equation}{\label{eqn:sideslipAngle}}
\beta = \arcsin{\frac{v_T}{V_T}}
\end{equation}

$\tilde{p}, \tilde{r}$ are dimensionless angular rates given by :

 \begin{equation}{\label{eqn:dimensionlessAngularVelocties}}
\tilde{p}=\frac{bp}{2V_T} \qquad \tilde{r}=\frac{br}{2V_T}
\end{equation}

Here, $b$ is the wingspan, $p,q$ are angular rates and $V_T$ is the norm of the airspeed vector such as :

\begin{equation}{\label{eqn:airspeedVector}}
V_T=\sqrt{u_T^2 + v_T^2 + w_T^2}
\end{equation}

Airflow acting on the aircraft is described by airspeed vector $\bm{V_T}$ (see Fig.~\ref{fig:windFrame}). The wind frame is shown in Fig.~\ref{fig:windFrame} with axes $(x_w, y_w, z_w)$ where $x_w$ points along the airspeed vector $\bm{V}_T$.
Airspeed vector in terms of its components in body frame, and in wind frame can be given as

\begin{figure}
\begin{center}
%\includegraphics[width=11cm]{figures/windFrame}    % The printed column width is 8.4 cm.
\includegraphics[width=15cm]{figures/ZagiWindframe}    % The printed column width is 8.4 cm.
\caption{Wind frame, airspeed vector $\bm{V}_T$, angle of attack $\alpha$ and side slip angle $\beta$ representation \cite{ducard2009fault}} 
\label{fig:windFrame}
\end{center}
\end{figure}

\begin{equation}{\label{eqn:airspeedComponents}}
\bm{V}_{\bm{T}_B}
=\,
\begin{bmatrix}
u_{T} \\
v_{T}\\
w_{T}\\
\end{bmatrix}
\qquad \,
\bm{V}_{\bm{T}_W}
=\,
\begin{bmatrix}
V_{T} \\
0\\
0\\
\end{bmatrix}
\end{equation}

The relationship between $\bm{V}_{\bm{T}_B}$ and $\bm{V}_{\bm{T}_W}$ can be written as \cite{ducard2009fault}:

\begin{equation}{\label{eqn:airspeedVectorTransformation}}
\bm{V}_{\bm{T}_B}=\bm{C_W^B}\bm{V}_{\bm{T}_W}
\end{equation}

Note that the inertial velocity of the aircraft $\bm{v}_B = {[u \quad v \quad w]}^T$ is different than airspeed vector $\bm{V}_{\bm{T}_B} = {[u_T \quad v_T \quad w_T ]}^T$. 
Those two vectors are related to wind velocity $\bm{W}$ by :

\begin{equation}{\label{eqn:inertialVelocityAirspeedVectorWind}}
\bm{v}=\bm{V_T} + \bm{W}
\end{equation}

When those vectors are projected on to body frame, the relation becomes

\begin{equation}{\label{eqn:inertialVelocityAirspeedVectorWind}}
\bm{v}_B=\bm{V}_{\bm{T}_B} + \bm{C_N^B} \bm{W}_N
\end{equation}

\begin{figure}
\begin{center}
%\includegraphics[width=11cm]{figures/windDisturbance}    % The printed column width is 8.4 cm.
\includegraphics[width=13cm]{figures/ZagiWindDisturbance}    % The printed column width is 8.4 cm.
\caption{Relation revealed between the inertial velocity vector $\bm{v}$, airspeed vector $\bm{V}_T$ and wind disturbance $\bm{W}$ \cite{ducard2009fault}} 
\label{fig:windDisturbance}
\end{center}
\end{figure}

and with components :

\begin{equation}{\label{eqn:inertialVelocityAirspeedVectorWindComponents}}
\begin{bmatrix}
u \\
v\\
w\\
\end{bmatrix}
=\,
\begin{bmatrix}
u_{T} \\
v_{T}\\
w_{T}\\
\end{bmatrix}
+\,
\bm{C_N^B}
\begin{bmatrix}
W_n \\
W_e\\
W_d\\
\end{bmatrix}
\end{equation}


\subsubsection{Pitch torque}

Pitch torque is given as a multiplication of dynamic pressure $\bar{q}$, wing surface area $S$, mean aero chord $\bar{c}$,  and dimensionless pitch torque $C_M$ as :

\begin{equation}{\label{eqn:pitchTorque}}
M_B = \bar{q} \, S \, \bar{c} \, C_M
\end{equation}

where the dimensionless pitch torque is given as for ETH drone as :

\begin{equation}{\label{eqn:bigCraftC_M}}
C_M = C_{M_{1}} + C_{M_{a1}} \, \delta_{a1} + C_{M_{a2}} \, \delta_{a2} + C_{M_{e1}} \, \delta_{e1} + C_{M_{e2}} \, \delta_{e2} + C_{M_{\tilde{q}}} \, \tilde{q} +  C_{M_\alpha} \, \alpha 
\end{equation}

and for MAKO :

\begin{equation}{\label{eqn:makoC_M}}
C_M =  C_{M_{0}} + C_{M_{e}} \, \delta_{e} + C_{M_{\tilde{q}}} \, \tilde{q} +  C_{M_\alpha} \, \alpha 
\end{equation}

where the variables not introduced up to now are the dimensionless roll rate $\tilde{q}$, elevator deflection $\delta_{e}$ and angle of attack $\alpha$. Angle of attack $\alpha$ is defined as the angle between the projection of the airspeed vector $\bm{V_T}$ onto the $x_b-z_b$ plane and $x_b$ axis as can be seen in Fig.~\ref{fig:windFrame}, and calculated as :

\begin{equation}{\label{eqn:angleOfAttack}}
\alpha = \arctan{\Big( \frac{w_T}{u_T} \Big)}
\end{equation}

\subsubsection{Yaw torque}

Yaw torque can be given as

\begin{equation}{\label{eqn:yawTorque}}
N_B = \bar{q} \, S \, b \, C_N \\
\end{equation}

with a dimensionless yaw torque for the ETH drone : 

\begin{equation}{\label{eqn:bigCraftC_N}}
C_N= C_{N_{\delta r}} \, \delta_{r} + C_{N_{\tilde{r}}} \, \tilde{r} +  C_{N_\beta} \, \beta 
\end{equation}

and for the MAKO :

\begin{equation}{\label{eqn:makoC_N}}
C_N= C_{N_{0}} + C_{N_{a}} \, \delta_{a} + C_{N_{\tilde{p}}} \, \tilde{p} + C_{N_{\tilde{r}}} \, \tilde{r} +  C_{N_\beta} \, \beta 
\end{equation}

where $\bar{q}$ is the dynamic pressure, $V_T$ is the total airspeed of the aircraft, $\rho$ is the air density, $S$ is the wing total surface, $b$ is the wing span, and $\bar{c}$ mean aerodynamic wing chord. 


\subsection{Modeling of aerodynamic forces}

The position of the aircraft changes with the forces applied to the airframe. The calculation of aerodynamic forces, lift force, drag force, lateral force and thrust force are given below.

%%%% LIFT FORCE %%%%

\subsubsection{Lift force}

The lift force is given in the wind frame as : 

\begin{equation}{\label{eqn:liftForce}}
Z^w = \bar{q} \, S \,  C_Z(\alpha)
\end{equation}

where the dimensionless lift coefficient is calculated as below :

\begin{equation}{\label{eqn:liftCoef}}
C_Z(\alpha) = C_{Z_0} + C_{Z_{\alpha}} \alpha 
\end{equation}


%%%% DRAG FORCE %%%%

\subsubsection{Drag force}

Drag force in the wind frame is calculated as a multiplication of the dynamic pressure, wing surface area, and the drag coefficient as given in Equ.~\ref{eqn:dragForceETHcraft} :

\begin{equation}{\label{eqn:dragForceETHcraft}}
X^w = \bar{q} \, S \,  C_X(\alpha, \beta)
\end{equation}

where the dimensionless drag coefficient for ETH drone is given as :

\begin{equation}{\label{eqn:dragCoeffETHcraft}}
C_X(\alpha, \beta) = C_{X_1} + C_{X_{\alpha}} \alpha + C_{X_{\alpha 2}} {\alpha}^2 + C_{X_{\beta 2}} {\beta}^2 
\end{equation}

While for MAKO the drag force is given by \cite{bronz2016aerodynamic} :

\begin{equation}{\label{eqn:dragForceMAKO}}
X^w = \bar{q} \, S \, C_{Z}(\alpha, \beta)
\end{equation}

and the dimensionless drag coefficient for MAKO \cite{bronz2016aerodynamic} :

\begin{equation}{\label{eqn:dragCoeffMAKO}}
C_X(\alpha) = C_{X_0} + C_{X_k} \, C_Z^2 = C_{X_0} + C_{X_k} \, (C_{Z_0} + C_{Z_{\alpha}} \alpha )^2
\end{equation}

%%%% LATERAL FORCE %%%%

\subsubsection{Lateral force}

The lateral force is given as :

\begin{equation}{\label{eqn:lateralForceETHcraft}}
Y^w = \bar{q} \, S \,  C_Y(\beta)
\end{equation}

where the dimensionless lateral coefficient for ETH drone is given as :

\begin{equation}{\label{eqn:lateralCoeffETHcraft}}
C_Y(\beta) = C_{Y_\beta} \beta
\end{equation}

and for MAKO :

\begin{equation}{\label{eqn:lateralCoeffMAKO}}
C_Y(\beta, \, \tilde{p}, \, \tilde{r}, \, \delta_a) = C_{Y_\beta} \beta + C_{Y_{\tilde{p}}} \, \tilde{p} + C_{Y_{\tilde{r}}} \, \tilde{r} + C_{Y_a} \, \delta_a 
\end{equation}

where the coefficients can be seen in Tables~\ref{arm:forcesETHcraft} and~\ref{arm:forcesMAKO}.

\begin{table}
\caption{Aerodynamic force derivatives for ETH UAV \cite{ducard2009fault}}
\label{arm:forcesETHcraft}
\begin{center}
\begin{tabular}{ ||p{3cm}|p{3cm}|p{4cm}||}\hline
\textbf{Parameter} & \textbf{Value} & \textbf{Definition} \\\hline
$C_{Z_0}$                             & $\ \ \,1.29 \times 10^{-2}$	   & lift derivative \\\hline
$C_{Z_{\alpha}}$                   & $-3.25 $                                 & lift derivative \\\hline
$C_{X_1}$                             & $-2.12 \times 10^{-2}$	   & drag derivative \\\hline
$C_{X_{\alpha}}$                   & $-2.66 \times 10^{-2}$          & drag derivative \\\hline
$C_{X_{\alpha 2}}$                & $-1.55 $	                            & drag derivative \\\hline
$C_{X_{\beta 2}} $                 & $-4.01 \times 10^{-1}$	   & drag derivative \\\hline
$C_{Y_\beta} $                      & $-3.79 \times 10^{-1}$          & side force derivative \\\hline
\end{tabular}
\end{center}
\end{table}


\begin{table}
\caption{Aerodynamic force derivatives for MAKO extracted from AVL program at $14 m/s$ equilibrium cruise speed\cite{bronz2016aerodynamic}}
\label{arm:forcesMAKO}
\begin{center}
\begin{tabular}{ ||p{3cm}|p{3cm}|p{4cm}||}\hline
\textbf{Parameter} & \textbf{Value} & \textbf{Definition} \\\hline
$C_{Z_0}$                             & $-8.53 \times 10^{-2}$	         & lift derivative \\\hline
$C_{Z_{\alpha}}$                   & $\ \ \,3.9444$                               & lift derivative \\\hline
$C_{Z_q}$                             & $\ \ \,4.8198$	       		         & lift derivative \\\hline
$C_{Z_e}$                             & $\ \ \,1.6558 \times 10^{-2}$        & lift derivative \\\hline
$C_{X_0}$                             & $\ \ \, 2.313 \times 10^{-2}$	   & drag derivative \\\hline
$C_{X_k}$                              & $\ \ \, 1.897 \times 10^{-1}$          & drag derivative \\\hline
$C_{Y_\beta}$ 			     & $-2.708 \times 10^{-1}$             & side force derivative \\\hline
$C_{Y_{\tilde{p}}}$                 & $\ \ \, 1.695 \times 10^{-2}$	& side force derivative \\\hline
$C_{Y_{\tilde{r}}}$                  & $\ \ \, 5.003 \times 10^{-2}$ 	& side force derivative \\\hline
$C_{Y_a}$                             & $\ \ \, 0.0254 \times 10^{-2}$	& side force derivative \\\hline
\end{tabular}
\end{center}
\end{table}

%%%% THRUST FORCE %%%%

\subsubsection{Thrust force model}

The thrust generated by the propeller can be written as :

\begin{equation}{\label{eqn:thrustModel}}
F_T = \rho n^2 D^4 C_{F_T}
\end{equation}

where the dimensionless thrust coefficient for ETH drone is calculated as :

\begin{equation}{\label{eqn:thrustETHcraft}}
C_{F_T} = C_{F_{T1}} + C_{F_{T2}} J + C_{F_{T3}} J^2 
\end{equation}

Here, $J$ is the advance ratio, and for ETH drone, it is given by

\begin{equation}{\label{eqn:advRatETHcraft}}
J = \frac{V_T}{n \pi D}
\end{equation}

\begin{table}
\label{arm:ETHcraft}
\caption{Thrust force coefficients for propeller ETH UAV \cite{ducard2009fault}}
\label{arm:thrustForce}
\begin{center}
\begin{tabular}{ ||p{3cm}|p{3cm}|p{4cm}||}\hline
\textbf{Parameter} & \textbf{Value} & \textbf{Definition} \\\hline
$C_{F_{T1}} $                 & $\ \ \, 8.42 \times 10^{-2}$	   & thrust derivative \\\hline
$C_{F_{T2}} $                 & $-1.36 \times 10^{-1}$	           & thrust derivative \\\hline
$C_{F_{T3}} $                 & $-9.28 \times 10^{-1}$                 & thrust derivative \\\hline
$D$                                 & $\ \ \, 0.79 \, m$                           & propeller diameter \\\hline
\end{tabular}
\end{center}
\end{table}

%%%% TAken from 
The dimensionless thrust coefficient for MAKO is given by \cite{bronz2017flight} :

\begin{equation}{\label{eqn:thrustCoefMAKO}}
C_{F_T} = C_{F_{T1}} + C_{F_{T2}} \cdot J^\prime + C_{F_{T_{rpm}}} \cdot n \cdot 60 
\end{equation}

with advance ratio :

\begin{equation}{\label{eqn:advRatMAKO}}
J^\prime = \frac{V_T}{n D}
\end{equation}

\begin{table}
\label{arm:MAKO}
\caption{Thrust force coefficients for propeller APC SF $9 \times 6$ from wind tunnel experiments \cite{bronz2017flight}}
\label{arm:thrustForce}
\begin{center}
\begin{tabular}{ ||p{3cm}|p{3cm}|p{4cm}||}\hline
\textbf{Parameter} & \textbf{Value} & \textbf{Definition} \\\hline
$C_{F_{T1}}$                   & $\ \ \, 1.342 \times 10^{-1}$	   & thrust derivative \\\hline
$C_{F_{T2}}$                 & $-1.975 \times 10^{-1}$	          & thrust derivative \\\hline
$C_{F_{T_{rpm}}}  $           & $\ \ \, 7.048 \times 10^{-6}$    & thrust derivative \\\hline
$D$                              & $\ \ \, 0.228 \, m$                          & propeller diameter \\\hline
\end{tabular}
\end{center}
\end{table}

\subsection{Shortcut to modeling}

After the derivation of the aircraft flight kinematics \& dynamics for translational and attitude motion, the system of first order differential equations can be summarized as follows :

\begin{empheq}[box=\fbox]{align}{\label{eqn:compactEquOfMotion}}
\begin{split}
\dot{\bm{x}}_{N} &= \bm{C}_B^N \bm{v}_B\\
\dot{\bm{v}}_B &= \frac{1}{m} \big[ m \bm{g}_B + \bm{F}_{t_B} + \bm{F}_{a_B} \big] - {\big[ {\bm{\omega}^{B/N}_B} \big] }^\times \bm{v}_B  \\
\dot{q}_0 &= -\frac{1}{2} \bm{q}_\nu^T \bm{\omega}^{B/N}_B\\
\dot{\bm{q}}_\nu &= \frac{1}{2}\Big(\bm{q}_\nu^\times + q_0 \bm{I}_3 \Big) \bm{\omega}^{B/N}_B \\
\bm{J} \dot{\bm{\omega}}^{B/N}_B &= \bm{M}_B - { \big[ {\bm{\omega}^{B/N}_B} \big]}^\times \bm{J} \bm{\omega}^{B/N}_B\\
\end{split}
\end{empheq}

where $\bm{x}_{N} \in {\rm I\!R^3}  $ is the position of the center of mass of UAV in navigation frame $N$, $\bm{v}_B$ is the velocity of the center of mass of UAV in body frame $B$,  $\bm{q} = [q_0, \bm{q}_v^T] ^T \in {\rm I\!R^3} \times {\rm I\!R}$ is the unit quaternion representing the attitude of the body frame $B$ with respect to navigation frame $N$ expressed in the body frame $B$, $\bm{\omega}^{B/N}_B$ is the angular velocity of the body frame $B$ with respect to navigation frame $N$ expressed in the body frame $B$, $ \bm{J} \in {\rm I\!R^{3\times3}}  $ is the positive definite inertia matrix of the drone, $\bm{M}_B \in {\rm I\!R^3}$ represents the moments acting on the drone, $ \bm{C}_B^N$ is the direction cosine matrix which transforms a vector expressed in the body frame to its equivalent expressed in the navigation frame $N$, $\bm{I}_3  \in {\rm I\!R^{3\times3}}$ is the identity matrix, $\bm{F}_{t_B} \in {\rm I\!R^3}$ is the thrust force expressed in the body frame,  $\bm{F}_{a_B} \in {\rm I\!R^3}$ are the aerodynamic forces given in the body frame $B$. 
The navigation frame is assumed to be a local inertial frame in which Newton's Laws apply. 
The notation $\bm{x} ^{\times} $ for a vector $x = [x_1 \quad x_2 \quad x_3]^ {\rm{T}}$ represents the skew-symmetric matrix as given in Equ.~\ref{skew_symmetric}.

\iffalse
\subsection{Verification with Matlab Simulink 6DOF block}

To validate the written translational and attitude motion dynamics and kinematics, MATLAB Simulink \textit{6DOF} block has been used.
This block accepts inputs as the forces and moments and outputs the states of aircraft motion (Fig.~\ref{figure:validationSimulink}).
To compare the generated model and Simulink \textit{6DOF} block, forces and moments have been calculated via equations and constants given in Appendix A and Appendix B.
The simulated states from the model script have been saved in advance and called from Simulink by \textit{From Workspace} blocks then compared with the \textit{6DOF} outputs.
The difference has been found to be negligible indicating the validity of the model. 


\begin{figure*}
\center
\includegraphics[width=1.1\columnwidth]{figures/validationViaSimulink}
\caption{Validation with Simulink 6DOF aircraft model. The scripted equations of motion and numeric integration are validated via Simulink's 6DOF Block. Forces and moments are directly given to the 6DOF block and the calculated states are compared with the ones calculated by the script in Matlab.}
\label{figure:validationSimulink}
\end{figure*}

\fi

\subsection{Sensor Models}

Accelerometer and gyro measurements are simulated using the angular velocity $\bm{\omega}^{B/N}_B$ and translational acceleration $\dot{\bm{v}}_B$ given by the system of equations of drone summarized in Equ.~\ref{eqn:compactEquOfMotion} and the specifications of the hardware used in \emph{Apogee Autopilot} of \emph{Paparazzi Autopilot System}.
The sensor suit simulated is the InvenSense MPU-9250 Nine-axis (Gyro + Accelerometer + Compass) MEMS MotionTracking Device.
 
The accelerometer and gyro data is simulated as 
 
 \begin{align}
\bm{z}_{gyro} &= \bm{k}_{gyro} \bm{\omega}_{B/I}^B + \bm{\beta}_{gyro} + \bm{\eta}_{gyro}\\
\bm{z}_{acc} &= \bm{k}_{acc} \bm{\omega}_{B/I}^B + \bm{\beta}_{acc} + \bm{\eta}_{acc}
% hic bir sey yazmazsan esitlikleri alt alta hizaliyor canim benim&=alo \\
% $ tek dolar arasi $ inline denklem
% $$ cift dolar arasi $$ satir atlayarak ortada denklem
\end{align}

where $\bm{\beta}$ is the bias, and $\bm{\eta}$ is the zero mean Gaussian process with $\bm{\sigma}^2$ variance with values given in Table ~\ref{arm:sensorSpecs}.

\begin{table}[!htbp]
\caption{Specifications of the sensor suit InvenSense MPU-9250 Nine-axis (Gyro + Accelerometer + Compass) MEMS MotionTracking Device\cite{condomines2015developpement}}
\label{arm:sensorSpecs}
\begin{center}
\begin{tabular}{ ||p{3cm}|p{2cm}|p{1.5cm}||}\hline
\textbf{Measurement} & $ \bm{\beta}$ &  $ \bm{\sigma}$ \\\hline
${\bm{z}_{acc}}_x$                  & $\ \ \, 0.142 $	   & $\ \ \, 0.0319$ \\\hline
${\bm{z}_{acc}}_y$       & $ -0.3 $           &  $\ \ \, 0.0985$ \\\hline
${\bm{z}_{acc}}_z$           & $\ \ \, 0.19$           & $\ \ \, 0.049$ \\\hline
${\bm{z}_{gyro}}_x$                  & $-1.55 $	   & $\ \ \, 0.0825$ \\\hline
${\bm{z}_{gyro}}_y$       & $ -1.13 $           &  $\ \ \, 0.1673$ \\\hline
${\bm{z}_{gyro}}_z$           & $-1.7$           & $\ \ \, 0.2214$ \\\hline
\end{tabular}
\end{center}
\end{table}


\subsection{Fault Models}

Probable faults in the control surfaces can be mainly grouped under two main categories \cite{zhong2014contribution}: 

\begin{itemize}
\item{A total control loss of the control surface actuators. The actuator does not respond to the control signals at all. Lock-in-place, hard-over and floating around trim are such failures (see Fig.~\ref{fig:actuatorFaults}).}
\item{A partial control loss of the control surface actuators. The actuator does respond to the control signals but does so in an abnormal way. Loss of effectiveness failure is shown in Fig.~\ref{fig:actuatorFaults}.}
\end{itemize}

When the actuators are healthy, actual control input signal will be equal to the given input signal.
In case of a fault the actual signal can be modeled as

\begin{equation}
\bm{u}\left(t\right)= \bm{E}\bm{u}_c + \bm{u}_f
\end{equation}

where $\bm{u}_c $ is the desired control signal, $\bm{E} = diag(e_1, e_2, e_3)$ is the effectiveness of the actuators where $0 \leq e_i \leq 1 $ with $(i = 1, 2 ,3)$ and $\bm{u_f}$ additive actuator fault. This model makes it possible to simulate all four types of actuator faults shown in Fig.~\ref{fig:actuatorFaults}.

%\begin{equation}
%\bm{u}\left(t\right)= \begin{bmatrix} {\delta}_{a}\ {\delta}_{e}\ n \end{bmatrix}^{\rm T}
%\end{equation}

%Here $ \delta_{a}$ aileron deflection angle in degrees, $ \delta_{e}$ elevator deflection angle in degrees, $n$ engine speed in rev/s. 


%\begin{figure}
%\begin{center}
%\includegraphics[width=11cm]{figures/sensorFaults}    % The printed column width is 8.4 cm.
%\caption{Probable sensor faults \cite{ducard2009fault}} 
%\label{fig:sensorFaults}
%\end{center}
%\end{figure}

\begin{figure}
\begin{center}
\includegraphics[width=14cm]{figures/actuatorFaults}    % The printed column width is 8.4 cm.
\caption{Probable actuator faults \cite{ducard2009fault}. (a) Loss of effectiveness: The actuator does respond to the control signals but does so in an abnormal way such as low actuation level or low response time. (b) Lock-in-place: A total control loss of the control surface actuators. The actuator freezes at a particular position. (c) Hard-over: A total control loss of the control surface actuators. The actuator freezes at the minimum or maximum position limit. (d) A total control loss of the control surface actuators. Actuator does not contribute to the control authority.} 
\label{fig:actuatorFaults}
\end{center}
\end{figure}




%\begin{equation}
%\bm{\dot{x}}\left(t\right) = \bm{f}\left(\bm{x}\left(t\right) , \bm{u}\left(t\right) \right) 
%\end{equation}


%\begin{align}
%{{\mu }_{nf}}\frac{{{\partial }^{2}}u}{\partial {{y}^{2}}}+{{\left( \rho \beta  \right)}_{nf}}g\sin \phi \left( T-{{T}_{w2}} \right)-{{\sigma }_{nf}}B_{0}^{2} \sin^{2}\left( \alpha + \phi \right) u&=\frac{\partial p}{\partial x}  \label{equ1} \\
%\frac{{{\partial }^{2}}T}{\partial {{y}^{2}}}+\frac{{{\mu }_{nf}}}{{{k}_{nf}}}{{\left( \frac{\partial u}{\partial y} \right)}^{2}}+\frac{{{\sigma }_{nf}}}{{{k}_{nf}}}B_{0}^{2} \sin^{2}\left( \alpha + \phi \right){{u}^{2}}&=0 \label{equ2} \ 
% hic bir sey yazmazsan esitlikleri alt alta hizaliyor canim benim&=alo \\
% $ tek dolar arasi $ inline denklem
% $$ cift dolar arasi $$ satir atlayarak ortada denklem
%\end{align}
%%%%% End of Eq 1 & Eq 2 %%%%%
%The system of equations in  Eq (1-2) is subject to boundary conditions given in (3-4)
%%%%%% Eq 3 & Eq 4 %%%%%%
%\begin{align}
%u(H / 2)&=0   & u(-H / 2)&=0\\
%T(H / 2)&= {T}_{w1} & T(- H / 2)&= {T}_{w2}0
%\end{align}

%\begin{figure}
%\begin{center}
%\includegraphics[width=8.3cm]{FTCmethods}    % The printed column width is 8.4 cm.
%\caption{Variations of fault tolerant control systems } 
%\label{fig:FTCmethods}
%\end{center}
%\end{figure}

%\begin{table}
%\caption{Attitude representations comparison}
%\label{tab:attRepSelection}
%\begin{center}
%\begin{tabular}{||l|l||}\hline
%Representation & Number of parameter set & Properties \\\hline
%our	   & friends \\\hline
%\end{tabular}
%\end{center}
%\end{table}

\section{Conclusion}
In this chapter, equations of motion of an aircraft are given. 
Motion of an aircraft usually involves both translation and rotation. 
Here, equations for translational and rotational motion are discussed separately in detail.
After the equations are derived for a generic aircraft, calculation of forces and moments which are specific to an individual drone is presented (aerodynamic force derivatives and stability derivatives are specific for each drone). For two different drone examples, a drone from ETH Zurich and MAKO (used in ENAC UAV LAB), calculation of forces and moments are given.
Those forces and moments are inputs to the dynamic equations of motion. 

The models derived here are not used in detection and diagnosis algorithms. 
The detection and diagnosis algorithms used in this thesis use only data. 
The need for data is the driving factor to simulate the drone motion. 
Thus, data is simulated using MAKO drone model and specifications of IMU \emph{InvenSense MPU-9250 Nine-axis} used in \emph{Paparazzi Autopilot Apogee} onboard. 
 
%To clarify, machine learning methods also train a model. 
%But in that case, the model is not the equations of motion of an aircraft but rather a simpler model and may not explicitly describe the physical . 
%The only exception to not using physical models in this work for diagnosis, is the part that the diagnosis is realized using spinors as features. To calculate the spinors, kinematic equations are numerically. But still, our idea to generalize by having no models holds since only kinematic equations are used. 

In this thesis, the diagnosis is implemented on two types of data: 1. Simulated data, 2.
Flight data. 
If the reader is not interested in detection and diagnosis via simulated data, s/he can skip this chapter since this chapter explains the models that are used to simulate measurements.
Even then, having a background information on the physics of the system might help to have a grasp on the features (translational acceleration and angular velocities) used in both model-based and data-driven fault diagnosis.

Since the data is generated and available now for implementations, the methodology used in this thesis to diagnose faults are discussed in the next chapter. 
In this thesis, \emph{Support Vector Machines} (SVM) are used as a classification method to diagnose faults onboard a drone. 
Since SVM is a machine learning method, an introduction to machine learning methods have been given from an implementation perspective. 
It starts with an introduction to terminology for machine learning methods, gives general explanations to prepare the inputs to the machine learning algorithm. 
Then, many application details are discussed that would help the user to apply those methods to their own problems. 
Finally, SVM is introduced and three phases encountered during its implementation, \emph{training}, \emph{tuning} and \emph{evaluation} are explained.
