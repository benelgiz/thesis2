%% This is an example first chapter.  You should put chapter/appendix that you
%% write into a separate file, and add a line \include{yourfilename} to
%% main.tex, where `yourfilename.tex' is the name of the chapter/appendix file.
%% You can process specific files by typing their names in at the 
%% \files=
%% prompt when you run the file main.tex through LaTeX.

\chapter{Introduction}

Lately, the popularity and reachability of Unmanned Aircraft System (UAS) have risen steeply thanks to the advancements in electronic components and their decrease in cost. 
This accelerating trend towards small but capable flying vehicles is extending the limits of hardware and software potential in industry and academia.
Regulators from the aviation community have become increasingly concerned by the number of incidences of drones flying close to civil aircraft and airports. 
With the advent of the new era of UAS, different institutions around the world are addressing safe integration of UAS into airspace \cite{baskaya2016flexible}, specifically the National Aeronautics and Space Administration (NASA) 
\cite{kopardekarunmanned} and the Federal Aviation Administration (FAA) \cite{FAA_UASintegration} in the US, European Aviation Safety Agency (EASA) \cite{A_NPA_EASA2015} in Europe and international bases such as International Civil Aviation Organization (ICAO) \cite{ICAO_Circular}.
The U-Space concept in Europe (UTM in the US, which aims at enabling safe integration into airspace), provides the insight by showing in their roadmaps that the level of drone automatization and the level of drone connectivity (drone to drone and drone to infrastructure) will guide the pace of the U-Space services (U1 to U4) \cite{undertaking2017u}. 
These enablers will allow intelligent agents to share information and automate complex procedures in case of emergencies. 
Thus, future aviation will inevitably move towards automatization. 
Here we propose a concept that will contribute to the automatization of drones, to make them more intelligent and thus contribute to their safer integration into the airspace. 
These awareness abilities allow the mitigation of risks in accordance with the risk assessment procedures offered by JARUS (SORA) \cite{SORA} defined by \cite{EASAopinion2018}. 
The functioning abilities of a drone following an emergency should be assessed and, depending on the availability of the environment, a recovery procedure may be initiated. 
For cases in which recovery is not possible, safe ditching may be required in order to reduce the potential harm in the air or on the ground. 

\section{Motivation}

Improving the flight reliability is considered to be one of the main concerns for integrating UAVs into civil airspace, according to the Unmanned Systems Roadmap by the US Office of the Secretary of Defense, DoD \cite{UnmannedSystemsRoadmapDoD}. 
Achieving safe flight is not a straightforward task, considering the multiplexity of unknowns in the system hardware, environment, and possible system faults/failures yet to emerge. 
The expectation that UAVs should be less expensive than their manned counterparts effects their reliability. Cost saving measures --- other than the need to support a pilot/crew onboard or a reduction in size --- may lead to a decrease in system reliability.

Under the research and development programs and initiatives identified by the DoD to develop technologies and capabilities for UAS, the greatest area of control technologies is health management and adaptive control, with a budget of 74.3 m dollars. 
Other safety features such as the validation and verification of flight critical intelligent software are the second area, with 57.8 m dollars \cite{UnmannedSystemsRoadmapDoD}. 

Systems are often susceptible to faults of differing nature. Existing irregularities in sensors, actuators or controller may be intensified due to the control system design and lead to failures. A fault may be hidden due to the control action \cite{ducard2009fault}.

The most widely-used method to increase reliability is to increase the use of more reliable components and/or hardware redundancy. Both require an increase in the cost of the UAS, conflicting with one of the main reasons of UAS' popularity \cite{angelov2012sense}. In order to provide solutions for the different foreseen categories within the airspace, a variety of approaches should be considered. While hardware redundancy may cope with failure situations of UAVs in certified operations, it may not be suitable for UAVs in open category operations or some operations in specific category, due to budget constraints. Analytical redundancy is another solution. This may be not as effective and straightforward as hardware redundancy, but relies on the design of intelligent methods in order to utilize the information on board the aircraft to deal with all potential circumstances.  

\section{Contribution}

Integration of drones into airspace needs the introduction of indigenous designs that will serve safe solutions for drones. This necessitates intelligent approaches to address the emergency situations. One of the aspects of the problem is to assure a safe flight by designing fault detection and diagnosis with cheaper avionics common in a vast number of drones.
The hardware limitations for these small vehicles point the utilization of \emph{analytical redundancy} rather than the usual practice of hardware redundancy in manned aviation. 
For that purpose, we introduce an end-to-end design to achieve data-driven fault diagnosis for control surface faults on drones.
In the course of this thesis, machine learning practices are implemented to diagnose faults on a small fixed-wing UAV to avoid the burden of accurate modeling needed in model-based fault diagnosis. 
We aim to design a classifier via SVM to solve FDD as a classification problem for drones with actuator faults.
All data and code are available in code sharing and versioning system \emph{Github}. 

In this thesis, fault classification simulations are investigated under two main sections: classification of faults based on simulated flight measurements and classification of faults based on real flight data. 

For classification on data generated from simulations, model of a MAKO UAV \cite{baskaya2017flight} is simulated.
Sensor measurements (accelerometer and gyro data) have been calculated using information of drone's motion and the specifications of the sensors. 
Generated data is usually more structured compared to the real flight data. 
In this preliminary application of SVM to fault diagnosis, we aimed to start with an easier problem, and used data generated from models.
The results show that SVM classifier was very accurate and fast in diagnosing the fault on the control surfaces with a classification accuracy of $99.999\%$.

Accurate results on classification of faults with generated data encouraged us to go further and investigate fault detection with real flight data. 
Since SVM is a supervised classification method, labeled data is necessary to train the algorithm. For that reason, real flights have been arranged to generate faulty flight data by manipulating the open source autopilot \emph{Paparazzi}.  
The training is held offline due to the need for labeled data and computational burden of the tuning phase of the classifiers. 
Two types of faults have been mainly investigated, the control surface stuck and loss of effectiveness of the elevon. Results indicate that the control surface stuck can be detected relatively easily with 3 gyros and 3 accelerometers data compared to loss of effectiveness (efficiency). 
The results show that over the flight data, SVM yields an F1 score of 0.98 for classification of control surface stuck fault. 
Addition of features to accommodate previous measurements improve classification performance for tuned classifiers while the untuned classification performance deteriorates. 
Classification performs poorly for loss of efficiency faults especially for smaller ineffectiveness values. 
For the loss of efficiency fault, some feature engineering, involving the addition of past measurements is needed to attain the same classification performance.

A very promising result is discovered when \emph{spinors} are used as features instead of angular velocities. 
Result show that by using spinors for classification, there is a big improvement in the classification accuracy especially when the classifiers are untuned. Using spinors and a Gaussian Kernel, untuned classifier gives F1 score of 0.9555 which is 0.2712 when gyro measurements are used as features.

In general, this work shows that SVM gives satisfactory performance for classification of faults on control surfaces of a drone using flight data.

\section{Thesis Outline}

\textbf{Chapter 2} presents the state of the art on the following topics: 
\begin{itemize}
\item{Integration of drones into airspace}
\item{Paparazzi Autopilot}
\item{Fault Tolerant Control Systems with a focus on Fault Detection and Diagnosis}
\item{Machine Learning and Artificial Intelligence in general}
\end{itemize}
Integration for drones into the airspace constitutes the motivation of this thesis. To ensure a safe integration of drones into the airspace, design of vehicle health management systems is crucial. Fault detection and diagnosis plays an important role in vehicle health management, thus literature review of that topic is also presented in this chapter. In this thesis, the method selected to detect and diagnose faults is Support Vector Machines, which is a machine learning method. Thus, state of the art for machine learning is also given. \emph{Paparazzi} autopilot has been used to realize flights to generate faulty data, thus it is also introduced in this chapter.\\
\textbf{Chapter 3} focuses on equations of motion of an aircraft. In this chapter, equations for translational and rotational motion are discussed separately in detail. After the equations are derived for a generic aircraft, calculation of forces and moments which are specific to an individual drone is presented. Then, modeling accelerometer and gyro data is explained. Modeling faults in the control surfaces of an aircraft is also discussed in this chapter.\\
\textbf{Chapter 4} is dedicated to machine learning algorithms with a focus on their implementations. After the general practices of machine learning applications are presented, Support Vector Machines (SVM) is discussed in more detail.\\
\textbf{Chapter 5} focuses on the results of fault classification results under two main sections: classification of faults based on simulated  flight measurements and classification of faults based on real flight data. In the first part, flight data is simulated using the mathematical equations explained in Chapter 3. SVM algorithm is trained with the simulated data to classify the faulty and nominal phases of the simulated flight. The second part starts with explanations on faulty flight experiments realized. \emph{Paparazzi} Autopilot has been modified to introduce faults to control surfaces during the flight. Then, classification of control surface stuck and loss of effectiveness faults have been investigated separately.\\
\textbf{Chapter 6} concludes the thesis with an outlook on the contributions and results acquired during the thesis.