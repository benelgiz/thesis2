%% This is an example first chapter.  You should put chapter/appendix that you
%% write into a separate file, and add a line \include{yourfilename} to
%% main.tex, where `yourfilename.tex' is the name of the chapter/appendix file.
%% You can process specific files by typing their names in at the 
%% \files=
%% prompt when you run the file main.tex through LaTeX.

\chapter{Introduction}


Lately, the popularity and reachability of UAS have risen steeply thanks to the advancements in electronic components and yet the decrease in their cost. 
This accelerating trend towards small but capable flying vehicles is pushing the limits of both hardware and software potentials of industry and academia. 
This called the regulators to duty due to the concerns from the manned aviation community after incidences of drones flying close to civil aircrafts and airports as well as the public witnessing a number of accidents. 
Increasing usage of these vehicles for a variety of missions pushes a further liability to secure the flight.
With the advent of the new era of UAS, different institutions all over the world, specifically National Aeronautics and Space Administration (NASA) 
\cite{kopardekarunmanned} and Federal Aviation Administration (FAA) \cite{FAA_UASintegration} in US, European Aviation Safety Agency (EASA) \cite{A_NPA_EASA2015} in Europe and international bases such as International Civil Aviation Organization (ICAO) \cite{ICAO_Circular} are addressing safe integration of UAS in airspace \cite{baskaya2016flexible}.
The U-Space concept in Europe (UTM in the US, which aims at enabling safe integration into airspace), gives the insight by showing in their roadmaps that level of drone automatization and level of drone connectivity (drone to drone and drone to infrastructure) will guide the pace of the U-Space services (U1 to U4) \cite{undertaking2017u}. 
These enablers will allow intelligent agents to share information and automate complex situations such as emergencies. 
Thus future aviation will inevitably go towards automatization. 
Here we propose a concept that will contribute to the automatization of drones, that will make them more intelligent and thus contribute to a safer integration of drones into airspace. 
These awareness abilities allow to mitigate risks in accordance with the risk assessment procedures offered by JARUS (SORA) \cite{SORA} defined by \cite{EASAopinion2018}. 
The abilities of a drone after an emergency should be assessed and depending on the availability of the environment, a recovery procedure might be initiated. 
For the cases that a recovery is not possible, safe ditching might be needed to reduce the harm in the air or on the ground. 
 

\section{Motivation}

Improvement of the reliability of the flight is considered to be one of the main goals for integrating UAVs into civil airspace according to Unmanned systems roadmap by US Office of the Secretary of Defense, DoD \cite{UnmannedSystemsRoadmapDoD}. 
To achieve a safe flight is not an easy task considering the unknowns of the systems hardware, environment and possible system faults and failures to emerge. 
Also, increasing demand on cost effective systems, resulting in the smaller sensors and actuators with less accuracy, impose the software to achieve even more. 
The expectation that UAVs should be less expensive than their manned counterparts might have a hit on reliability of the system. Cost saving measures other than the need to support a pilot/crew onboard or decrement in size would probably lead to decrease in system reliability.

Under the research and development programs and initiatives identified by DoD in order to develop technologies and capabilities for UAS, the biggest chunk in control technologies is the health management and adaptive control with a budget of 74.3 M dollars. 
Other safety features such as validation and verification of flight critical intelligent software is the second with 57.8 M dollars \cite{UnmannedSystemsRoadmapDoD}. 

To achieve a safe flight is not an easy task considering the unknowns of the systems hardware, environment and possible system faults and failures to emerge. Also, increasing demand on cost effective systems, resulting in the smaller sensors and actuators with less accuracy, impose the software to achieve even more. The expectation that UAVs should be less expensive than their manned counterparts might have a hit on reliability of the system. Cost saving measures other than the need to support a pilot/crew on board or decrement in size would probably lead to decrease in system reliability.

Systems are often susceptible to faults of different nature. Existing irregularities in sensors, actuators, or controller could be amplified due to the control system design and lead to failures. A fault could be hidden thanks to the control action \cite{ducard2009fault}.

The widely used method to increase reliability is to use more reliable components and/or hardware redundancy. Both requires an increase in the cost of the UAS conflicting one of the main reasons of UAS design itself band consumer expectations \cite{angelov2012sense}. To offer solutions for all different foreseen categories of airspace, a variety of approaches should be considered. While hardware redundancy could cope with the failure situations of UAVs in the certified airspace, it may not be suitable for UAVs in open or some subsets of specific categories due to budget constraints. Analytical redundancy is another solution, may be not as effective and simple as hardware redundancy, but relies on the design of intelligent methods to utilize every bit of information on board aircraft wisely to deal with the instances.  

\section{Contribution}

This work introduces an end-to-end design to diagnose faults on drones. 
The discussion starts by fault tolerant control systems in general and is followed by two different methods for fault detection and diagnosis, model-based and data-driven approaches. 
Then, as a data-driven approach, SVM is introduced. 
SVM, being a supervised classification method, requires labeled data, both faulty and nominal. 
This can be achieved via simulations or real flights. Since a number of works discuss the fault detection \& diagnosis (FDD) based on simulated measurements and a few utilizes real flight data, we have selected to issue real flights to include many effects which is usually ignored, such as the controller's effects on the diagnosis.
Open-source autopilot, \emph{Paparazzi}, is utilized to inject faults during flights and is the next to be explained briefly. 
After that, modifications to autopilot control system and modes is discussed. 
Labeling data is done offline after the flight and is explained under the section post-processing of data after the flight. 
Since labeled data and the method is presented to the reader at that point, we proceed to apply the SVM classification to labeled flight data. 
In the following section, three different phases (training, tuning, evaluation) of SVM application to classification are explained with a focus on drone actuator fault detection. 
Finally, tuned classifiers are evaluated for two types of faults (loss of effectiveness and control surface stuck) and the results are given.