%% This is an example first chapter.  You should put chapter/appendix that you
%% write into a separate file, and add a line \include{yourfilename} to
%% main.tex, where `yourfilename.tex' is the name of the chapter/appendix file.
%% You can process specific files by typing their names in at the 
%% \files=
%% prompt when you run the file main.tex through LaTeX.

\chapter{Introduction}

Lately, the popularity and reachability of Unmanned Aircraft System (UAS) have risen steeply thanks to the advancements in electronic components and their decrease in cost. 
This accelerating trend towards small but capable flying vehicles is extending the limits of hardware and software potential in industry and academia.
Regulators from the aviation community have become increasingly concerned by the number of incidences of drones flying close to civil aircraft and airports. 
With the advent of the new era of UAS, different institutions around the world are addressing safe integration of UAS into airspace \cite{baskaya2016flexible}, specifically the National Aeronautics and Space Administration (NASA) 
\cite{kopardekarunmanned} and the Federal Aviation Administration (FAA) \cite{FAA_UASintegration} in the US, European Aviation Safety Agency (EASA) \cite{A_NPA_EASA2015} in Europe and international bases such as International Civil Aviation Organization (ICAO) \cite{ICAO_Circular}.
The U-Space concept in Europe (UTM in the US, which aims at enabling safe integration into airspace), provides the insight by showing in their roadmaps that the level of drone automatization and the level of drone connectivity (drone to drone and drone to infrastructure) will guide the pace of the U-Space services (U1 to U4) \cite{undertaking2017u}. 
These enablers will allow intelligent agents to share information and automate complex procedures in case of emergencies. 
Thus, future aviation will inevitably move towards automatization. 
Here we propose a concept that will contribute to the automatization of drones, to make them more intelligent and thus contribute to their safer integration into the airspace. 
These awareness abilities allow the mitigation of risks in accordance with the risk assessment procedures offered by JARUS (SORA) \cite{SORA} defined by \cite{EASAopinion2018}. 
The functioning abilities of a drone following an emergency should be assessed and, depending on the availability of the environment, a recovery procedure may be initiated. 
For cases in which recovery is not possible, safe ditching may be required in order to reduce the potential harm in the air or on the ground. 

\section{Motivation}

Improving the flight reliability is considered to be one of the main concerns for integrating UAVs into civil airspace, according to the Unmanned Systems Roadmap by the US Office of the Secretary of Defense, DoD \cite{UnmannedSystemsRoadmapDoD}. 
Achieving safe flight is not a straightforward task, considering the multiplexity of unknowns in the system hardware, environment, and possible system faults/failures yet to emerge. 
The expectation that UAVs should be less expensive than their manned counterparts effects their reliability. Cost saving measures --- other than the need to support a pilot/crew onboard or a reduction in size --- may lead to a decrease in system reliability.

Under the research and development programs and initiatives identified by the DoD to develop technologies and capabilities for UAS, the greatest area of control technologies is health management and adaptive control, with a budget of 74.3 m dollars. 
Other safety features such as the validation and verification of flight critical intelligent software are the second area, with 57.8 m dollars \cite{UnmannedSystemsRoadmapDoD}. 

Systems are often susceptible to faults of differing nature. Existing irregularities in sensors, actuators or controller may be intensified due to the control system design and lead to failures. A fault may be hidden due to the control action \cite{ducard2009fault}.

The most widely-used method to increase reliability is to increase the use of more reliable components and/or hardware redundancy. Both require an increase in the cost of the UAS, conflicting with one of the main reasons of UAS' popularity \cite{angelov2012sense}. In order to provide solutions for the different foreseen categories within the airspace, a variety of approaches should be considered. While hardware redundancy may cope with failure situations of UAVs in certified operations, it may not be suitable for UAVs in open category operations or some operations in specific category, due to budget constraints. Analytical redundancy is another solution. This may be not as effective and straightforward as hardware redundancy, but relies on the design of intelligent methods in order to utilize the information on board the aircraft to deal with all potential circumstances.  

\section{Contribution}

The integration of drones into the airspace requires the introduction of ingenious design to ensure safe solutions for drones. This necessitates intelligent approaches to address all potential emergency situations. One of these aspects is to ensure safe flight by designing fault detection and diagnosis systems using less expensive avionics, common in a vast number of drones.
The hardware limitations for these small vehicles point to the utilization of \emph{analytical redundancy} rather than the usual practice of hardware redundancy in manned aviation. 
For that purpose, we introduce an end-to-end design to achieve data-driven fault diagnosis for control surface faults on drones.
In the course of this thesis, machine learning practices are implemented in order to diagnose faults on a small fixed-wing UAV, and to avoid the burden of accurate modeling required in model-based fault diagnosis. 
We aim to design a classifier via SVM to solve fault detection and diagnosis (FDD) as a classification problem for drones with actuator faults.
All data and code are available in the code sharing and versioning system, \emph{Github}. 

In this thesis, fault classification simulations are investigated under two main sections: first, the classification of faults based on simulated flight measurements; and second, the classification of faults based on real flight data. 

For the classification on data generated from simulations, a model of a MAKO UAV \cite{baskaya2017flight} is simulated.
Sensor measurements (accelerometer and gyro data) have been calculated using information on the drone's motion and the specifications of the sensors. 
Generated data is usually more structured compared to real flight data. 
In this preliminary application of SVM to fault diagnosis, we started with simpler problem and used data generated from models.
Results show that the SVM classifier was accurate and fast in diagnosing faults on the control surfaces, with a classification accuracy of $99.999\%$.

Accurate results on the classification of faults with generated data have thus encouraged us to further investigate fault detection with real flight data. 
Since SVM is a supervised classification method, labeled data is necessary to train the algorithm. Real flights have therefore been arranged in order to generate faulty flight data by manipulating the open source autopilot, \emph{Paparazzi}.  
Training is held offline due to the need for labeled data and the computational burden of the tuning phase of the classifiers. 
Two main types of faults have been investigated; the control surface stuck fault and the loss of effectiveness of the elevon. Results indicate that the control surface stuck fault can be detected relatively easily with the data from three gyros and three accelerometers, compared to the loss of effectiveness (efficiency). 
The results show that over the flight data, SVM yields an F1 score of 0.98 for the classification of the control surface stuck fault. 
The addition of features to accommodate previous measurements improve the classification performance for tuned classifiers while the performance for untuned classification deteriorates. 
Classification performs poorly for the loss of efficiency faults, especially for the smaller ineffectiveness values. 
For the loss of efficiency fault, some feature engineering --- involving the addition of past measurements --- is needed in order to attain the same classification performance.

A very promising result is discovered when \emph{spinors} are used as features instead of angular velocities. 
Result show that by using spinors for classification, there is a significant improvement in classification accuracy, especially when the classifiers are untuned. Using spinors and a Gaussian Kernel, the untuned classifier gives an F1 score of 0.9555, which is 0.2712 when the gyro measurements are used as features.

In general, this work shows that SVM gives a satisfactory performance for the classification of faults on the control surfaces of a drone using flight data.

\section{Thesis Outline}

\textbf{Chapter 2} presents the state of the art on the following topics: 
\begin{itemize}
\item{the integration of drones into the airspace;}
\item{\emph{Paparazzi} autopilot;}
\item{fault tolerant control systems with a focus on fault detection and diagnosis;}
\item{machine learning and artificial intelligence in general.}
\end{itemize}
The safe integration of drones into the airspace is the motivation for this thesis. To ensure this safe integration, the design of vehicle health management systems is crucial. Fault detection and diagnosis plays an important role in vehicle health management, and we provide a literature review of this topic. In this thesis, the method selected to detect and diagnose faults is Support Vector Machines (SVM), which is a machine learning method. Thus, the current state of the art for machine learning is also provided. \emph{Paparazzi} autopilot has been used to realize flights to generate faulty data. This is also introduced  in here.\\
\textbf{Chapter 3} focuses on equations of motion of an aircraft. In this chapter, equations for translational and rotational motion are discussed separately in detail. After the equations are derived for a generic aircraft, calculation of forces and moments which are specific to an individual drone is presented. Then, modeling accelerometer and gyro data is explained. Modeling faults in the control surfaces of an aircraft is also discussed in this chapter.\\
\textbf{Chapter 4} is dedicated to machine learning algorithms with a focus on their implementations. After the general practices of machine learning applications are presented, Support Vector Machines (SVM) is discussed in more detail.\\
\textbf{Chapter 5} focuses on the results of fault classification results under two main sections: classification of faults based on simulated  flight measurements and classification of faults based on real flight data. In the first part, flight data is simulated using the mathematical equations explained in Chapter 3. SVM algorithm is trained with the simulated data to classify the faulty and nominal phases of the simulated flight. The second part starts with explanations on faulty flight experiments realized. \emph{Paparazzi} Autopilot has been modified to introduce faults to control surfaces during the flight. Then, classification of control surface stuck and loss of effectiveness faults have been investigated separately.\\
\textbf{Chapter 6} concludes the thesis with an outlook on the contributions and results acquired during the thesis.