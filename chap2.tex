%% This is an example first chapter.  You should put chapter/appendix that you
%% write into a separate file, and add a line \include{yourfilename} to
%% main.tex, where `yourfilename.tex' is the name of the chapter/appendix file.
%% You can process specific files by typing their names in at the 
%% \files=
%% prompt when you run the file main.tex through LaTeX.

%\makeglossaries
\newacronym{utm}{UTM}{Unmanned Aircraft System Traffic Management}
\newacronym{uav}{UAV}{Unmanned Air Vehicle}
\newacronym{uas}{UAS}{Unmanned Aircraft System}
\newacronym{nas}{NAS}{National Air Space}
\newacronym{fpv}{FPV}{First Person View}
\newacronym{vll}{VLL}{Very Low Level}
\newacronym{sme}{SME}{Small and Medium Enterprise}
\newacronym{tcas}{TCAS}{Traffic Collision Avoidance System}
\newacronym{gcs}{GCS}{Ground Control Station}
\newacronym{icao}{ICAO}{International Civil Aviation Organization}
\newacronym{faa}{FAA}{Federal Aviation Administration}
\newacronym{nasa}{NASA}{National Aeronautics and Space Administration}
\newacronym{easa}{EASA}{European Aviation Safety Agency}

\chapter{Safe Integration of Drones into Airpace}


Air safety authorities are forced to develop regulations for UAS due to incidents 
disturbing public safety and demands from companies who desire to utilize them. 
There has been numerous studies, from both the FAA in US and EASA in Europe, 
but none of them decided on a regulations set for the UAVs to satisfy. Improvement 
of the reliability of the flight is considered to be one of the main obstacles for 
integrating UAVs into civil airspace. To achieve a safe flight is not an easy task 
considering the unknowns of the systems, environment and possible system faults 
and failures to emerge. To tackle the safety challenges and help the regulation 
development, NASA is currently carrying out a four years research program (up to 2019) 
to enable Unmanned aircraft traffic management solutions which are structured yet 
flexible when needed. To ensure safety, this integration needs to be achieved through 
airspace management and UAS reliability.

The preliminary airspace designs, like the one proposed by Amazon, identify different 
zones depending on the UAS capabilities, population density and altitude. Plus, 
different national rules and their progressive refinement pushes to cope with a variety 
of requirements. Open source and modular architectures are key to adapt these 
requirements. As a specific example, NASA's UTM builds, later to be refined by FAA, 
make modularity essential for UAS software to follow their evolution. 

Concerning reliability, current regulations focus on flight constraints but they might be 
expected to involve regulations on software and hardware components as well. 
In such case, the increased cost will be inevitable for the demands of certification. 
This could put too many constraints on UAS manufacturers who desire to access the G airspace. 

In Europe, Regulation (EC) No 216/2008 mandates the European Aviation Safety Agency 
to regulate Unmanned Aircraft Systems (UAS) and in particular Remotely Piloted Aircraft 
Systems (RPAS), when used for civil applications and with an operating mass of 150 Kg or more.

\section{UAVs populating airspace}\label{ch2:certificationOfAnalyticalApproaches}

The cost effectiveness and reachability of commercial off-the-shelf elements, and 
shrinking size of electronics serve as a perfect environment for small flying vehicles 
to emerge. Although inherited as military purposes in its infancy, nowadays \gls{uas} 
are becoming efficient platforms for scientific/commercial domains. They offer 
benefits in terms of cost, flexibility, endurance as well as realizing missions that 
would be impossible with a human onboard. 
Increasing usage of these vehicles for a variety of missions, such as defense, 
civilian tasks including transportation, communication, agriculture, disaster mitigation 
applications pushes demand on the airspace. Furthermore, this congestion is predicted 
to accelerate with the growing diversity of these systems. 

Commercial advantages, offered by these efficient systems, are already targeted by 
big companies worldwide, specifically in the US. The airspace regulatory authorities 
seem to be squeezed in between the companies, demanding a fast as possible 
access to airspace, and the concerns of the public about potential privacy breaches, 
safety and liability issues \cite{droneDisasters,droneImageProblem}. Even with today's 
strictly regulated airspace, reported occurrences show that there are hurdles to solve 
before a further integration of \gls{uas} to airspace.

Advent of the new era of \gls{uas} seems to be hold by an unseen barrier of lack of 
regulatory framework for now. Different institutions all over the world, specifically \gls{nasa} 
and faa in US, easa in Europe and international bases such as \gls{icao} are addressing 
safe integration of \gls{uas} in airspace. Although the approaches of regulatory bodies 
may vary, the aim remains the same: safe integration as soon as possible. 

The tackles of safety during integration of \gls{uas} to airspace refer to different technical 
and organizational aspects including but not limited to control of traffic in segregated and 
non-segregated airspace, reliable communication, robust control of the \gls{uav}, trajectory planning, detect\&avoid.


\section{Modularity}\label{ch2:modularity}

The current evolving nature of regulations and the variety of organizations in charge 
of the airspace rule making calls for flexible solutions to cope with these fruitful environments. 

\subsection{Airspace Categorization}
The \gls{uas} in the \gls{nas} project points to a performance-based routine access to 
all segments of the national airspace for all unmanned aircraft system classes, after the 
safety and technical issues are addressed thoroughly. As a start, \gls{nasa} and faa 
seem to have a short term goal to integrate \gls{uas} in low-altitude airspace as early 
as possible. They further aim to accommodate increased demand safely, efficiently in 
the long term. \gls{nasa} and faa seem to handle the airspace above 500 feet and the 
one below separately. easa, tasked by the European Union, is planning a risk based approach, 
accommodating the \gls{uas} in the airspace under three different categories, low risk, 
specific and high risk. Both regulators seem to categorize the airspace and scale regulatory 
needs according to some criteria. To answer different needs of different categories, flexibility 
given by the high level of modularity of open source autopilot systems will be a handy tool. 

\subsection{National Regulations}
Circulation of \gls{uas} internationally is somewhat prevented by the Chicago convention 
unless an agreement holds between Contracting States \cite{A_NPA_EASA2015}. \gls{icao} is aiming 
to develop international standards and recommended practices to which the member states 
could refer to when developing their national civil aviation regulations. Even though a similar 
base is aimed, national aviation legislations will not be the same because of the different 
expectations of nations about \gls{uas} aviation. 

\subsection{Accommodating evolution of regulations}
Prescriptive rules seem to cause some difficulties since the technical area on \gls{uas} 
systems develop rapidly \cite{A_NPA_EASA2015}. Innovations both on the aircraft and the operation 
type of \gls{uas} will accelerate especially after the regulations are set. Thus, regulatory 
bodies call for refinable operational requirements and system architectures to evolve into 
a safer and efficient integration of \gls{uas} into airspace. The systems to cope with the 
regulations should also be modular and flexible in order not to be superseded by the 
innovations in the area. 
Thus, the aviation regulatory bodies aim to achieve designs with flexibility where possible, 
structure where needed. Having flexible hardware and software points to modularity, which 
is pretty much best supported via open source systems.



\section{Congestion management}\label{ch2:congestion}

According to \textit{UAV Factory}, one of the large European \gls{uas} companies, 
``The future of the UAV industry is likely to be shaped by airspace congestion'' 
\cite{europe_report_civilian_drone}. Indeed, high level airspaces are getting crowded 
and large scale solutions, such as NextGen (US) or SESAR-JU (EU), are necessary to 
increase airspace capacity while maintaining the current safety levels. However, there is 
no such management solution existing for \gls{vll}. Yet, large projects like Amazon's 
Prime Air and Google's Project Wing are already waiting to populate the \gls{vll} airspace.	
	
Part of the congestion management problem is to avoid conflicts, and more importantly 
collisions, between \gls{uas} through strategic deconfliction and safety nets.
Another mission of the congestion management system is to make sure that \gls{uas} 
do not go where they are not supposed to go, thus requiring geofencing. In order to implement 
the previously mentioned systems, the \gls{uas} autopilot needs to be able to perform complex 
operations, e.g. static waypoints following is likely to be insufficient.

In the following, we divide these issues into four topics of interest: 4-D trajectory management, 
geofencing, safety nets, complex operations.
	
\subsection{4-D Trajectory Management}
As noted in \cite{erzberger_4D_2002}, 4-D trajectories will be central in future airspace 
management methods. The principle of 4-D trajectory management is to have every 
\gls{uas} broadcast its trajectory up to some time horizon and receive \gls{utm}'s 
clearances under the form of trajectories. The trajectory information contains a path, 
the series of points through which the \gls{uas} will pass, and times associated to each of these points. 
Thanks to this information, the idea is to perform pro-active deconfliction, as 
explained by Thomas et al. in \cite{thomas_4D_2015}. In clear, it implies that \gls{utm} 
detects future conflicts along the trajectories of all \gls{uas} and deconflicting them 
as safely and early as possible. 
				
\subsection{Safety Nets}
Trajectory deconfliction is the first step to manage congestion, however safety nets 
are also part of the congestion management. Indeed, safety nets such as self-separation 
and collision avoidance allow \gls{uas} to fly close to each other while preserving an 
acceptable safety level.
		
\subsection{Geofencing}
Keeping \gls{uas} away from each other is an important point. But keeping them 
out of forbidden areas is also crucial. Geofencing allows determining no-fly zones 
where the \gls{uas} should not enter.

To accommodate land owners while managing traffic and limiting congestion, Foina et al. 
\cite{foina_air_parcelle_2015} proposed a participative dynamical airspace management 
method: the air-parcel model. It allows land owners to authorize/forbid flights over their 
lands through a web interface. However, this type of approach asks from the \gls{uas}s 
to be able to handle dynamical geofencing. Plus, though initially this model considers only 
cuboid parcels, the need for more precise airspace shapes may emerge making 3-D geofencing a need.
	
\subsection{Complex Operations}
Having 4-D trajectory management, safety nets and geofencing is useless if the \gls{uas}s 
cannot follow the instruction from \gls{utm} regarding these tools. Indeed, new \gls{utm} 
paradigms imply being able to change flight plans dynamically to answer to \gls{utm} demands. 
In \cite{wargo_complex_2015} two types of complex operations examples are mentioned: 
space transition corridors and temporary flight restriction. Both these airspace management 
methods require from the \gls{uas} to be able to modify its flight plan according to new \gls{utm} instructions.
		

\section{Reliability}\label{ch2:reliability}

Improvement of the reliability of the flight is considered to be one of the main goals for 
integrating military \gls{uav}s into civil airspace according to Unmanned systems roadmap 
by US Office of the Secretary of Defense, DoD \cite{UnmannedSystemsRoadmapDoD}. 
Compared to manned counterparts, \gls{uav}s experienced failures with a frequency of two 
orders of magnitude more in the military domain.  Although this changed last years with 
the technological improvements, making the \gls{uav}s as reliable as early manned military 
aircraft, it seems not enough from the DoD perspective. This can be realized by checking 
the biggest chunk of control technologies budget for research and development, which is 
health management and adaptive control.

To achieve a safe flight is not an easy task considering the unknowns of the systems hardware, 
environment and possible system faults and failures to emerge. Also, increasing demand on 
cost effective systems, resulting in smaller sensors and actuators with less accuracy, 
impose the software to achieve even more. The expectation that \gls{uav}s should be less 
expensive than their manned counterparts might have a hit on reliability of the systems. 
Cost saving measures other than the need to support a pilot/crew onboard or decrement 
in size would probably lead to decrease in system reliability. 

\subsection{\gls{sme}s and Certification Costs}

Utilizing drones for quicker and cheaper deliveries could be rewarding for \gls{sme}s 
since cost per mile of a drone is less then 1 / 30 of the average diesel truck. Being an 
early bird might put the \gls{sme}s in an advantageous position, considering the increase 
in the capabilities of the drones with inevitable acceleration thrusted by research activities 
and their widened application areas. 

Nevertheless, the fairly cheap access to drones and their relatively cheap utilization cost 
does not seem to be enough to put them to air right now due to the heavy cost of certification 
and regulatory hurdles \cite{UAVreliabilityStudy}. In this concern, capable open source solutions 
could be a good way to loosen the tie.  Otherwise, \gls{sme}s, an important factor in drone 
business might not survive. Even worse, they might operate them without relevant permission, 
scarifying a substantial fine as reported by Civil Aviation Authority (CAA). This will compromise 
security in the system contradicting the hopes for reliable integration of drones into airspace. 

\subsection{Individuals and Education}

Individuals, as well as \gls{sme}s, suffer from the same budget constraints. Personal \gls{uav} 
usage counts for a substantial amount of the drone ecosystem.  Both US and European 
authorities mention the importance of individuals in utilizations of drones. There is a community 
with a passion of aviation and potential, but most probably not very experienced.  

\subsection{Flight Heritage for Risk Assessment}

Drone industry being extremely innovative, technical developments could supersede the 
prescriptive rules as regulations. Thus a solution might be to follow a risk based approach 
rather that to have strict rules to cope with. Predicted regulations in Europe seems to evolve 
under different categories dedicated to specific operation risks. Flight heritage and occurrence 
reporting is expected to be an inevitable part of safety risk assessment to achieve reliable flight. 

\subsection{Support for real time planning and onboard vehicle automation}
To access low-altitude airspace with the use of small unmanned aircraft safely, an important 
ability could be to implement real-time planning and on-board vehicle automation. Amazon 
offers that this approach will allow some flexibility to adapt to variable situations such as 
weather changes, severe winds or any other emergency needs.

To conclude, the introduction of \gls{uas}s in the \gls{vll} airspace comes with numerous challenges. 
Though various ways to address them have been proposed, there is still no certainty about how it will be done in the end.

One sure thing is that it will involve two main actors: \gls{utm} and \gls{uas}s. This work focused on the later.	

\section{Certification of analytical approaches NOT FINISHED }\label{ch2:certificationOfAnalyticalApproaches}


Due to the lack of the modeling for interaction in-between different FPE  (Flight Parameter Estimation), FDD and FTC modules in simulations, leaves them free from practical limitations, offering a long way to go for these analytical methods to be certified. 

[REF : certificationOfFDD]
[REF : clearanceOptimBased]


safety issues

http://www.techrepublic.com/article/12-drone-disasters-that-show-why-the-faa-hates-drones/

%FP7_RECONFIGURE REF : FP7RECONFIGUREgeneral page : 978 starting with 4 980 section 4. VERIFICATION AND VALIDATION

Regulations EU : https://easa.europa.eu/unmanned-aircraft-systems-uas-and-remotely-piloted-aircraft-systems-rpas

The increased cost will be inevitable for the demands of certification. REF : UAVreliabilityStudy

This increment could be even beyond expectations that it could lead to a cancellation of project:  REF : http://www.defenseindustrydaily.com/euro-hawk-program-cleared-for-takeoff-03051/

Regulations US : 

US Delivery Drones (News from Guardian http://www.theguardian.com/technology/2015/nov/06/alphabet-and-facebook-compete-with-secret-drone-plans)
--------------------
Even if the companies solve the technical challenges of keeping drones aloft for long periods, sharing data via lasers and serving city-sized areas, both Alphabet and Facebook still face regulatory hurdles. Neither company has been granted a waiver to the current FAA blanket ban on the commercial operation of unmanned aircraft.

Alphabet has applied for an exemption, called ?333? after a section of the FAA regulations, for its Project Wing delivery drones, but that is yet to be granted and in any case would only clear operation to a maximum height of 400ft. Google has also been testing its delivery drones in the US under a Certificate of Waiver or Authorization (COA) with Nasa, which permitted flights intended to help Nasa develop an automated air traffic control system for low-flying drones. This would not apply to drones flying far above other manned and unmanned aircraft.

?There?s not a lot to run into between 60,000 and 90,000 feet,? says Cummings. ?But I?m sure regulators would be deeply suspicious if Google and Facebook were flying these over the US.?

Facebook and Alphabet would not comment.

What's really standing in the way of drone delivery?
---------------------------------------------------------
http://www.theverge.com/2016/1/16/10777144/delivery-drones-regulations-safety-faa-autonomous-flight


 SORASORA SORA SORA Fig.~\ref{fig:sora_barriersSmallerVersion}
\begin{figure}
\begin{center}
\includegraphics[width=24cm,angle=90,origin=c]{figures/sora_barriersSmallerVersion}    % The printed column width is 8.4 cm.
\caption{SORA structure: threats, threat barriers, hazard, harm barriers, harms} 
\label{fig:sora_barriersSmallerVersion}
\end{center}
\end{figure}

