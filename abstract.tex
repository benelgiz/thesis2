% $Log: abstract.tex,v $
% Revision 1.1  93/05/14  14:56:25  starflt
% Initial revision
% 
% Revision 1.1  90/05/04  10:41:01  lwvanels
% Initial revision
% 
%
%% The text of your abstract and nothing else (other than comments) goes here.
%% It will be single-spaced and the rest of the text that is supposed to go on
%% the abstract page will be generated by the abstractpage environment.  This
%% file should be \input (not \include 'd) from cover.tex.

This new era of small UAVs currently populating the airspace introduces many safety concerns, due to the absence of a pilot onboard and the less accurate nature of the sensors. 
This necessitates intelligent approaches to address the emergency situations that will inevitably arise for all classes of UAV operations as defined by EASA (European Aviation Safety Agency). 
Hardware limitations for these small vehicles point to the utilization of \emph{analytical redundancy}, rather than to the usual practice of hardware redundancy in manned aviation. 
In the course of this study, machine learning practices are implemented in order to diagnose faults on a small fixed-wing UAV to avoid the burden of accurate modeling needed in model-based fault diagnosis. 
A supervised classification method, SVM (Support Vector Machines) is used to classify the faults. 
The data used to diagnose the faults are gyro and accelerometer measurements. 
The idea to restrict the data set to accelerometer and gyro measurements is to check the method's classification ability, with a small and inexpensive chip set and without the need to access the data from the autopilot, such as the control input information. 
% To access information from the autopilot used was quite easy since it is open-source, but still a solution for a wider community is pursued involving the systems equipped with closed solutions. 
% Although serving for flexibility, lacking control input information from the autopilot introduces challenges mainly due to lack of knowledge about the controller's compensation for the fault. 
This work addresses the faults in the control surfaces of a UAV. 
More specifically, the faults considered are the control surface stuck at an angle and the loss of effectiveness. 
% Since SVM is a supervised machine learning algorithm, labeled data is needed to accomplish the training of the classifier. 
First, a model of an aircraft is simulated. This model is not used for the design of Fault Detection and Diagnosis (FDD) algorithms, but is instead utilized to generate data. 
Simulated data are used instead of flight data in order to isolate the probable effects of the controller on the diagnosis, which may complicate a preliminary study on FDD for drones.
The results show that for simulated measurements, SVM gives very accurate results on the classification of the loss of effectiveness faults on the control surfaces. 
These promising results call for further investigation so as to assess SVM performance on fault classification with flight data.
Real flights were arranged to generate faulty flight data by manipulating the open source autopilot, \emph{Paparazzi}. 
All data and the code are available in the code sharing and versioning system, \emph{Github}. 
Training is held offline due to the need for labeled data and the computational burden of the tuning phase of the classifiers. 
Results show that from the flight data, SVM yields an F1 score of 0.98 for the classification of control surface stuck faults.
For the loss of efficiency faults, some feature engineering, involving the addition of past measurements is needed in order to attain the same classification performance. 
A promising result is discovered when \emph{spinors} are used as features instead of angular velocities. 
Results show that by using spinors for classification, there is a vast improvement in classification accuracy, especially when the classifiers are untuned. Using spinors and a Gaussian Kernel, an untuned classifier gives an F1 score of 0.9555, which was 0.2712 when gyro measurements were used as features.
In summary, this work shows that SVM gives a satisfactory performance for the classification of faults on the control surfaces of a drone using flight data.
